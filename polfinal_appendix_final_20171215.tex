%% LyX 2.2.3 created this file.  For more info, see http://www.lyx.org/.
%% Do not edit unless you really know what you are doing.
\documentclass[english]{article}
\usepackage[T1]{fontenc}
\usepackage[latin9]{inputenc}
\usepackage{geometry}
\geometry{verbose,tmargin=1in,bmargin=1in,lmargin=1in,rmargin=1in}
\usepackage{amsmath}

\makeatletter

%%%%%%%%%%%%%%%%%%%%%%%%%%%%%% LyX specific LaTeX commands.
%% Because html converters don't know tabularnewline
\providecommand{\tabularnewline}{\\}

\makeatother

\usepackage{babel}
\begin{document}

\appendix
In this appendix, we formally derive our main PDE system (1.1) as
the continuum limit of a discrete agent-based model of crime with
police patrol. This discrete model is an extension of the two-agent
Short model in {[}11{]} by introducing a police agent which patrols
strategically towards crime hotspots and affect crime events indirectly
by deterring potential criminals at the same location. 

The idea of police enforcement in the vicincity of crime hotspots
is not new. Variants of such a strategy is broadly known as ``cops-on-the-dots''
(cf. {[}11,22,33{]} and the references therein). In some modelling
efforts (e.g. in {[}33{]}), a ``cops-on-the-dots'' strategy could
entail a top-down coordination among police officers equipped a bird's
eye view of the current distrubution of criminal events or density,
so as to optimize police deployment against crime hotspots. In contrast,
with the consideration that information about attractiveness of locations
and distribution of criminals (who are yet to commit a crime) cannot
be easily determined with precision, we instead take a conservative
approach to the ability of police officers to focus on hotspots and
assume that each police officer move independently of each other as
random walkers, but biased towards locations with higher attractiveness
in a way analogous to criminals. This is more in line with the strategy
presented in {[}11, Jones-Brantingham-Chayes{]}, which coincidentally
was also called ``cops-on-the-dots'' by the authors. Our approach
in this paper gives rise to a non-linear diffusion term in the police
PDE (a feature that is also present in {[}11{]} and {[}22{]} but not
{[}33{]}), but with both the ``random walker'' and ``cops-on-dot-dots''
strategies in {[}11{]} as special cases. In addition, our model also
allows for a different relative speed of movement for police and a
focus parameter that describes how motivated police officers are towards
roaming to more crime-attractive locations.

As for how police presence affects crime, we consider the influence
to be indirect, and consider only its effect on criminals but not
attractiveness. More precisely, we only model for a deterrance effect
(compare ``behavior modification'' in {[}11{]}), meaning that in
the presence of police agents, the number of criminals at the same
location will be reduced. This reduction could occur in a way analogous
to prey-predator interaction, in which criminals that are already
present go home upon contact with police officers instead of continuing
to roam. This is the approach used in {[}11{]}. Another way is to
assume that the rate new criminal emerges at each location is reduced
in proportion to the number of police agents present. In other words,
the police's deterrance effect precludes some potential criminals
to enter into the system altogether. While we are interested in both
approaches, we find the latter approach mathematically more tractable
and it is thus employed in this paper.

We now proceed briefly review previous work and highlight a few unique
details and explain how the three-component PDE system (1.1) was derived.
The full details of both the discrete model and the formal continuum
limit derivations are availble in {[}short et al, chayes et al, my
thesis...{]}

\subsection*{The Discrete Agent-based Model}

Our model is a three-agent model that couples the dynamics of two
human agents: criminals $\mathrm{N}$ and police officers $\mathrm{R}$
that move on a 2D rectangular lattice (with spacing $\ell)$ with
an underlying scalar field $\mathrm{A}$ known as attractiveness.
The attractiveness field $\text{A}$ of any location $\underbar{x}=(i,j)$
in discrete time $\text{t}$ serves the following two purposes. Firstly,
$\mathrm{A}(\underbar{x},\text{t})$ gives a measure of risk of crime,
quantized by the likelihood whereby each criminal occupying the location
would commit burglary and then return home:
\[
p_{v}(\underbar{x},t)=\frac{\mathrm{A}(\underbar{x},\text{t})}{1+\mathrm{A}(\underbar{x},\text{t})}.
\]
Secondly, $\mathrm{A}(\underbar{x},\text{t})$ gives a measure of
attractiveness for both criminals and police officers in the four
neighboring locations $\underbar{x}'\in\{(i+1,j),\,(i-1,j),\;(i,j+1),\,(i,j-1)\}$
(which we would denote $\underbar{x}'\sim\underbar{x}$ from this
point onwards), to decide, in a stochastic manner, to move there or
not:
\[
p_{m}(\underbar{x}',\text{t};\underbar{x})=\frac{\mathrm{A}(\underbar{x}',\text{t})}{\sum_{\underbar{x}''\sim\underbar{x}}\mathrm{A}(\underbar{x}'',\text{t})},\qquad p_{k}(\underbar{x}',\text{t};\underbar{x})=\frac{\mathrm{A}^{k}(\underbar{x}',\text{t})}{\sum_{\underbar{x}''\sim\underbar{x}}\mathrm{A}^{k}(\underbar{x}'',\text{t})},\,\text{where }k\geq0\text{ is fixed}.
\]
Notice that in both cases we assume the movement is biased towards
higher attractiveness. The difference, and the novelty of this model,
lies in the fact that a power law is applied on attractiveness field
$\mathrm{A}$ to parametrize a degree of \emph{focus }of police patrol.
We summarize the effect of the parameter $k$ as follows: 
\begin{center}
\begin{tabular}{|c|c|}
\hline 
Range of $k$ & Effect on patrolling strategy\tabularnewline
\hline 
\hline 
$0$ & unbiased random walk\tabularnewline
\hline 
$(0,1)$ & less focussed than criminals\tabularnewline
\hline 
1 & ``mimicry'' movement\tabularnewline
\hline 
$(1,\infty)$ & more focussed than criminals\tabularnewline
\hline 
$\to\infty$ & deterministic: $p_{k}(\underbar{x}',t;x)=\begin{cases}
1 & \text{for }\underbar{x}'=\max_{\underbar{x}''\sim\underbar{x}}\mathrm{A}(\underbar{x}'',\text{t})\\
0 & \text{otherwise}
\end{cases}$\tabularnewline
\hline 
\end{tabular}
\par\end{center}

We assume as in {[}11,22,33{]} that the values of $\mathrm{A}$ and
$\mathrm{N}$ are updated at discrete time steps of $\delta\text{t}$.
A critical component of the discrete Short model {[}11{]} that contribute
to the existence of hotspot patterns is the decomposition of attractiveness
into static and dynamic components: $\mathrm{A}(\underbar{x},t)=\mathrm{A_{sta}}(\underbar{x})+\mathrm{A_{dyn}}(\underbar{x},\text{t})$,
and then modeling the phenomena of repeat and near-repeat victimization
respectively by introducing a feedback effect when crime events occur,
and a localized diffusion of the dynamic portion of the attractiveness
field:

\[
\text{A}(\underbar{x},\text{t}+\text{\ensuremath{\delta}t})=\left[(1-\epsilon)\text{\ensuremath{A_{dyn}}}(\underbar{x},\text{t})+\frac{\epsilon}{4}\sum_{\underbar{x}'\sim\underbar{x}}\text{\ensuremath{A_{dyn}}}(\underbar{x}',\text{t})\right](1-\omega)+\theta\text{N}(\underbar{x},\text{t})\text{A}(\underbar{x},\text{t})+\text{\ensuremath{A_{sta}}}(\underbar{x})
\]
where $\epsilon,\,\omega>0$ describe rates of diffusion (to model
localized spread of risk) and decay (to model fading memory), while
$\theta>0$ measures the effect on attractiveness by each crime event,
whose number is assumed to be proportional to $\text{N}(\underbar{x},\text{t})\text{A}(\underbar{x},\text{t})$.
In this paper, we also assume $\text{\ensuremath{A_{sta}}}(\underbar{x})\equiv\alpha$
to be spatially homogeneous.

We augment the two-component Short model by coupling the third police
equation to the criminal equation through the background criminal
reintroduction rate to $\Gamma-\nu\text{R}(\underbar{x},\text{t})$,
where $\Gamma>0$ is the spatially homogeneous rate in the absence
of police and $\nu>0$ a constant. We remark that setting $\nu=0$
recovers the Short the model (and also the corresponding equations
in the variants appearing in {[}11,22,33{]}), and our approach amounts
to using an inhomogeneous criminal reintroduction rate and positing
that the first order correction is due to police presence. For the
police equation, we rescale the time stepping of $\text{R}$ by a
factor of $\tau$ to distinguish the speed of movements of the two
human agents, i.e. $\text{R}$ is updated at discrete time steps of
$\tau_{u}\delta\text{t}$. The flow chart for programming the discrete
simulation would thus be necessarily broken into separete loops because
the Monte Carlo steps would not occur at the same discrete time values
in general, but we do not go into detail here as it is not the focus
of this paper.

Finally, with the form of biased random walk specified above for the
police officers, we may state the final form of our discrete model
as follows:

\begin{align*}
\text{A}(\underbar{x},\text{t+\ensuremath{\delta}t}) & =\left[(1-\epsilon)\left(\text{A}(\underbar{x},\text{t})-\alpha\right)+\frac{\epsilon}{4}\sum_{\underbar{x}'\sim\underbar{x}}\left(\text{A}(\underbar{x}',\text{t})-\alpha\right)\right](1-\omega\text{\ensuremath{\delta}t})+\theta\text{A}(\underbar{x},\text{t})\text{N}(\underbar{x},\text{t})+\alpha,\\
\text{N}(\underbar{x},\text{t+\ensuremath{\delta}t}) & =\sum_{\underbar{x}'\sim\underbar{x}}\text{N}(\underbar{x}',\text{t})\left(1-p_{v}(\underbar{x}',\text{t})\right)p_{m}(\underbar{x};\text{t},\text{\ensuremath{\underbar{x}}}')+\Gamma-\nu\text{R}(\underbar{x},\text{t}),\\
\text{R}(\underbar{x},\text{t+\ensuremath{\tau\delta}t}) & =\sum_{\underbar{x}'\sim\underbar{x}}\text{R}(\underbar{x}',\text{t})p_{k}(\underbar{x};\text{t},\underbar{x}').
\end{align*}
Note that under the summations, the order of the arguments $\underbar{x}$
and $\underbar{x}'$ of $p_{m}$ and $p_{k}$ as compared to the definition,
are correctly swapped as shown above. One would as a result observe
that, for $\text{N}$ and $\text{R}$, the calculation of the next
step depends on information of all the neighbors of each neighbor
$\underbar{x}'$ of $\underbar{x}$, which totals to nine points and
includes $\underbar{x}$ itself.

\subsection*{Formal Continuum Limit}

We define the continuous density functions $A,\,\rho,\,U$ on $[0,S]\times[0,\infty)$
by taking formal continuum limits of their discrete counterparts and
replacing the arguments by continuous variables: 
\[
\text{N(\ensuremath{\underbar{x}},\text{t})}/\ell^{2}\rightarrow\rho(x,t),\;\text{R(\ensuremath{\underbar{x}},\text{t})}/\ell^{2}\to U(x,t),\;\text{A(\ensuremath{\underbar{x}},\text{t})}\to A(x,t)\quad\text{as }\ell,\delta\text{t}\to0\text{ with }\ell^{2}/\delta\text{t}\text{ fixed.}
\]
The main purpose of this section is to provide details to the derivation
of the new PDE for $U$ in (1.1). We refer readers to {[}11,33, my
thesis, KWW{]} for details of the derivations of $A$ and $\rho$.

To this end, we need to apply the discrete Laplacian to {[}our discrete
equations above{]} to obtain expressions evaluated at $(\underbar{x},t)$
as follows. 

We use
\[
\triangle\text{A}^{k}(\underline{\text{x}}',\text{t})\equiv\frac{1}{\ell^{2}}\sum_{\underbar{x}''\sim\underbar{x}'}\left(\text{A}^{k}(\text{\ensuremath{\underline{x}}}'',\text{t})-\text{A}^{k}(\text{\ensuremath{\underline{x}}'},\text{t})\right)
\]
to rewrite the expected number police moving from each neighboring
points $\underbar{x}'$ to $\underbar{x}$ as s
\begin{align*}
\text{R}(\underbar{x}',\text{t})p_{k}(\underbar{x},\text{t};\underbar{x}') & =\text{A}^{k}(\underbar{x},\text{t})\cdot\left[\frac{\text{R}(\underbar{x}',\text{t})}{\ell^{2}\triangle\text{A}^{k}(\underbar{x}',\text{t})+4\text{A}^{k}(\underbar{x}',\text{t})}\right]\equiv\text{A}^{k}(\underbar{x},\text{t})\cdot\text{T}(\underbar{x}',\text{t})
\end{align*}
where $\text{T}$ is defined by the expression in square brackets.
Then, we sum for all $\underbar{x}'\sim\underbar{x}$ and apply the
discrete Laplacian again on $\text{T}$ to write
\[
\text{R}(\underbar{x},\text{t}+\tau_{u}\text{\ensuremath{\delta}t})=\text{A}^{k}(\underbar{x},\text{t})\cdot\sum_{\underbar{x}'\sim\underbar{x}}\text{T}(\underbar{x}',t)=\text{A}^{k}(\underbar{x},\text{t})\left(\ell^{2}\triangle\text{T}(\underbar{x},\text{t})+4\text{T}(\underbar{x},\text{t})\right)
\]

To derive an equation for time derivative of police density $U$,
we observe that 
\begin{align*}
\tau_{u}\frac{\partial U}{\partial t} & \approx\frac{\text{R}(\underbar{x},\text{t}+\tau_{u}\delta\text{t})-\text{R}(\underbar{x},\text{t})}{\ell^{2}\delta\text{t}}\\
 & =\frac{\text{A}^{k}(\underbar{x},\text{t})\triangle\text{T}(\underbar{x},\text{t})}{\delta\text{t}}+\frac{4\text{A}^{k}(\underbar{x},\text{t})\text{T}(\underbar{x},\text{t})-\text{R}(\underbar{x},\text{t})}{\ell^{2}\delta\text{t}}
\end{align*}
which we assume to converge as $\ell,\delta\text{t}\to0$ with $\ell^{2}/\delta\text{t}=4D$
fixed to define the left-hand-side. We may now drop the arguments
$(\underbar{x},\text{t})$ from this point onwards for clarity of
presentation.

As $\ell,\text{\ensuremath{\delta}t}\to0$, we estimate that 
\[
\text{T}=\frac{\text{R}}{4\text{A}^{k}}\left(1+\frac{\ell^{2}}{4\text{A}^{k}}\triangle\text{A}^{k}\right)^{-1}\approx\frac{\text{R}}{4\text{A}^{k}}\left(1-\frac{\ell^{2}}{4\text{A}^{k}}\triangle\text{A}^{k}\right).
\]
Hence, by replacing $\ell^{2}=4D\text{\ensuremath{\delta}t}$ and
$R/\ell^{2}\to U$, we obtain these two formal limits:
\begin{align*}
\frac{\text{T}}{\text{\ensuremath{\delta}t}} & \approx\frac{\text{R}}{4\text{A}^{k}\delta\text{t}}\to D\cdot\frac{U}{A^{k}},\text{ and}\\
\frac{4\text{A}^{k}\text{T}-\text{R}}{\ell^{2}\text{\ensuremath{\delta}t}} & \approx-\frac{\text{R}}{4\text{A}^{k}\delta\text{t}}\triangle\text{A}^{k}\to-D\cdot\frac{U}{A^{k}}\Delta A^{k}.
\end{align*}
We may then substitute these into {[}$\partial U/\partial t$, ?{]}
and take limits to give a PDE for $U$: 
\[
\tau_{u}\frac{\partial U}{\partial t}=D\left(A^{k}\Delta\left(\frac{U}{A^{k}}\right)-\frac{U}{A^{k}}\Delta A^{k}\right)
\]
By renaming $k=q/2$, the right-hand-side above can be shown via Leibniz
rule to be equivalent to
\[
D\nabla\cdot\left(\nabla U-\frac{qU}{A}\nabla A\right).
\]

\end{document}
