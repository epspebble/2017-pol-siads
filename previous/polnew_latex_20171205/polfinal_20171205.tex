\documentclass{article}%
\usepackage{pbox}
\usepackage{graphicx}
\usepackage{amsmath}
\usepackage{amssymb}
\usepackage{subfigure}
\usepackage{verbatim}
\usepackage[font={small,it}]{caption}
\usepackage{epstopdf}
%\usepackage[colorlinks=true]{hyperref}
\usepackage[affil-it]{authblk}
\usepackage{float}
\usepackage{color}
\usepackage{amsfonts}
\usepackage[margin=1in]{geometry}%
\setcounter{MaxMatrixCols}{30}
\usepackage[title]{appendix}

\newcommand{\highred}[1]{{\color{red}#1}}
\newcommand{\highblue}[1]{{\color{blue}#1}}
\newcommand{\highgreen}[1]{{\color{green}#1}}

\def\ni{\noindent}
\def\proof{{\ni \bf \underline{Proof:} }}
\def\endproof{\hfill$\blacksquare$\vspace{6pt}}

\newtheorem{theorem}{Theorem}[section]
\newtheorem{lemma}[theorem]{Lemma}
\newtheorem{mainres}[theorem]{Main Result}
\newtheorem{lem}[theorem]{Lemma}
\newtheorem{rem}[theorem]{Remark}
\newtheorem{thm}[theorem]{Theorem}
\newtheorem{conj}[theorem]{Conjecture}
\newtheorem{openq}[theorem]{Open question}
\newtheorem{prop}[theorem]{Proposition}
\newtheorem{proposition}[theorem]{Proposition}
\newtheorem{cor}[theorem]{Corollary}
\newtheorem{fact}[theorem]{Fact}
\newtheorem{resu}{Principal Result}[section]

\renewcommand\thesection{\arabic{section}}
\renewcommand\thesubsection{\thesection.\arabic{subsection}}
\renewcommand\thesubsubsection{\thesubsection.\arabic{subsubsection}}
\renewcommand\theequation{\thesection.\arabic{equation}}
\renewcommand{\eqref}[1]{(\ref{#1})}
\usepackage[numbers,sort&compress]{natbib}


\textwidth=7.4in
\hoffset=-0.4in
\textheight=9.0in

\newenvironment{lyxlist}[1]
{\begin{list}{}
{\settowidth{\labelwidth}{#1}
 \setlength{\leftmargin}{\labelwidth}
 \addtolength{\leftmargin}{\labelsep}
 \renewcommand{\makelabel}[1]{##1\hfil}}}
{\end{list}}
\newenvironment{lyxcode}
{\par\begin{list}{}{
\setlength{\rightmargin}{\leftmargin}
\setlength{\listparindent}{0pt}% needed for AMS classes
\raggedright
\setlength{\itemsep}{0pt}
\setlength{\parsep}{0pt}
\normalfont\ttfamily}%
 \item[]}
{\end{list}}

\newcommand{\eps}{{\displaystyle \varepsilon}}
\newcommand{\lam}{{\displaystyle \lambda}}

\DeclareMathOperator{\sech}{sech}
\newcommand{\eproof}{\; \Box }
\newcommand{\abuv}[2]{\genfrac{}{}{0pt}{}{#1}{#2}}
\newcommand{\bsub}{\begin{subequations}}
\newcommand{\esub}{\end{subequations}$\!$}
\renewcommand{\theequation}{\arabic{section}.\arabic{equation}}
\renewcommand{\thesection}{\arabic{section}}

\newcommand{\dzjp}{D^{+}_{0,j}}
\newcommand{\dzjm}{D^{-}_{0,j}}
\newcommand{\dustar}{D_{\textrm{up}}^{\star}}
\newcommand{\dlstar}{D_{\textrm{low}}^{\star}}

\begin{document}

\title{The Stability of Hotspots for a Reaction-Diffusion Model of Urban
Crime with Focused Police Patrol}
\date{\today} \author{Wang Hung Tse\thanks{Dept. of Mathematics, UBC,
     Vancouver, Canada} \enspace,
  Michael J. Ward\thanks{Dept. of Mathematics, UBC, Vancouver,
    Canada. (corresponding author {\tt ward@math.ubc.ca})}}
\baselineskip=16pt

\maketitle

\begin{abstract}

\end{abstract}


\setcounter{equation}{0}
\setcounter{section}{0}
\section{Introduction}\label{sec:int}

In this chapter, we consider the simple interaction case for the
police-criminal dynamics ($I(\rho,U)=U$) in the three-component
reaction-diffusion model. The following model is included as a special
case of a general form proposed in \cite{rick}. In particular, the
simple interaction term $-U$ in the $\rho$-equation (criminal density)
represents a criminal removal rate proportional to the number of
police present at the same spatial location. The nonlinear police
movement term corresponds to a choice of the function $v(A)$ not
explicitly studied in \cite{rick}, given by
\[
v(A)=qD\nabla\log A \,.
\]



Our simple police interaction model on the one-dimensional domain 
$0\leq x \leq S$ is formulated as
\bsub\label{eq:crs-main}
\begin{alignat}{1}
A_{t} & =\epsilon^{2}A_{xx}-A+\rho A^{3}+\alpha\,, \label{eq:crs-main-A}\\
\rho_{t} & =D\left(\rho_{x}-2\rho A_{x}/A\right)_{x}-\rho A^3 +\gamma-\alpha-
U\,, \label{eq:crs-main-rho}\\
\tau_{u}U_{t} & =D\left(U_{x}-qUA_{x}/A\right)_{x}\,,  \label{eq:crs-main-U}
\end{alignat}
\esub 
where $A_x=\rho_x=U_x=0$ at $x=0,S$. By integrating
(\ref{eq:crs-main-U}) over the domain, we obtain that the total level
$U_0$ of police deployment is conserved in time, so that
\begin{equation}\label{ss:pol_con}
   U_0 \equiv \int_{0}^{S} U(x,t)\, dx \,.
\end{equation}
In (\ref{eq:crs-main-U}), the parameter $q>0$ measures the degree of
focus in the police patrol toward maxima of the attractiveness field $A$.
We will assume in our analysis below that $q>1$.  The choice $q=2$
models a ``cops-on-the-dots'' strategy (cf.~\cite{jbc}, \cite{rick},
\cite{zipkin}) whereby the police have the same degree of focus as do
the criminals towards maxima of $A$. When $q$ is above or below $2$,
the police force drift in a less or more focused manner,
respectively, as compared to the movement of the criminals. In
(\ref{eq:crs-main-U}), the ``diffusivity'' of the police is
$D_p\equiv {D/\tau_{u}}$, so that when $\tau_{u}<1$, the police are more mobile
than the criminals. Conversely, $\tau_{u}>1$ corresponds to the police
being comparatively more ``sluggish'' in their movements.

For the analysis, it is convenient to introduce the new variables
$V$ (cf.~\cite{kww_crime}) and $u$ by
\begin{equation}\label{ss:new_var}
\rho=VA^{2}\,, \qquad U=uA^{q} \,.
\end{equation}
When $\epsilon\ll 1$ the attractiveness field $A$ is spatially
localized. For the diffusivity $D$, we consider the regime
$D={\mathcal O}(\epsilon^{-2})$ where steady-state hotspot patterns
were found to have a stability threshold for the basic crime model
(cf.~\cite{kww_crime}). For this regime in $D$, we take $V={\mathcal
  O}(\epsilon^{2})$ to obtain a distinguished balance
(cf.~\cite{kww_crime}).  This motivates the rescaling
\begin{equation}\label{ss:new_rescale}
V=\epsilon^{2}v \,,\qquad D= \epsilon^{-2} {\mathcal D}_{0} \,,
\end{equation}
so that in terms of $v$, $u$, and ${\mathcal D}_0={\mathcal O}(1)$,
(\ref{eq:crs-main}) becomes 
\bsub\label{eq:pol-main}
\begin{align}
A_{t} & =\epsilon^{2}A_{xx}-A+\epsilon^{2}vA^{3}+\alpha\,,
 \label{eq:pol-main-A}\\
\epsilon^{2}\left(A^{2}v\right)_{t} & ={\mathcal D}_{0}\left(A^{2}v_{x}\right)_{x}
-\epsilon^{2}vA^{3}+\gamma-\alpha-uA^{q}\,, \label{eq:pol-main-v}\\ 
\tau_{u}\epsilon^{2}\left(A^{q}u\right)_{t}
& ={\mathcal D}_{0}\left(A^{q}u_{x}\right)_{x} \,. \label{eq:pol-main-u}
\end{align}
\esub

One key issue that we will study for (\ref{eq:pol-main}) is to
determine whether there is an optimum degree of focus in the
patrolling so as to reduce the number of possible crime hotspots in a
given 1-D spatial region. More specifically, we will investigate
whether there are optimal values of $q$ and $\tau_{u}$ so as to obtain
the fewest number of stable steady-state hotspots on a given domain
length. In mathematical terms, we will determine how the stability
threshold in the diffusivity $D$, with $D=\epsilon^{-2}{\mathcal
  D}_0$, for steady-state hotspot patterns depends on $q$, the police
diffusivity ${D/\tau_u}$, and the total level $U_0$ of police
deployment.

To analyze the linear stability of a $K$-hotspot steady-state
solution, in \S \ref{sec:nlep_deriv} we use asymptotic analysis to
derive the nonlocal eigenvalue problem (NLEP), characterizing
${\mathcal O}(1)$ time-scale instabilities of the pattern. The
methodology to derive this NLEP involves using a reference domain
$|x|\leq \ell$ containing a single hotspot centered at $x=0$, and
then imposing Floquet-type boundary conditions at $x=\pm \ell$.  In
terms of this reference problem, the NLEP for the finite-domain
problem $0<x<S$ with Neumann conditions at $x=0,S$ can then readily be
extracted, as similar to that done in \cite{kww_crime} for the basic
two-component crime model.  Such a Floquet-based approach to study the
linear stability of multi-spike steady-states was first introduced in
\cite{floq-ref} in the context of 1-D spatially-periodic spike patterns
for a class of two-component reaction-diffusion systems. It has
subsequently been extended to study the linear stability of 1-D mesa
patterns \cite{mk_mesa}, of 1-D spikes for a competition model with
cross-diffusion effects \cite{kw_xdiff}, and 1-D hotspot patterns for
the basic crime model \cite{kww_crime}. There are two novel features
in the derivation of the NLEP for our three-component RD system with
police. Firstly, in our asymptotic analysis, we must carefully derive
rather intricate jump conditions across the hotspot region. Secondly,
the resulting NLEP that we obtain has two nonlocal terms, instead of
one. As a result, its analysis is seemingly beyond the general NLEP
stability theory with a single nonlocal term, as surveyed in
\cite{wei_rev}.  However, by using some key identities that are
particular to our three-component RD crime model, we show how to
reformulate the NLEP more conveniently in terms of a single nonlocal
term, which can then be more readily analyzed.

\begin{figure}[htbp]
\centering
\includegraphics[width=9cm,height=5.0cm]{figs/ss_2_pol_u02.eps}
\includegraphics[width=9cm,height=5.0cm]{figs/hopf_2_pol_u02.eps}
\caption{\label{fig:hopf_pol_intro} Left panel: The steady-state
  two-hotspot solution for $S=6$, $\gamma=2$, $\alpha=1$, $U_0=2$,
  ${\mathcal D}_0=0.5$, $\epsilon=0.075$, and $q=3$. Right panel: the
  shaded region of linear stability in the (scaled) police diffusivity
  $\epsilon^{2}D_p\equiv {{\mathcal D}_0/\tau_u}$ versus ${\mathcal
    D}_0$ parameter plane. The thin vertical line is the competition
  stability threshold ${\mathcal D}_{0,c}$ given in Proposition
  \ref{q3:main_twospots}, while the leftmost edge of the instability
  region (at $D_p=0$) is ${{\mathcal D}_{0,c}/(1+{3U_0/\omega})}$
  where $\omega=S(\gamma-\alpha)-U_0$.  For ${\mathcal D}_0>{\mathcal
    D}_{0c}$ the hotspot solution is unstable due to a competition
  instability, while in the unshaded region for ${\mathcal
    D}_0<{\mathcal D}_{0c}$, the hotspot steady-state is unstable to
  an asynchronous oscillatory instability of the hotspot amplitudes. The full
  PDE simulations in Fig.~\ref{fig:valid_2spot_q3} are done at the marked 
  points.}
\end{figure}

% left figure: eps^2 D_p=0.20, D_0=0.5
% middle figure: eps^2 D_p=0.32, D_0=0.8
% right figure: eps^2 D_p=0.4, D_0=0.5
%
\begin{figure}[htbp]
\centering
\includegraphics[width=0.32\textwidth,height=5.0cm]{figs/amp_2_run1.eps}
\includegraphics[width=0.32\textwidth,height=5.0cm]{figs/amp_2_run2.eps}
\includegraphics[width=0.32\textwidth,height=5.0cm]{figs/amp_2_run3.eps}
\caption{\label{fig:valid_2spot_q3} Plot of the spot amplitudes
  computed numerically from the full PDE system (\ref{eq:pol-main})
  for a two-spot pattern with $S=6$, $\gamma=2$, $\alpha=1$, $U_0=2$,
  $\epsilon=0.075$, and $q=3$, at the marked points in the right panel of
  Fig.~\ref{fig:hopf_pol_intro}.  Left panel: $\hat{\tau}_u=2.5$ and
  ${\mathcal D}_0=0.5$, so that $\epsilon^2 D_p=0.2$. Spot amplitudes
  are unstable to asynchronous oscillations, which leads to the
  collapse of one hotspot.  Middle panel: $\hat{\tau}_u=2.5$ and
  ${\mathcal D}_0=0.8$, so that $\epsilon^2 D_p=0.32$. Spot amplitudes
  are unstable to a competition instability. Right panel:
  $\hat{\tau}_u=1.25$ and ${\mathcal D}_0=0.5$, so that
  $\epsilon^2D_p=0.4$. Spot amplitudes are stable to asynchronous
  oscillations and there is no competition instability.  These results
  are consistent with the linear stability predictions in the right
  panel of Fig.~\ref{fig:hopf_pol_intro} (see also the left panel of
  Fig.~\ref{fig:hopf_tau_2} below).}
\end{figure}

In \S \ref{sec:stab_q3} we show for the special case where $q=3$ that
the spectrum of the NLEP can be reduced to the study of a simple
algebraic equation for the eigenvalue parameter. More specifically, we
show that the problem of determining unstable eigenvalues of the NLEP
reduces to determining roots $\lambda$ in $\mbox{Re}(\lambda)>0$ to a
quadratic equation. By analyzing this simple spectral problem for
$q=3$, we show explicitly that, when $D$ is below a certain
competition instability threshold, a spatial pattern of two hotspots
can be destabilized by an asynchronous, or anti-phase, temporal
oscillation in the hotspot amplitudes when the police diffusivity
${D/\tau_u}$ falls below a Hopf bifurcation value. This Hopf
bifurcation threshold can be determined analytically. The existence of
such robust asynchronous temporal oscillations in the hotspot
amplitudes is a qualitatively new phenomena, which does not occur in
the study of spike stability for other RD systems such as the
Gierer-Meinhardt and Gray-Scott models. For these two-component
models, previous NLEP stability analyzes have shown that the dominant
oscillatory instability of the spike amplitudes is always a
synchronous instability.


\setcounter{equation}{0}
\setcounter{section}{1}
\section{\label{sec:pol-construction}Asymptotic Construction of a Multiple
Hotspot Steady-State}

In this section we construct a steady-state solution to
(\ref{eq:pol-main}) on $0\leq x\leq S$ with $K\geq 1$ interior
hotspots. 

To construct a steady-state with $K$ interior hotspots to
(\ref{eq:pol-main}) on $0<x<S$, where the hotspots have a common
amplitude, we will first construct a one-hotspot solution to
(\ref{eq:pol-main}) on $|x|\leq l$ centered at $x=0$. Then, by using
the translation-invariance property of (\ref{eq:pol-main}), we obtain
a $K$ interior hotspot steady-state solution on the original domain of
length $S=(2\ell)K$. In terms of this reference domain $|x|\leq \ell$,
(\ref{ss:pol_con}) yields that
\begin{equation}
U_{0}=K\int_{-\ell}^{\ell}U\, dx \,. \label{eq:U0-with-K}
\end{equation}
In this way, we need only construct a one-hotspot steady-state
solution to (\ref{eq:pol-main}) centered at $x=0$ and impose
$A_x=v_x=u_x=0$ at $x=\pm \ell$. We refer to this as the {\em
  canonical} hotspot problem.

From the steady-state of (\ref{eq:pol-main-u}), together with
$U=uA^{q}$ and (\ref{eq:U0-with-K}), it follows that $u$ is spatially 
constant and given by
\begin{equation}
u=\frac{U_{0}}{K\int_{-\ell}^{\ell}A^{q}\, dx} \,. \label{eq:pol-u-uniform}
\end{equation}
By using (\ref{eq:pol-u-uniform}) in (\ref{eq:pol-main}), the
steady-state problem for the three-component system reduces to the
following two-component problem with a nonlocal term:
\bsub\label{eq:pol-stdy}
\begin{gather}
\epsilon^{2}A_{xx}-A+\epsilon^{2}vA^{3}+\alpha=0\,,\qquad |x|\leq \ell\,;
 \qquad A_x=0 \quad x=\pm \ell \,, \label{eq:pol-stdy-A}\\
{\mathcal D}_{0}\left(A^{2}v_{x}\right)_{x}-\epsilon^{2}vA^{3}+\gamma-\alpha-
\frac{U_{0}}{K}\frac{A^{q}}{\int_{-\ell}^{\ell}A^{q}\, dx}=0\,, \qquad
 |x|\leq \ell\,; \qquad v_x=0 \quad x=\pm \ell \,. \label{eq:pol-stdy-v}
\end{gather}
\esub

We now construct the solution to (\ref{eq:pol-stdy}) with a single
hotspot centered at $x=0$. In the outer region we have
$A\sim\alpha+{\mathcal O}(\epsilon^{2})$, while in the inner region we
put $y=\epsilon^{-1}x$ and $A\sim A_{0}/\epsilon$ to obtain on
$-\infty<y<\infty$ that
\begin{equation*}
A_{0yy}-A_{0}+vA_{0}^{3}+\epsilon\alpha = 0 \,, \qquad
{\mathcal D}_{0}\epsilon^{-4}\left(A_{0}^{2}v_{y}\right)_{y}+{\mathcal
  O}(\epsilon^{-1}) = 0\,.
\end{equation*}
Therefore, to leading order it follows that $v\sim v_{0}$ is a constant,
and that
\begin{equation}
A_{0} \sim \frac{w(y)}{\sqrt{v_{0}}} \,, \label{eq:pol-A-leading}
\end{equation}
where $w(y)=\sqrt{2}\sech y$ is the homoclinic solution of 
\begin{equation}\label{wground}
 w^{\prime\prime}-w+w^{3}=0\,,\qquad-\infty<y<\infty\,; \qquad
w(0)>0\,,\quad w^{\prime}(0)=0\,,\quad w\to0 \quad \mbox{ as }\,\,\, 
y\to\pm\infty\,.
\end{equation}
The integrals of $w(y)$ that are needed below are
\begin{equation}\label{ss:int} 
 \int_{-\infty}^{\infty}w \, dy=
\int_{-\infty}^{\infty}w^{3} \, dy=\sqrt{2}\pi\,, \quad
\int_{-\infty}^{\infty}w^{2}\,dy=4\,,\quad\int_{-\infty}^{\infty}w^{4}\, dy=
  \frac{16}{3}\,, 
\quad \frac{\int_{-\infty}^{\infty}w^{5}\, dy}{\int_{-\infty}^{\infty}w^{3}\,dy}
 =\frac{3}{2}\,.
\end{equation}
More generally, we can readily calculate in terms of the usual Gamma
function $\Gamma(z)$ that
\begin{equation}
I_{q}\equiv \int_{-\infty}^{\infty}w^{q}\, dy=
2^{3q/2-1}\frac{\left[\Gamma(q/2)\right]^2}{\Gamma(q)}\,. 
\label{eq:I_q-general-formula}
\end{equation}

We return to (\ref{eq:pol-u-uniform}), and for $q>1$ we estimate the key 
integral
\[
\int_{-\ell}^{\ell}A^{q}\, dx\sim 2\ell\alpha+\epsilon^{1-q}v_{0}^{-q/2}
\int_{-\infty}^{\infty}w^{q} \, dy= {\mathcal O}(\epsilon^{1-q}) \gg 1 \,.
\]
From (\ref{eq:pol-u-uniform}), it follows that $u={\mathcal
  O}(\epsilon^{q-1})\ll 1$ since $q>1$.  With our assumption 
$q>1$, the integral $\int_{-\ell}^{\ell}A^{q}\, dx$, and thus $u$, depend
to leading-order only on the inner region contribution from
$A^{q}$. For $q>1$, we obtain to leading-order 
from (\ref{eq:pol-u-uniform}) that 
\begin{equation}
  u\sim\epsilon^{q-1}\tilde{u}_{e}\,, \qquad 
\mbox{where} \qquad \tilde{u}_{e} \equiv
\frac{U_{0} v_{0}^{q/2}}{K I_{q}}\,.\label{eq:pol-u-leading}
\end{equation}

Next, we determine $v_{0}$ by integrating (\ref{eq:pol-stdy-v})
on $-\ell<x<\ell$ and then imposing $v_{x}(\pm\ell)=0$. This yields that
\[
\epsilon^{2}\int_{-\ell}^{\ell}vA^{3}\,dx=
2\ell\left(\gamma-\alpha\right)-U_{0}/K\,.
\]
Therefore, since $A\sim\alpha={\mathcal O}(1)$ in the outer region,
while $A={\mathcal O}(\epsilon^{-1})$ in the inner region, it follows
that, when $q>1$, the dominant contribution to the integral arises from
the inner region where $v\sim v_0$. In this way, we estimate
\begin{equation}
\frac{\int_{-\infty}^{\infty}w^{3}\, dy}{\sqrt{v_0}}  =  
2\ell\left(\gamma-\alpha\right)-
U_{0}/K \,. \label{eq:pol-v0-relation0-with-pdepar} 
\end{equation}
From (\ref{eq:pol-v0-relation0-with-pdepar}), a
steady-state hotspot solution exists only when the total level $U_0$ of
police deployment is below a threshold given by
\begin{equation}
U_{0}<U_{0,{\textrm max}}\equiv2\ell K\left(\gamma-\alpha\right) = 
S  (\gamma-\alpha) \,.
\label{eq:U0max}
\end{equation}
Here $S=2\ell K$ is the original domain length. We will
assume that (\ref{eq:U0max}) holds, so that a $K$-hotspot steady-state
exists. 

Upon solving (\ref{eq:pol-v0-relation0-with-pdepar}) for $v_0$, and
using (\ref{ss:int}) for $\int_{-\infty}^{\infty} w^3\, dy$, we get
\begin{equation}
 v_{0}  =  2\pi^{2}\left[2\ell\left(\gamma-\alpha\right)-U_{0}/K\right]^{-2}
  = 2\pi^2 K^2 \left[S(\gamma-\alpha)-U_{0}\right]^{-2}\,.
\label{eq:pol-v0-formula}
\end{equation}
Since $v_0$ increases when either $K$ increases or the total level
$U_0$ of police increases, it follows from (\ref{eq:pol-A-leading})
that the maximum $A_{\max}\equiv A(0)\gg 1$ of the attractiveness field, given by
\begin{equation}\label{eq:amax}
    A_{\max}\equiv A(0) \sim \epsilon^{-1} A_{0}(0) = \frac{\epsilon^{-1}}{\pi K} 
   \left[S(\gamma-\alpha)-U_0\right] \,,
\end{equation}
decreases with increasing $K$ or increasing police deployment
$U_0$. However, this maximum value of $A$ is independent of the police
patrol focus parameter $q$.

To complete the asymptotic construction of the hotspot, we must
determine $v$. In the outer region, we expand $v\sim v_{e}(x)+\dots$
and recall that $A\sim\alpha+{\mathcal O}(\epsilon^{2})$ so that
${A^q/\int_{-\ell}^{\ell} A^q \, dx}={\mathcal O}(\eps^{q-1})\ll 1$
since $q>1$. In this way, from (\ref{eq:pol-stdy-v}), we obtain to
leading order that $v_{e}(x)$ satisfies
\begin{equation}
 {\mathcal D}_{0}v_{exx}=-\frac{\left(\gamma-\alpha\right)}{\alpha^{2}} 
  \,, \quad -\ell <x<\ell \,; \qquad
v_{e}(0)=v_{0}\,,\; \qquad v_{ex}(\pm \ell)=0 \,. \label{eq:pol-v-outer-ode}
\end{equation}
The solution to (\ref{eq:pol-v-outer-ode}) is
\begin{equation}
v_{e}(x)=\frac{\zeta}{2}\left[\left(\ell-|x|\right)^{2}-\ell^{2}\right]+v_{0}
\,, \qquad 0<|x|\leq\ell \,; \qquad \zeta \equiv -
\frac{(\gamma-\alpha)}{{\mathcal D}_0 \alpha^2} \,, \label{eq:pol-v-outer-sol}
\end{equation}
where $v_{0}$ is given in (\ref{eq:pol-v0-formula}). This expression is
a uniformly valid leading order solution for $v$ on $|x|\leq \ell$.

We summarize the results for our leading-order construction of a
steady-state $K$-hotspot pattern as follows:

\begin{prop}
\label{thm: pol-symm-K-hotspots-for-main} Let $\epsilon\to 0$, $q>1$,
and $0<U_0<U_{0,\textrm{max}}$, as in (\ref{eq:U0max}). Then,
(\ref{eq:pol-main}) admits a steady-state solution on $(0,S)$ with $K$
interior hotspots of a common amplitude. On each sub-domain of length
$2\ell=S/K$, and translated to $(-\ell,\ell)$ to contain exactly one
hotspot at $x=0$, the steady-state solution, to leading order, is
given by
\bsub\label{thm:main_eq}
\begin{gather}
\begin{gathered}A\sim\frac{w(x/\epsilon)}{\epsilon \sqrt{v_{0}}}\,, \quad
\mbox{ if }\quad x={\mathcal O}(\epsilon)\,;\end{gathered}
\qquad A\sim\alpha\,, \quad \mbox{ if } \quad x={\mathcal O}(1)\,, \\
v\sim v_{e}=\frac{\zeta}{2}\left[\left(\ell-|x|\right)^{2}-\ell^{2}\right]+v_{0}
  \,, \qquad \mbox{where} \quad 
  v_{0}=2\pi^{2}K^2 \left[S(\gamma-\alpha)-U_{0}\right]^{-2}\,,\\
u\sim\epsilon^{q-1}\tilde{u}_{e}\,, \qquad \mbox{where} \quad
  \tilde{u}_{e}\equiv\frac{U_{0} v_{0}^{q/2}}{K I_{q}} \,, \qquad 
 I_{q}\equiv \int_{-\infty}^{\infty} w^q \, dy =
  2^{3q/2-1}\frac{\left[\Gamma(q/2)\right]^2}{\Gamma(q)} \,.
\end{gather}
\esub Here $w(y)=\sqrt{2}\sech y$ is the homoclinic of (\ref{wground}).
\end{prop}

In terms of the criminal and police densities, given respectively by
$\rho=\epsilon^2 v A^{2}$ and $U=uA^{q}$ (see (\ref{ss:new_var}) and
(\ref{ss:new_rescale})), we can write (\ref{thm:main_eq}) as follows:

\begin{cor}\label{cor:pol-K-hotspots-in-A-rho-U} Under the same conditions
as in Proposition \ref{thm: pol-symm-K-hotspots-for-main}, 
(\ref{thm:main_eq}) yields to leading-order that
\bsub \label{thm:main_eq_corr}
\begin{gather}
\begin{gathered} A\sim\frac{w(x/\epsilon)}{\epsilon \sqrt{v_{0}}}\,,
 \quad \mbox{ if } \quad x={\mathcal O}(\epsilon)\,;\end{gathered}
  \qquad A\sim\alpha\,, \quad \mbox{ if } \quad {\mathcal
    O}(\epsilon)\ll |x|<\ell\,, \\ 
\rho\sim \left[w(x/\epsilon)\right]^2\,, 
 \quad \mbox{ if } \quad x={\mathcal O}(\epsilon)
 \,;
  \qquad\rho\sim\epsilon^{2}v_{e}\alpha^{2}\,, \quad \mbox{ if } \quad
  {\mathcal O}(\epsilon)\ll|x|<\ell\,,
  \\ U\sim \frac{U_{0}}{ \epsilon K I_{q}}\left[w(x/\epsilon)\right]^q
  \,, \quad \mbox{ if }
  \quad x={\mathcal O}(\epsilon)\,;\qquad
  U\sim\epsilon^{q-1}\alpha^q \frac{U_{0} v_{0}^{q/2}}{K I_{q}} \,,
  \quad \mbox{ if } \quad {\mathcal O}(\epsilon)\ll|x|<\ell \,,
\end{gather}
where $v_e$ and $v_0$ are given in (\ref{thm:main_eq}) and
$w(y)=\sqrt{2}\sech y$. 
\esub
\end{cor}

From (\ref{thm:main_eq_corr}), we observe that the criminal density
near a hotspot is independent of the police deployment $U_0$ and
patrol focus $q$. However, the maximum of the attractiveness field is
decreased by increasing $U_0$.

\setcounter{equation}{0}
\setcounter{section}{2}
\section{Derivation of the NLEP for a $K$-Hotspot Steady-State Pattern}\label{sec:nlep_deriv}

To analyze the linear stability of a $K$-hotspot steady-state
solution, we must use asymptotic analysis to derive the corresponding
nonlocal eigenvalue problem (NLEP). To do so, we first follow the
methodology in \cite{kww_crime} by deriving the NLEP for a one-hotspot
solution on the reference domain $|x|\leq \ell$, with Floquet-type
boundary conditions imposed at $x=\pm \ell$. In terms of this
reference problem, the NLEP for the finite-domain problem $0<x<S$ with
Neumann conditions at $x=0,S$ is then readily recovered, as similar to
that done in \cite{kww_crime} for the basic crime model.

\subsection{Linearization with Floquet Boundary Conditions}

To study the linear stability of a $K$-hotspot steady-state we
introduce the perturbation 
\begin{equation}
A=A_{e}+e^{\lambda t}\phi \,,\qquad v=v_{e}+e^{\lambda t}\epsilon\psi\,,
\qquad u=u_{e}+e^{\lambda t}\epsilon^{q}\eta\,, \label{eq:pol-perturb}
\end{equation}
where $(A_{e},v_{e},u_{e})$ is the steady-state with a single hotspot
centered at the origin in $|x|\leq \ell$.  The orders of the
perturbations (${\mathcal O}(1)$, ${\mathcal O}(\epsilon$) and
${\mathcal O}(\epsilon^{q})$ for the $A$, $v$ and $u$ components,
respectively) are chosen so that $\phi$, $\psi$, and $\eta$ are all
${\mathcal O}(1)$ in the inner region.  Upon substituting
(\ref{eq:pol-perturb}) into (\ref{eq:pol-main}) and linearizing, we
obtain that 
\bsub\label{eq:pol-linearized}
\begin{gather}
\epsilon^{2}\phi_{xx}-\phi+3\epsilon^{2}v_{e}A_{e}^{2}\phi+
\epsilon^{3}A_{e}^{3}\psi  =\lambda\phi\,, \label{eq:pol-linearized-1}\\ 
{\mathcal D}_{0}\left(2A_{e}v_{ex}\phi+\epsilon A_{e}^{2}\psi_{x}\right)_{x} 
-3\epsilon^{2}A_{e}^{2}v_{e}\phi-\epsilon^{3}A_{e}^{3}\psi 
-qu_{e}A_{e}^{q-1}\phi-\epsilon^{q}\eta A_{e}^{q} 
=\lambda\epsilon^{2}\left(2A_{e}v_{e}\phi+\epsilon
A_{e}^{2}\psi\right)\,, \label{eq:pol-linearized-2}\\ 
{\mathcal D}_{0}\left(qA_{e}^{q-1}\phi u_{ex}+\epsilon^{q}A_{e}^{q}\eta_{x}
\right)_{x} 
=\epsilon^{2}\tau_{u}\lambda\left(qA_{e}^{q-1}u_{e}\phi+\epsilon^{q}
A_{e}^{q}\eta\right)\,.
\label{eq:pol-linearized-3}
\end{gather}
\esub For $K\geq 2$, we will impose for the long-range components
$\psi$ and $\eta$ in (\ref{eq:pol-linearized-2}) and
(\ref{eq:pol-linearized-3}) the following Floquet-type boundary
conditions at $x=\pm\ell$:
\begin{equation}
\left(\begin{array}{c}
\eta(\ell)\\
\psi(\ell)
\end{array}\right)=z\left(\begin{array}{c}
\eta(-\ell)\\
\psi(-\ell)
\end{array}\right)\,, \qquad\left(\begin{array}{c}
\eta(\ell)\\
\psi(\ell)
\end{array}\right)=z\left(\begin{array}{c}
\eta_{x}(-\ell)\\
\psi_{x}(-\ell)
\end{array}\right)\,,\label{eq:pol-floquet-bc}
\end{equation}
where $z$ is a complex-valued parameter.   For the $K=1$ case,
considered separately in \S \ref{stab:onespot} below, we need only
impose Neumann conditions at $x=\pm l$ for the perturbations. Here we
treat the $K\geq 2$ case.

For $K\geq 2$, the NLEP associated with a $K$-hotspot pattern on
$[-l,(2K-1)l]$ with {\em periodic boundary conditions}, on a domain of
length $2Kl$, is obtained by setting $z^K=1$, which yields
\begin{equation}
    z_j = e^{2\pi i j/K} \,, \qquad j=0,\ldots,K-1 \,. \label{3:zj}
\end{equation}
By using these values of $z_j$ in (\ref{eq:pol-floquet-bc}) we 
obtain the spectral problem for the linear stability of a $K$-hotspot
solution on a domain of length $2Kl$ subject to periodic boundary
conditions. The next step is then to relate the spectra of the
periodic problem to the Neumann problem in such a way that the Neumann
problem is still posed on a domain of length $S$ (cf.~\cite{kww_crime}). As
discussed in \S 3 of \cite{kww_crime} for $K\geq 2$, this involves
simply replacing $2K$ with $K$ in (\ref{3:zj}).  As such, our Floquet
parameter in (\ref{eq:pol-floquet-bc}) for a hotspot steady-state on a
domain of length $S=2lK$, having $K\geq 2$ interior hotspots and
Neumann boundary conditions at $x=0$ and $x=S$ is $z=e^{\pi i
  j/K}$. With this value of $z$, we calculate the identity
\begin{equation}\label{3:ziden}
    \frac{(z-1)^2}{2 z} = \mbox{Re}(z)-1 = 
\cos\left(\frac{\pi j}{K}\right) - 1 \,, \quad j=0,\ldots,K-1 \,,
\end{equation}
which is needed in our analysis below.

We now begin our derivation of the NLEP. For
(\ref{eq:pol-linearized-1}), in the inner region where
$A_{e}\sim \epsilon^{-1} {w/\sqrt{v_0}}$, $v_e\sim v_0$, and 
$\psi\sim \psi(0)\equiv \psi_{0}$, we obtain to leading-order that
\begin{equation}
\Phi^{\prime\prime}-\Phi+3w^{2}\Phi+\frac{\psi(0)}{v_{0}^{3/2}}w^{3}=\lambda\Phi\,.
\label{eq:pol-nlep-generic}
\end{equation}
Here $\Phi(y)=\phi(\epsilon y)$ is the leading order term for the
inner expansion of $\phi$. In contrast, in the outer region, we obtain
to leading order from (\ref{eq:pol-linearized}) that 
\begin{equation}
\phi\sim\epsilon^{3}\alpha^{3}\psi/[\lambda+1-3\epsilon^{2}\alpha^{2}v_{e}]=
{\mathcal O}(\epsilon^{3}),\qquad\psi_{xx}\approx 0\,,\qquad\eta_{xx}\approx 0\,.
\label{stab:phi_out}
\end{equation}
To determine $\psi(0)$, which from (\ref{eq:pol-nlep-generic}) will yield 
the NLEP, we must first carefully derive appropriate jump conditions for
$\psi_x$ and $\eta_x$ across the hotspot region centered at $x=0$.
This is done in the next sub-section.

\subsection{Jump Conditions and the Derivation of the NLEP for $K\geq 2$}

To derive the appropriate jump condition for $\psi_x$ across the
hotspot region, we integrate (\ref{eq:pol-linearized-2}) over an
intermediate domain $-\delta<x<\delta$ with $\epsilon\ll \delta\ll
1$. We use the facts that $A_{e}\sim \epsilon^{-1} {w/\sqrt{v_{0}}}$,
$\phi\sim\Phi(y)$, $A_{e}(\pm\delta)\sim\alpha$, and
$u_{e}=\epsilon^{q-1}\tilde{u}_e$ as given in (\ref{thm:main_eq}), to
obtain, upon letting ${\delta/\epsilon}\to +\infty$, that
\begin{eqnarray*}
\epsilon {\mathcal D}_{0}\alpha^{2}\left[\psi_{x}\right]_{0}+
2{\mathcal D}_{0}\alpha\left[v_{ex}\phi\right]_{0} & = & 
3\epsilon\int_{-\infty}^{\infty}w^{2}\Phi \, dy+
\frac{\epsilon\psi(0)}{v_{0}^{3/2}}\int_{-\infty}^{\infty}w^{3} \, dy\\
 &  & \qquad +\frac{\epsilon q\tilde{u}_{e}}{v_{0}^{(q-1)/2}}
\int_{-\infty}^{\infty}w^{q-1}\Phi \, dy+\frac{\epsilon\eta(0)}{v_{0}^{q/2}}
\int_{-\infty}^{\infty}w^{q} \, dy+{\mathcal O}(\epsilon^{2}\lambda)\,,
\end{eqnarray*}
where we have introduced the notation $[a]_{0}\equiv
a(0^{+})-a(0^{-})$ to indicate that the evaluation is to be done with
the outer solution. In addition, below we will use the convenient
shorthand notation that
$\int\left(\dots\right)\equiv\int_{-\infty}^{\infty}\left(\dots\right)\,
dy$.  Since $\phi={\mathcal O}(\epsilon^{3})$ in the outer region from
(\ref{stab:phi_out}), we can neglect the second term on the left-hand
side of the expression above.  For eigenvalues for which $\lambda\ll
{\mathcal O}(\epsilon^{-1})$, we obtain that
\begin{equation}
{\mathcal D}_{0}\alpha^{2}[\psi_{x}]_{0}=3\int
w^{2}\Phi+\frac{\psi(0)}{v_{0}^{3/2}}\int
w^{3}+\frac{q\tilde{u}_{e}}{v_{0}^{(q-1)/2}}\int
w^{q-1}\Phi+\frac{\eta(0)}{v_{0}^{q/2}}\int w^{q} \,. \label{eq:pol-psi-1}
\end{equation}

Now from (\ref{eq:pol-linearized-2}), we use $\phi={\mathcal
  O}(\epsilon^{3})$ in the outer region, together the fact
$\epsilon^{q}\eta A_{e}^{q}\ll {\mathcal O}(\epsilon)$ since $q>1$. In
this way, from (\ref{eq:pol-linearized-2}) and (\ref{eq:pol-psi-1}),
we obtain the following leading-order BVP problem for $\psi$ with
a jump condition for $\psi_x$ across $x=0$:
\bsub \label{stab:psi_prob}
\begin{equation}
\psi_{xx}  =  0\,, \quad |x|\leq\ell\,; \qquad
e_{0}\left[\psi_{x}\right]_{0}  =  e_{1}\psi(0)+e_{2}\eta(0)+e_{3}\,, \quad
\psi(\ell)  =  z\psi(-\ell)\,,\quad\psi_{x}(\ell)=z\psi_{x}(-\ell)\,,
\label{eq:pol-psi-ode}
\end{equation}
where we have defined $e_j$, for $j=0,\ldots,3$, by
\begin{equation}
e_{0}\equiv {\mathcal D}_{0}\alpha^{2}\,, \qquad e_{1}\equiv \frac{1}{v_{0}^{3/2}}
\int w^{3}\,, \qquad e_{2}\equiv \frac{1}{v_{0}^{q/2}}\int w^{q} \,, \qquad
e_{3}\equiv 3\int w^{2}\Phi+\frac{q\tilde{u}_{e}}{v_{0}^{(q-1)/2}}\int 
w^{q-1}\Phi\,. \label{eq:pol-psi-e0123}
\end{equation}
\esub

This BVP is defined in terms of $\eta(0)$, which itself must be
calculated from a separate BVP. To formulate this additional BVP, we
integrate (\ref{eq:pol-linearized-3}) over $-\delta<x<\delta$, with
$\epsilon\ll \delta\ll 1$, and let ${\delta/\epsilon}\to\infty$ to obtain
\begin{equation}\label{stab:eta_jump}
{\mathcal D}_{0}\epsilon^{q}\alpha^{q}\left[\eta_{x}\right]_{0}+
 {\mathcal D}_{0}q\alpha^{q-1}{\mathcal O}(\epsilon^{q+2})
=\epsilon^{3}\tau_{u}\lambda\left[\frac{ q \tilde{u}_\epsilon}
  {v_{0}^{(q-1)/2}}\int w^{q-1}\Phi  + \frac{\eta(0)}{v_{0}^{q/2}}
 \int w^{q}\right]\,.
\end{equation}
To achieve a distinguished balance in (\ref{stab:eta_jump}), we
introduce $\hat{\tau}_u$ defined by 
\begin{equation}
\hat{\tau}_{u} \equiv\epsilon^{3-q}\tau_{u}\,. \label{eq:pol-tau-hat}
\end{equation}
With this scaling, the police diffusivity 
$D_{p}\equiv \epsilon^{-2}{{\mathcal D}_0/\tau_u}$, is
simply
\begin{equation}\label{stab:police_diff}
   D_p \equiv \epsilon^{1-q} {{\mathcal D}_{0}/\hat{\tau}_u} \,.
\end{equation}
In this way, (\ref{stab:eta_jump}) yields the following jump condition
for the outer solution:
\begin{equation}
{\mathcal D}_{0}\alpha^{q}\left[\eta_{x}\right]_{0}=
\hat{\tau}_{u}\lambda\left[\frac{q\tilde{u}_{e}}{v_{0}^{(q-1)/2}}\int
  w^{q-1}\Phi+\frac{\eta(0)}{v_{0}^{q/2}}\int w^{q}\right]\,. 
\label{eq:pol-eta-1}
\end{equation}

Now in the outer region we obtain from (\ref{eq:pol-linearized-3})
that
\begin{equation}\label{stab:out_eta}
{\mathcal D}_{0}\epsilon^{q}\alpha^{q}\eta_{xx}+{\mathcal
  O}(\epsilon^{q+2})=\epsilon^{2}\tau_{u}\lambda\left[{\mathcal
    O}(\epsilon^{q+2})+{\mathcal O}(\epsilon^{q})\right]\,.
\end{equation}
We will consider the range of $\tau_u$, and consequently $\hat{\tau}_u$,
where
\begin{equation}
\tau_{u}\ll {\mathcal O}(\epsilon^{-2})\,, \qquad \mbox{for which} \qquad
\hat{\tau}_{u} \ll {\mathcal O}(\epsilon^{1-q}) \,.
\label{eq:pol-tau_u-assumption}
\end{equation}
We will assume in our theory below that $\hat{\tau}_u={\mathcal O}(1)$,
so that (\ref{eq:pol-tau_u-assumption}) holds for all $q>1$.

For this range, (\ref{stab:out_eta}) reduces to $\eta_{xx}\approx 0$
to leading order. In this way, we obtain using (\ref{eq:pol-eta-1}),
the following BVP for the leading-order outer solution for $\eta$
with a jump condition for $\eta_x$ across $x=0$:
\bsub \label{stab:eta_prob}
\begin{equation}
\eta_{xx}  =  0\,, \quad|x|\leq\ell\,; \qquad
d_{0}\left[\eta_{x}\right]_{0}  =  d_{1}\eta(0)+d_{2}\,, 
\quad \eta(\ell)  =  z\eta(-\ell) \,, \quad
 \eta_{x}(\ell)=z\eta_{x}(-\ell)\,. \label{eq:pol-eta-ode}
\end{equation}
Here the constants $d_0$, $d_1$, and $d_2$, are defined by
\begin{equation}
d_{0}\equiv {\mathcal D}_{0}\alpha^{q},\, \qquad 
d_{1}\equiv \frac{\hat{\tau}_{u}\lambda}{v_{0}^{q/2}}\int w^{q},\, \qquad
d_{2}\equiv \frac{\hat{\tau}_{u}\lambda q\tilde{u}_{e}}{v_{0}^{(q-1)/2}}\int 
w^{q-1}\Phi \,. \label{eq:pol-eta-d012}
\end{equation}
\esub
To solve the BVPs (\ref{stab:eta_prob}) and (\ref{stab:psi_prob}), and in
this way determine $\psi(0)$ and $\eta(0)$, we need to establish a simple lemma.

\begin{lem}\label{lemma:floq_BVP} Consider the  BVP for $y=y(x)$ on 
$-\ell<x<\ell$ given by
\begin{equation}\label{lemma:bvp}
y_{xx} =0\,, \quad -\ell<x<\ell \,; \qquad
f_0 \left[y_{x}\right]_{0}= f_1 y(0)+ f_2 \,; \qquad
y(\ell) = z y(-\ell)\,, \quad y_{x}(\ell)=z y_{x}(-\ell)\,,
\end{equation}
where $f_0$, $f_1$ and $f_2$, are nonzero constants, and let $z$ satisfy
(\ref{3:ziden}). Then, $y(0)$ is given by
\begin{equation}\label{lemma:y0}
  y(0)= f_2 \left[ \frac{f_{0}}{\ell}\frac{\left(z-1\right)^{2}}{2z}-
f_{1} \right]^{-1} = -\frac{f_2}
{f_0 {\left(1-\cos\left({\pi j/K}\right)\right)/\ell}+f_{1}}\,.
\end{equation}
\end{lem}

\begin{proof}
Let $y_{0}=y(0)$. The solution of this BVP is continuous but
not differentiable at $x=0$, and has the form
\[
y(x)=\begin{cases}
y_{0}+A_{+}x & \mbox{ if }0<x<\ell\,, \\
y_{0}+A_{-}x & \mbox{ if }-\ell<x<0\,,
\end{cases}
\]
where $y_0\equiv y(0)$.  Upon imposing the Floquet boundary conditions
we obtain $A_{+}=zA_{-}$ and
$y_{0}+A_{+}\ell=z\left(y_0-A_{-}\ell\right)= zy_{0}-A_{+}\ell$, which
yields that $A_{+}={(z-1)y_{0}/(2\ell)}$. Then, upon imposing the
jump condition across $x=0$ we get
\begin{equation*}
f_{1} y_0 +f_{2}  =  f_{0}\left[y_{x}\right]_{0} = f_{0}
 \left(A_{+}-A_{-}\right) = \frac{f_{0}y_{0}}{2\ell}
 \left(z-1\right)\left(1-\frac{1}{z}\right) \,.
\end{equation*}
Upon solving for $y(0)$, and recalling the identify (\ref{3:ziden}),
we obtain (\ref{lemma:y0}) for $y(0)$.
\end{proof}

Upon using Lemma \ref{lemma:floq_BVP} with $f_0=e_0$, $f_1=e_1$, and
$f_2=e_2\eta(0)+e_3$, we calculate from (\ref{stab:psi_prob}) that 
\begin{equation}
\psi(0)=-\frac{e_{2}\eta(0)+e_{3}}{e_{0}
{\left(1-\cos(\pi j/K)\right)/\ell}+e_{1}}\,. \label{eq:pol-psi_0-1}
\end{equation}
Then, by applying Lemma \ref{lemma:floq_BVP} with
$f_0=d_0$, $f_1=d_1$, and $f_2=d_2$, we calculate from
(\ref{stab:eta_prob}) that 
\begin{equation}
\eta(0)=-\frac{d_{2}}{d_{0}{\left(1-\cos(\pi j/K)\right)/\ell} +
d_{1}}\,. \label{eq:pol-eta_0-1}
\end{equation}
The final step in the derivation of the NLEP is to substitute
(\ref{eq:pol-eta_0-1}) into (\ref{eq:pol-psi_0-1}) and simplify the
resulting expression for $\psi(0)$ so as to express it explicitly in
terms of the original parameters. This will identify the key
coefficient ${\psi(0)/v_{0}^{3/2}}$ in (\ref{eq:pol-nlep-generic}).

We first define $D_{j}$ by
\begin{equation}
D_{j}\equiv\frac{{\mathcal D}_{0}}{\ell}
 \left( 1-\cos\left( \frac{\pi j}{K}\right)\right)\,, \qquad
 j=0,\ldots,K-1 \,, \quad \mbox{where}  \quad l={S/(2K)} \,, \label{eq:pol-D_j}
\end{equation}
so that $D_{j}<D_{j+1}$ for any $j=0,1,2,\ldots,K-2$. Then, by calculating
$e_0$ and $d_0$ from (\ref{eq:pol-psi-e0123}) and
(\ref{eq:pol-eta-d012}), and substituting (\ref{eq:pol-eta_0-1}) into
(\ref{eq:pol-psi_0-1}), we obtain
\begin{equation}
\psi(0)  =  -\frac{1}{D_{j}\alpha^2 +e_{1}}\left[e_{3}-
\frac{e_{2}d_{2}}{D_{j}\alpha^q+d_{1}} \right]\,. \label{eq:pol-psi_0-pre-1}
\end{equation}
Then, by using the expression for $\tilde{u}_{e}$, as given
in (\ref{eq:pol-u-leading}), we rewrite the expressions for
$e_{1},\, e_{2},\, e_{3},\, d_{1},$ and $d_{2}$ in
(\ref{eq:pol-psi-e0123}) and (\ref{eq:pol-eta-d012}), as
\bsub \label{stab:all_ed}
\begin{gather}
e_{1}=\frac{\int w^{3}}{v_{0}^{3/2}}\,,\qquad
 e_{2}=\frac{\int w^{q}}{v_{0}^{q/2}}\,,\qquad
e_{3}=3\int w^{2}\Phi+\frac{U_{0}\sqrt{v_{0}}}{K}\frac{q \int w^{q-1}\Phi}
{\int w^{q}} \,, \\
d_{1}=\hat{\tau}_{u}\lambda\frac{\int w^{q}}{v_{0}^{q/2}},\qquad
 d_{2}=\hat{\tau}_{u}\lambda\left(\frac{U_{0}\sqrt{v_{0}}}{K}
\frac{q\int w^{q-1}\Phi}{\int w^{q}}\right)\,.
\end{gather}
\esub
Upon substituting (\ref{stab:all_ed}) into (\ref{eq:pol-psi_0-pre-1}), we
obtain, after some algebra, that
\begin{equation}
 {\mathcal B}(\lambda) \equiv -\frac{\psi(0)}{v_{0}^{3/2}}=
\frac{1}{\left(1+{v_{0}^{3/2}D_{j}\alpha^2/\int w^{3}}
\right)}\left[\frac{3\int w^{2}\Phi}{\int w^{3}}+
 \frac{v_{0}^{q/2}D_{j}\alpha^q}{v_{0}^{q/2}D_{j}\alpha^q +
 \hat{\tau}_u \lambda \int w^q }\left(\frac{U_{0}\sqrt{v_{0}}}
{K\int w^{3}}\right) \left(\frac{q\int w^{q-1}\Phi}{\int w^{q}}\right)\right]\,. 
\label{eq:pol-chi-1}
\end{equation}

We first consider the synchronous mode for which $j=0$, and ${\mathcal
  D}_0=0$ from (\ref{eq:pol-D_j}). In this case, upon substituting
(\ref{eq:pol-chi-1}) into (\ref{eq:pol-nlep-generic}) we obtain the
following NLEP for the synchronous mode $j=0$:
\begin{equation}\label{stab:nlep_onespot}
L_{0}\Phi-3w^{3}\frac{\int w^{2}\Phi}{\int w^{3}}=\lambda\Phi\,, \qquad
\Phi \to 0 \quad \mbox{as} \quad |y|\to \infty \,.
\end{equation}
From Lemma 3.2 of \cite{kww_crime} it follows that any nonzero
eigenvalue of (\ref{stab:nlep_onespot}) must satisfy
$\mbox{Re}(\lambda)<0$. We summarize this result as follows:

\begin{prop}\label{prop:sync} For $\epsilon\to 0$, $K\geq 2$, $q>1$, 
$0<U_0<U_{0,\textrm{max}}$, ${\mathcal D}_0=\epsilon^2 D = {\mathcal
    O}(1)$, and $\tau_u\ll {\mathcal O}(\eps^{-2})$, a $K$-hotspot
  steady-state solution for (\ref{eq:pol-main}) is always linearly
  stable on an ${\mathcal O}(1)$ time-scale to synchronous
  perturbations of the hotspot amplitudes.
\end{prop}

\begin{rem}\label{rd:sync} 
In the large diffusivity ratio limit, for the two-component
Gierer-Meinhardt and Gray-Scott RD systems in 1-D, the dominant
oscillatory temporal instability in the spike amplitudes is always due
to the synchronous mode (cf.~\cite{mjww_1}, \cite{kww_gs}). In
contrast, for our three-component RD system (\ref{eq:pol-main}),
Proposition \ref{prop:sync} shows that synchronous mode is always
linearly stable.
\end{rem}

Therefore, in our linear stability analysis we need only consider the
asynchronous modes $j=1,\ldots,K-1$, for which $D_j\neq 0$. For these
modes, (\ref{eq:pol-chi-1}) motivates the introduction
of new quantities $\chi_{0,j}$, $\chi_{1,j}$ and ${\mathcal
  C}_q(\lambda)$, defined by
\begin{equation}
\chi_{0,j}\equiv\frac{1}{1+v_{0}^{3/2}D_{j}\alpha^2/\int w^{3}}\,,\qquad
\chi_{1,j}\equiv \left( \frac{U_{0}\sqrt{v_{0}}}{K\int w^{3}}\right)
\frac{\chi_{0,j}}{\mathcal{C}_{q}(\lambda)}\,, \qquad
\mathcal{C}_{q}(\lambda)\equiv 1+\frac{\hat{\tau}_{u}\lambda \int w^q }
{v_{0}^{q/2}D_{j}\alpha^q } \,. \label{eq:pol-chi_0j1j-Cq}
\end{equation}
Then ${\mathcal B}(\lambda)$ in (\ref{eq:pol-chi-1}) can be written
compactly as
\begin{equation}
{\mathcal B}(\lambda)\equiv -\frac{\psi(0)}{v_0^{3/2}} =
\chi_{0,j}\frac{3\int w^{2}\Phi}{\int w^{3}}+
\chi_{1,j} \frac{q\int w^{q-1}\Phi}{\int w^{q}}\,.
\label{eq:pol-chi-2-terms}
\end{equation}
In this way, from (\ref{eq:pol-nlep-generic}) and
(\ref{eq:pol-chi-2-terms}), we obtain an NLEP with two nonlocal terms.
The result is summarized as follows:

\begin{prop}\label{main:stab_0} For $\epsilon\to 0$, $K\geq 2$, $q>1$, 
$0<U_0<U_{0,\textrm{max}}$, ${\mathcal D}_0=\epsilon^2 D = {\mathcal
    O}(1)$, and $\tau_u\ll {\mathcal O}(\eps^{-2})$, the linear
  stability on an ${\mathcal O}(1)$ time-scale of a $K$-hotspot
  steady-state solution for (\ref{eq:pol-main}), for the asynchronous
  modes $j=1,\ldots,K-1$, is characterized by the spectrum of the
  following NLEP for $\Phi(y)$ with two nonlocal terms:
\begin{equation}\label{stab:nlep_old}
  L_{0}\Phi- \chi_{0,j} w^3\frac{3 \int w^2\Phi }{\int w^3} - 
   \chi_{1,j} w^3 \frac{q \int w^{q-1} \Phi}{\int w^q} = \lambda
   \Phi\,, \qquad \mbox{where} \qquad
 L_{0}\Phi \equiv  \Phi^{\prime\prime}-\Phi+3w^{2}\Phi \,.
\end{equation}
Here $\chi_{0,j}$ and $\chi_{1,j}$ are defined in
(\ref{eq:pol-chi_0j1j-Cq}), and $w(y)=\sqrt{2}\sech y$.
\end{prop}

Next, we express $\chi_{0,j}$ and $\chi_{1,j}$ in the NLEP
(\ref{stab:nlep_old}) in terms of the original parameters. To do so,
we first substitute (\ref{eq:pol-v0-formula}) for $v_0$ into the
expression (\ref{eq:pol-chi_0j1j-Cq}) for $\chi_{0,j}$ and
$\chi_{1,j}$. This suggests that it is convenient to introduce two new
quantities $\kappa_q$ and $\omega$ defined by
\begin{equation}\label{stab:kappa_omega}
   \kappa_q \equiv \left( \int w^q \right)^{-1} 
  \left( \frac{\sqrt{2} \pi \alpha K}{\omega} \right)^{q} \,, \qquad
  \mbox{where} \qquad \omega \equiv S (\gamma -\alpha) - U_0 \,.
\end{equation}
In terms of these new variables, (\ref{eq:pol-chi_0j1j-Cq}) becomes
\begin{equation}\label{stab:newchi}
    \chi_{0,j} = \left( 1 + \frac{\kappa_3 D_j}{\alpha} \right)^{-1} \,,
  \qquad \chi_{1,j} = \frac{U_0}{\omega {\mathcal C}_q(\lambda)} \chi_{0,j}
   \,, \qquad 
{\mathcal C}_{q}(\lambda) \equiv 1+\frac{\hat{\tau}_u\lambda}{D_j \kappa_q} \,.
\end{equation} 

Next, we proceed to reformulate (\ref{stab:nlep_old}) as an NLEP
with a single nonlocal term. To do so, we use the special property of
the local operator $L_0$ that $L_{0}w^{2}=3w^{2}$
(cf.~\cite{kww_crime}). Owing to the decay of $\Phi$ and $w$ as
$|y|\to \infty$, and since $L_0$ is self-adjoint, we obtain from
Green's identity that $\int\left(w^{2}L_{0}\Phi-\Phi
L_{0}w^{2}\right)=0$. By using (\ref{stab:nlep_old}) for $L_0\Phi$,
together with $L_0w^2 = 3 w^2$ and the integral ratio
${\int w^5/\int w^3}={3/2}$ from (\ref{ss:int}), we conclude from this
Green's identity that
\begin{equation}\label{stab:key_ident}
   \left( \frac{\int w^2 \Phi}{\int w^3 }\right) \left[
   \frac{9\chi_{0,j}}{2} + (\lambda -3) \right] =
   - \frac{3q\chi_{1,j}}{2} \left( \frac{\int w^{q-1} \Phi}{\int w^q }\right) \,.
\end{equation}

There are several interesting limiting cases of the key identity
(\ref{stab:key_ident}) for any eigenpair of the NLEP
(\ref{stab:nlep_old}) with two nonlocal terms. Since $\chi_{1,j}$ is
proportional to $U_0$ from (\ref{stab:newchi}), we first observe from
(\ref{stab:key_ident}) that for any eigenpair for which $\int w^{m}\Phi
\neq 0$ for any $m>0$, we must have $\lambda=3 - {9\chi_{0,j}/2}$ if
and only if $U_0=0$.  We remark that this recovers the result in
equation (3.17) of \cite{kww_crime} that the unique discrete
eigenvalue of the linearization of a $K$-hotspot steady-state of the
two-component ``basic'' crime model with no police is
\begin{equation}\label{stab:no_police}
    \lambda = 3 - \frac{9 \chi_{0,j}}{2} \,,  
\end{equation}
where $\chi_{0,j}$ is defined in (\ref{stab:newchi}).  By setting
$\lambda=0$ in this expression, the stability threshold in equation
(3.19) of \cite{kww_crime} is recovered. This is discussed in more
detail in \S \ref{sec:stab_compt} below. 

A second special case of (\ref{stab:key_ident}), which is examined in
detail in \S \ref{sec:stab_q3}, is when $q=3$, for which 
(\ref{stab:key_ident}) becomes
\begin{equation}\label{stab:key_iden_3}
   \left( \frac{\int w^2 \Phi}{\int w^3 }\right) \left[
  \frac{9}{2}\left(\chi_{0,j}+\chi_{1,j}\right) + \lambda -3 \right]=0 \,.
\end{equation}
Therefore, when $q=3$, and for any eigenpair $\Phi$ and $\lambda$ of
(\ref{stab:nlep_old}) with $\int w^2 \Phi \neq 0$, we have that
$\lambda$ must satisfy
\begin{equation}\label{stab:q3_lambda}
    \frac{9}{2}\left(\chi_{0,j}+\chi_{1,j}\right) + \lambda -3 =0 \,.
\end{equation}
By using (\ref{stab:newchi}) for $\chi_{1,j}$, we obtain that
(\ref{stab:q3_lambda}) reduces to a family of quadratic equations for
$\lambda$ of the form
\bsub \label{stab:q3_quad}
\begin{equation}
c_{2}\lambda^{2}+c_{1}\lambda+c_{0}=0 \,, \label{stab:q3_quad_1}
\end{equation}
where $c_0$, $c_1$, and $c_2$, are defined for $j=1,\ldots,K-1$ by
\begin{equation}
c_{2}= \frac{\hat{\tau}_u}{3\chi_{0,j} D_j\kappa_3}\,, \qquad
c_{1}=\frac{\hat{\tau}_u}{D_j\kappa_3}
\left(\frac{3}{2}-\frac{1}{\chi_{0,j}}\right) +
\frac{1}{3\chi_{0,j}}\,, \qquad c_{0}=\frac{3U_{0}}{2\omega}+
\frac{3}{2}-\frac{1}{\chi_{0,j}}\,. \label{stab:q3_quad_2}
\end{equation}
\esub In \S \ref{sec:stab_q3} we will analyze the implications of
(\ref{stab:q3_quad}) for the possibility of Hopf bifurcations.

Since $U_0>0$, and since we only consider eigenfunctions for which $\int
w^{m}\Phi \neq 0$ for any $m>0$, we have $\lambda\neq 3
-{9\chi_{0,j}/2}$. Therefore, in (\ref{stab:key_ident}) we can isolate
$\int w^{2}\Phi$ as
\begin{equation*}
\frac{3\int w^{2}\Phi}{\int w^{3}}=\frac{-9}{9\chi_{0,j}+2(\lambda-3)}
\left(\chi_{1,j}\frac{q\int w^{q-1}\Phi}{\int w^{q}}\right) \,.
\end{equation*}
Upon substituting this expression back into (\ref{eq:pol-chi-2-terms})
for $\beta(\lambda)$ we eliminate the nonlocal term $\int w^2 \Phi$, and
obtain that
\begin{equation}\label{stab:nlep_1}
\beta(\lambda) = \chi(\lambda) \frac{\int w^{q-1}\Phi}{\int w^{q}}\,, \qquad
  \mbox{where} \qquad 
  \chi(\lambda) \equiv q \chi_{1,j} 
  \left( \frac{2(\lambda-3)}{9\chi_{0,j}+2(\lambda-3)} \right) \,.
\end{equation}

Finally, by substituting (\ref{stab:newchi}) for $\chi_{0,j}$ and
$\chi_{1,j}$ into (\ref{stab:nlep_1}) we obtain an NLEP with one
nonlocal term:

\begin{prop}\label{main:stab_1} For $\epsilon\to 0$, $K\geq 2$, 
$q>1$, $0<U_0<U_{0,\textrm{max}}$, ${\mathcal D}_0=\epsilon^2 D =
  {\mathcal O}(1)$, and $\tau_u\ll {\mathcal O}(\eps^{-2})$, the
  linear stability on an ${\mathcal O}(1)$ time-scale of a $K$-hotspot
  steady-state solution for (\ref{eq:pol-main}), for the asynchronous
  modes $j=1,\ldots,K-1$, is characterized by the spectrum of the NLEP
  for $\Phi(y)$ given by \bsub \label{stab:nlep_final_1}
\begin{equation}
   L_0 \Phi - \chi(\lambda) w^3 \frac{ \int w^{q-1} \Phi }{\int w^q} =
  \lambda \Phi \,, \qquad \Phi \to 0 \quad \mbox{as} \quad |y|\to \infty \,,
\end{equation}
where $L_{0}\Phi \equiv  \Phi^{\prime\prime}-\Phi+3w^{2}\Phi$. Here the
multiplier $\chi(\lambda)$ of the NLEP is defined by
\begin{equation}\label{stab:nlep_aux}
   \chi(\lambda) \equiv \frac{q U_0}{\omega {\mathcal C}_q(\lambda)} 
  \left(  \frac{(\lambda-3)\chi_{0,j}}{(\lambda -3) + 
 {9\chi_{0,j}/2}}\right) \,, \qquad \mbox{where} \quad 
\frac{1}{\chi_{0,j}}=1+ \frac{\kappa_3 D_j}{\alpha} \,, \quad \mbox{and}\quad
{\mathcal C}_{q}(\lambda) = 1+\frac{\hat{\tau}_u\lambda}{D_j \kappa_q} \,.
\end{equation}
\esub
Here $\kappa_q$ and $\omega$ are defined in (\ref{stab:kappa_omega}),
$D_j$ is defined in (\ref{eq:pol-D_j}), and $\hat{\tau}_u$ is related to
$\tau_u$ by (\ref{eq:pol-tau-hat}).
\end{prop}

\begin{rem} We observe that our NLEP (\ref{stab:nlep_final_1})
has the general form
\[
L_{0}\Phi-\left(\frac{a_{0}+a_{1}\lambda}{b_{0}+b_{1}\lambda+b_{2}\lambda^{2}}
 \right) w^{3}\frac{\int w^{q-1}\Phi}{\int w^{q}}=\lambda\Phi \,,
\]
where the coefficients $a_0$, $a_1$, $b_0$, $b_1$ and $b_2$ are
independent of $\lambda$. To our knowledge there have been no previous
studies of NLEPs in 1-D where the multiplier $\chi$ of the NLEP is a
proper rational function of degree two. Some results of this type are
given in \cite{RRW} for the linear stability analysis of spot patterns
on the sphere for the Brusselator RD system.
\end{rem}

The key model parameters we will use to analyze the NLEP are
$\hat{\tau}_u$, $q$, $U_0$, ${\mathcal D}_0$, and $\omega$. We remark
that $\omega=U_{0,\max}-U_{0}$ where $U_{0,\max}=S(\gamma-\alpha)$ is
the maximum police deployment for which a hotspot steady-state can
exist.

\subsection{Derivation of the NLEP for a Single Hotspot: $K=1$ case}
\label{stab:onespot} 

For $K\geq 2$, the NLEP (\ref{stab:nlep_final_1}) was derived by
imposing Floquet boundary conditions at $x=\pm \ell$. For the case of
a single hotspot, we can impose the Neumann boundary conditions
directly at $x=\pm \ell$, as the Floquet analysis is not needed. With
the same procedure as that leading to (\ref{stab:psi_prob}) and
(\ref{stab:eta_prob}) above, we now obtain 
\begin{equation}
\psi_{xx}  =  0\,, \quad|x|\leq\ell\,; \qquad
e_{0}\left[\psi_{x}\right]_{0}  =  e_{1}\psi(0)+e_{2}\eta(0)+e_{3}\,, \qquad
\psi_{x}(\pm\ell) =0\,, \label{eq:pol-psi-ode-neumann}
\end{equation}
together with the BVP for $\eta(x)$, given by
\begin{equation}
\eta_{xx}  =  0\,, \quad |x|\leq\ell\,; \qquad
d_{0}\left[\eta_{x}\right]_{0}  =  d_{1}\eta(0)+d_{2}\,,
\qquad \eta_{x}(\pm\ell) =0 \,. \label{eq:pol-eta-ode-neumann}
\end{equation}
Here the coefficients $e_{0}$, $e_{1}$, $e_{2}$, $e_{3}$ and $d_{0}$,
$d_{1}$ and $d_{2}$, are as defined in (\ref{eq:pol-psi-e0123}) and
(\ref{eq:pol-eta-d012}), respectively.

From these two problems it immediately follows that $\eta(x)=\eta(0)$
on $|x|\leq \ell$ and that $\eta(0)=-d_{2}/d_{1}$. In addition, we find that
$\psi(x)=\psi(0)$ on $|x|\leq \ell$, with $\psi(0)$ given by
\begin{equation*}
\psi(0) = -\frac{1}{e_{1}}\left(e_{2}\eta(0)+e_{3}\right) =
-\frac{1}{e_{1}}\left(e_{3}-\frac{e_{2}d_{2}}{d_{1}}\right)\,.
\end{equation*}
This is precisely the formula given in (\ref{eq:pol-psi_0-pre-1}) with
$D_{j}$ set to zero.

Therefore, by proceeding in the same way as was done in the Floquet
analysis performed earlier for the $K\geq 2$ case, we simply set
$D_{j}$ to zero in the expression (\ref{eq:pol-chi-1}), and in this
way determine $\beta(\lambda)$ as
\begin{equation}
\beta \equiv-\frac{\psi(0)}{v_{0}^{3/2}}=\frac{3\int w^{2}\Phi}{\int w^{3}}\,. 
\label{eq:pol-chi-neuman-1}
\end{equation}
By substituting (\ref{eq:pol-chi-neuman-1}) into
(\ref{eq:pol-nlep-generic}) we obtain that the NLEP for a single
hotspot solution is given by (\ref{stab:nlep_onespot}). For this NLEP,
Lemma 3.2 of \cite{kww_crime} proves that $\mbox{Re}(\lambda)<0$ for
eigenfunctions for which $\int w^2\Phi\neq 0$.  Therefore, we conclude
that a single hotspot steady-state solution is unconditionally stable
for any ${\mathcal D}_{0}$ when $\tau_{u}\ll {\mathcal
  O}(\epsilon^{-2})$.

\subsection{Reformulation of the NLEP as Zeros of a Meromorphic Function}\label{sec:stab_merom}

We now reformulate our NLEP (\ref{stab:nlep_final_1}) for a $K\geq 2$
hotspot steady-state so that its unstable discrete eigenvalues are the
zeros of a meromorphic function $\zeta(\lambda)$ in the right-half
${\mbox Re}(\lambda)\geq 0$ of the complex plane. To do so, we first
write (\ref{stab:nlep_final_1}) as
\[
\left(L_{0}-\lambda\right)\Phi=
 \chi(\lambda) w^3 \frac{\int w^{q-1}\Phi}{\int w^{q}} \,, \qquad
\mbox{so that} \qquad 
\Phi=\chi(\lambda) \frac{\int w^{q-1}\Phi}{\int w^{q}}
\left(L_{0}-\lambda\right)^{-1}w^{3}\,.
\]
We then multiply both sides of this expression by $w^{q-1}$ and
integrate to get
\begin{equation}
\left(\int w^{q-1}\Phi\right)\left[1- \chi(\lambda) 
\frac{\int w^{q-1}\left(L_{0}-\lambda\right)^{-1}w^{3}}{\int w^{q}}\right]=0\,.
\label{eq:pol-mathcal-F}
\end{equation}
Provided that the eigenfunction satisfies $\int w^{q-1}\Phi\neq0$, we
conclude that an eigenvalue $\lambda$ of the NLEP
(\ref{stab:nlep_final_1}) must be a root of 
\bsub \label{stab:merom}
\begin{equation}\label{stab:merom_1}
\zeta(\lambda)\equiv\mathcal{C}(\lambda)-\mathcal{F}(\lambda)=0\,, 
 \qquad \mbox{where} \qquad \mathcal{F}(\lambda)\equiv
\frac{\int w^{q-1}\left(L_{0}-\lambda\right)^{-1}w^{3}}{\int w^{q}}\,.
\end{equation}
Here ${\mathcal C}(\lambda)\equiv \left[\chi(\lambda)\right]^{-1}$, and
in terms of $\chi_{0,j}$  and ${\mathcal C}_q(\lambda)$, as defined in
(\ref{stab:nlep_aux}), we have
\begin{equation}
 {\mathcal C}(\lambda) = \frac{\omega {\mathcal C}_q(\lambda)}{q U_0}
  \left( \frac{1}{\chi_{0,j}} + \frac{9}{2(\lambda -3)} \right) \,.
\label{stab:merom_2}
\end{equation}
\esub

We will proceed to analyze the zeros of the meromorphic function
$\zeta(\lambda)\equiv \mathcal{C}(\lambda)-\mathcal{F}(\lambda)$ in
two cases: $q=3$ and $q>1$, with the former being explicitly solvable,
and the latter requiring the Nyquist criterion to count the number of
zeros in the unstable right half-plane
$\mbox{Re}(\lambda)>0$. Moreover, we will also investigate the
possibility of a Hopf bifurcation, by seeking a pure imaginary root of
the form $\lambda=\pm i\lambda_{I}$ to (\ref{stab:merom}) with
$\lambda_{I}>0$. Since $j=1,\ldots,K-1$, such a Hopf bifurcation will
correspond to an asynchronous temporal oscillation of the hotspot
amplitudes.

\begin{rem}
When $\int w^{q-1}\Phi=0$, the NLEP (\ref{stab:nlep_final_1}) reduces
to the local eigenvalue problem $L_{0}\Phi=\lambda\Phi$ with the extra
condition $\int w^{q-1}\Phi=0$.  From Proposition 5.6 of \cite{dgk_0},
$L_0$ has exactly two discrete eigenvalues. One is $\Phi=w^2$
with $\lambda=3$, arising from the identity $L_0 w^2=3w^2$, for which
$\int w^{q-1}\Phi\neq 0$, while the other is the odd eigenfunction
$\Phi=w^{\prime}$ for which $\lambda=0$ and $\int
w^{q-1}\Phi=0$. Therefore, since there are no instabilities associated
with modes for which $\int w^{q-1}\Phi=0$, the zeroes of
$\zeta(\lambda)$, as defined in (\ref{stab:merom}), in
$\mbox{Re}(\lambda)>0$ will determine any instability of the
$K$-hotspot steady-state with $K\geq 2$.
\end{rem}

\setcounter{equation}{0}
\setcounter{section}{3}
\section{Analysis of the NLEP: Competition Instability}\label{sec:stab_comp}

In order to analyze zero-eigenvalue crossings for the NLEP
(\ref{stab:nlep_final_1}), as well as the possibility of Hopf
bifurcations, in \S \ref{sec:nlep_rig} we need to provide some global
properties of ${\mathcal F}(\lambda)$, as defined in
(\ref{stab:merom_1}), on both the non-negative real axis $\lambda\geq
0$ and on the non-negative imaginary axis $\lambda=i\lambda_I$ with
$\lambda_I\geq 0$.  In \S \ref{sec:stab_arg} we apply the winding
number criterion of complex analysis to count the number of zeroes of
$\zeta(\lambda)$, defined in (\ref{stab:merom}), in the unstable right
half-plane $\mbox{Re}(\lambda)>0$.  With these properties, in \S
\ref{sec:stab_compt} we study the competition stability threshold
characterized by zero-eigenvalue crossings of the NLEP
(\ref{stab:nlep_final_1}). Oscillatory instabilities for $q=3$ and for
general $q>1$ due to a Hopf bifurcation are studied in detail in \S
\ref{sec:stab_q3} and \S \ref{sec:stab_qn3}, respectively.

Before summarizing the global properties of ${\mathcal F}(\lambda)$,
we first show that $\mathcal{F}(\lambda)$ can be found explicitly when
$q=3$ by using the identity $L_{0}w^{2}=3w^{2}$. When $q=3$, we
calculate the integral $I$ in the numerator for
$\mathcal{F}(\lambda)$, given in (\ref{stab:merom_1}), as
\begin{equation*}
I\equiv\int w^{2}\left(L_{0}-\lambda\right)^{-1}w^{3}= \frac{1}{3}
   \int \left( L_0 w^2 \right) \left(L_{0}-\lambda\right)^{-1}w^{3} =
  \int \left[ (L_0 -\lambda) w^2 + \lambda w^2 \right]
  \left(L_{0}-\lambda\right)^{-1}w^{3} \,.
\end{equation*}
Upon integrating this last expression by parts, we get the algebraic
equation $ I = {\left(\int w^5 + \lambda I\right)/3}$, so that
$I={\int w^5 /(3-\lambda)}$. Then, since ${\mathcal F}={I/\int w^3}$
and ${\int w^5/\int w^3}={3/2}$ from (\ref{ss:int}), we conclude that
\begin{equation}
\mathcal{F}(\lambda)=\frac{3}{2 (3-\lambda)} \,, \qquad \mbox{when} \quad
 q=3 \,. \label{eq:exactF_q3}
\end{equation}

\subsection{Key Global and Asymptotic Properties of $\mathcal{F}(\lambda)$}
\label{sec:nlep_rig}

We first recall some key properties of ${\mathcal F}(\lambda)$, defined
in (\ref{stab:merom_1}), on the non-negative real axis $\lambda\geq 0$.

\begin{proposition}\label{rig:real_f} On the non-negative real axis 
$\lambda \geq 0$, ${\mathcal F}(\lambda)$ given in (\ref{stab:merom_1})
satisfies
\begin{itemize}
\item [{(i)}] ${\mathcal F}(\lambda) \sim \frac{1}{2} + \frac{\lambda}{4}
 \left( 1- \frac{1}{q} \right) + {\mathcal O}(\lambda^2) 
    \quad \mbox{as} \quad \lambda \to 0 \,$.
\item [{(ii)}] ${\mathcal F}(\lambda) \to + \infty \quad \mbox{as}
  \quad \lambda \to 3^{-}\,$.
\item [{(iii)}] ${\mathcal F}^{\prime}(\lambda) >0\,, \quad \mbox{for}  
 \quad 0 < \lambda < 3,\,\,$ when $q=2,3,4\,$.
\item [{(iv)}] ${\mathcal F}(\lambda)<0\,, \quad \mbox{for} \quad 
\lambda>3 \,$.
\end{itemize}
\end{proposition}

\begin{proof} The statements in (i), (ii), and (iv), as well as in
(iii) for $q=2$ and $q=4$, were proved in Proposition 3.5 of
  \cite{mjww_1}. For $q=3$, the monotonicity result in (iii) is seen
  to hold by using the explicit form for ${\mathcal F}(\lambda)$
  given in (\ref{eq:exactF_q3}).
\end{proof}

\begin{figure}[htbp]
\centering
\includegraphics[width=9cm,height=5.0cm]{figs/ftreal}
\caption{\label{fig:conj_real}Plot of
  $\mathcal{F}(\lambda)$ on $0<\lambda<3$ for $q=2,3,4,5$. Note
  that $\mathcal{F}(0)={1/2}$ and that ${\mathcal F}(\lambda)\to +\infty$
  as $\lambda\to 3$ from below. We observe that $\mathcal{F}(\lambda)$ is 
  rather insensitive to changes in $q$.}
\end{figure}

In addition to the results (i), (ii), and (iv), which hold for all
$q>1$, we conjecture that the monotonicity result in (iii) holds not
just for $q=2,3,4$ but for all $q>1$. As additional support
of this conjecture, in Fig.~\ref{fig:conj_real} we plot the
numerically computed function ${\mathcal F}(\lambda)$ on $0<\lambda<3$
for $q=2,3,4,5$. 

\begin{conj}\label{conj:real} The monotonocity property (iii) of Proposition 
 \ref{rig:real_f} that ${\mathcal F}^{\prime}(\lambda)>0$ on
  $0<\lambda< 3$ holds for all $q>1$.
\end{conj}

Next, in order to count the number of eigenvalues of the NLEP 
(\ref{stab:nlep_final_1}) in $\mbox{Re}(\lambda)>0$ below, we need some
properties of $\mathcal{F}(\lambda)$, as defined in
(\ref{stab:merom_1}), as restricted to the non-negative imaginary axis.
By rewriting the operator as
\begin{equation*}
(L_{0}-i\lambda_{I})^{-1} = \left(L_{0}+i\lambda_{I}\right)
\left[\left(L_{0}+i\lambda_{I}\right)^{-1}(L_{0}-i\lambda_{I})^{-1}\right]
  = L_{0}\left[L_{0}^{2}+\lambda_{I}^{2}\right]^{-1}+
  i\lambda_{I}\left[L_{0}^{2}+\lambda_{I}^{2}\right]^{-1} \,,
\end{equation*}
we readily obtain upon upon separating 
${\mathcal F}(i\lambda_I)={\int w^{q-1}(L_{0}-i\lambda_{I})^{-1}w^{3}/
\int w^{q}}$ into real and imaginary parts that
\begin{equation}\label{eq:pol-F_R-F_I}
\mathcal{F}(i\lambda_{I})= {\mathcal F}_R(\lambda_I) + 
i {\mathcal F}_I(\lambda_I) \,; \qquad
\mathcal{F}_{R}(\lambda_{I})  =  \frac{\int w^{q-1}L_{0}
\left[L_{0}^{2}+\lambda_{I}^{2}\right]^{-1}w^{3}}{\int w^{q}} \,, \qquad
\mathcal{F}_{I}(\lambda_{I})  =  \frac{\lambda_{I}\int w^{q-1}
\left[L_{0}^{2}+\lambda_{I}^{2}\right]^{-1}w^{3}}{\int w^{q}} \,. 
\end{equation}
We then recall some rigorous results of \cite{mjww_1} for
$\mathcal{F}_{R}(\lambda_{I})$ and $\mathcal{F}_{I}(\lambda_{I})$ on
$\lambda_I\geq 0$.

\begin{proposition}\label{rig:imag_f} For $\lambda=i\lambda_I$ with
$\lambda_I> 0$, we have that
${\mathcal F}_R(\lambda_I)$ and ${\mathcal F}_I(\lambda_I)$ satisfy
\begin{itemize}
\item [{(i)}] $\mathcal{F}_{R}(\lambda_{I})={\mathcal O}(\lambda_{I}^{-2})
\quad \mbox{as}\quad \lambda_{I}\to+\infty\,,\quad\mathcal{F}_{R}(0)=1/2$.
\item [{(ii)}] $\mathcal{F}_{R}^{\prime}(\lambda_{I})<0\,, \quad \mbox{when}
\quad q=2,3 \,.$
\item [{(iii)}] $\mathcal{F}_{I}(\lambda_{I})=
{\mathcal O}(\lambda_{I}^{-1})\quad \mbox{as}\quad \lambda_{I}\to+\infty$.
\item [{(iv)}] $\mathcal{F}_{I}(\lambda_{I})\sim
\frac{\lambda_{I}}{4}\left(1-\frac{1}{q}\right)>0\quad \mbox{as}\quad
\lambda_{I}\to 0^{+}\,$.
\item [{(v)}] $\mathcal{F}_{I}(\lambda_{I})>0\,, \quad \mbox{when}\quad q=2,3,4$. 
\end{itemize}
\end{proposition}
\begin{proof} The statement in (i), and in (ii) for $q=2$, were proved
in Proposition 3.1 of \cite{mjww_1}; Statements (iii), (iv), and (v)
for $q=2,4$, were proved in Proposition 3.2 of \cite{mjww_1}. The
results in (ii) and (v) for $q=3$ follow by using the explicit formula
in (\ref{eq:exactF_q3}). For $q=3$, we have
$\mathcal{F}(i\lambda_{I}) = {3/\left[2({3-i\lambda_{I}})\right]}$, so
that
\begin{equation}
\mathcal{F}_{R}(\lambda_{I})=\frac{9}{2(9+\lambda_{I}^{2})}\,,\quad
\mathcal{F}_{R}^{\prime}(\lambda_{I})=-\frac{9\lambda_{I}}
        {\left(9+\lambda_{I}^{2}\right)^{2}}\,,\quad
        \mathcal{F}_{I}(\lambda_{I})=
        \frac{3\lambda_{I}}{2(9+\lambda_{I}^{2})}\,, \quad
\mathcal{F}_{I}^{\prime}(\lambda_{I})=
\frac{3(9-\lambda_I^2)}{2(9+\lambda_I^2)^2} \,, \quad
        \mbox{for} \quad q=3 \,.
\label{eq:pol-F_R-F_I-q=00003D3}
\end{equation}
This clearly shows that properties (ii) and (v) also hold for $q=3$.
Moreover, it follows that $\mathcal{F}_{I}$ has a unique local maximum
at the principal eigenvalue $\lambda_I=3$ of $L_{0}$.
\end{proof}

Although we only have a rigorous proof that
$\mathcal{F}_{R}^{\prime}(\lambda_{I})<0$ when $q=2,3$ and that
$\mathcal{F}_{I}(\lambda_{I})>0$ when $q=2,3,4$, we now conjecture
that these key properties hold for all $q>1$.  In
Fig.~\ref{fig:pol-F_R-F_I-for-q=00003D2to5} we plot the numerically
computed functions $\mathcal{F}_{R}(\lambda_{I})$ and
$\mathcal{F}_{I}(\lambda_{I})$ for various values of $q$, which give
numerical evidence for this conjecture. From this figure we observe that
 $\mathcal{F}_{R}(\lambda_{I})$ is rather insensitive to changes in $q$.

\begin{conj}\label{conj:imag}
\label{conj:pol-F(lambda)} Properties (ii) and (v) in Proposition
\ref{rig:imag_f} that $\mathcal{F}_{R}^{\prime}(\lambda_{I})<0$ and
$\mathcal{F}_{I}(\lambda_{I})>0$ on $\lambda_I>0$ hold for all
$q>1$.
\end{conj}

\begin{figure}[htbp]
\centering
\includegraphics[width=9cm,height=5.0cm]{figs/freal}
\includegraphics[width=9cm,height=5.0cm]{figs/fimag}
\caption{\label{fig:pol-F_R-F_I-for-q=00003D2to5}Plots of
  $\mathcal{F}_{R}(\lambda_I)$ (left panel) and
  $\mathcal{F}_{I}(\lambda_I)$ (right panel) for $q=2,3,4,5$. Note
  that $\mathcal{F}_{R}(0)=1/2$ and $\mathcal{F}_{I}(0)=0$, and that
  the maximum of $\mathcal{F}_{I}$ occurs near $\lambda_{I}=3$. In
  fact, the maximum does occur exactly at $\lambda_{I}=3$ when $q=3$.}
\end{figure}

\subsection{A Winding Number Criterion for the Number of Unstable Eigenvalues
of the NLEP}\label{sec:stab_arg}

We now use the argument principle of complex analysis to count the
number $N$ of eigenvalues of the NLEP (\ref{stab:nlep_final_1}) in
$\mbox{Re}(\lambda)>0$. For each $j=1,\ldots,K-1$, these discrete
eigenvalues are the complex zeroes of the function
$\zeta(\lambda)\equiv \mathcal{C}(\lambda)-\mathcal{F}(\lambda)$, as
defined in (\ref{stab:merom}). Here ${\mathcal F}(\lambda)$ is defined
in (\ref{stab:merom_1}) and from (\ref{stab:merom_2}) we have that
${\mathcal C}(\lambda)$ has the explicit form 
\bsub \label{stab:Cval}
\begin{equation}
\mathcal{C}(\lambda)=a(1+\tilde{\tau}_{j}\lambda)
\left(1-\frac{b}{3-\lambda}\right)\,, \label{stab:Cval_1}
\end{equation}
where $a$, $b$, and $\tilde{\tau}_{j}$, are defined for $j=1,\ldots,K-1$ by
\begin{equation} 
a\equiv \frac{\omega}{qU_{0}\chi_{0,j}} \,, \qquad
b\equiv \frac{9}{2} \chi_{0,j}\,, \qquad
\tilde{\tau}_{j} \equiv \frac{\hat{\tau}_{u}}{D_j \kappa_q} \,, \qquad
 \frac{1}{\chi_{0,j}} = 1 + \frac{\kappa_3 D_j}{\alpha} \,.
\label{stab:Cval_2}
\end{equation}
\esub
Here $\omega$ and $\kappa_q$ are given in (\ref{stab:kappa_omega}),
while $D_j$ and $\hat{\tau}_{u}$ are defined in (\ref{eq:pol-D_j}) and
\ref{eq:pol-tau-hat}, respectively.

From (\ref{stab:Cval_1}), $\mathcal{C}(\lambda)$ is a meromorphic
function with a simple pole at $\lambda=3$. Moreover,
$\mathcal{F}(\lambda)$ is analytic in $\mbox{Re}(\lambda)\geq 0$
except at the simple pole at $\lambda=3$. The simple poles of
$\mathcal{C}(\lambda)$ and $\mathcal{F}(\lambda)$ do not cancel as
$\lambda\to3^{-}$, since when restricted to the real line we get
$\mathcal{F}(\lambda)\to+\infty$ while
$\mathcal{C}(\lambda)\to-\infty$ as $\lambda\to 3^{-}$. Thus,
$\zeta(\lambda)=\mathcal{C}(\lambda)-\mathcal{F}(\lambda)$ has a
simple pole at $\lambda=3$.

\begin{figure}[htbp]
\centering
\includegraphics[width=13.0cm,height=5.0cm]{figs/nyq_cont}
\caption{\label{fig:nyquist}Schematic plot of the Nyquist contour
  $\Gamma$ used for determining the number $N$ of unstable eigenvalues
  of the NLEP (\ref{stab:nlep_final_1}).}
\end{figure}

To determine a formula for $N$, we calculate the winding of
$\zeta(\lambda)$ over the Nyquist contour $\Gamma$ traversed in the
counterclockwise direction that consists of the following segments in
the complex $\lambda$-plane (see the schematic in
Fig.~\ref{fig:nyquist}): $\Gamma_I^+$ ($0<\textrm{Im}(\lambda)<iR$,
$\textrm{Re}(\lambda)=0$), $\Gamma_I^-$ ($-iR<\textrm{Im}(\lambda)<0$,
$\textrm{Re}(\lambda)=0$), and $C_R$ defined by $|\lambda| = R >
0$ for $|\arg(\lambda)| < \pi/2$.

For each $j=1,\ldots,K-1$, $\zeta(\lambda)$ is analytic in
$\mbox{Re}(\lambda)\geq 0$ except at the simple pole $\lambda = 3$
corresponding to the unique positive eigenvalue of the local operator
$L_0$. Therefore, for each $j=1,\ldots,N-1$, and assuming that
$\zeta(\lambda)$ has no zeroes on the imaginary axis, we have by the
argument principle that $N = 1 + (2\pi)^{-1}\lim_{R\to\infty} \left[\arg
  \zeta \right]_\Gamma$, where $\left[\arg \zeta \right]_\Gamma$
denotes the change in the argument of $\zeta$ over $\Gamma$.  Since
${\mathcal F}(\lambda)={\mathcal O}(|\lambda|^{-1})$ on the
semi-circle $C_R$ as $R=|\lambda|\to \infty$, we have that
$\lim_{R\to\infty}\left[\arg \zeta \right]_{C_R}=
\lim_{R\to\infty}\left[\arg {\mathcal C} \right]_{C_R}$. From
(\ref{stab:Cval_1}) we calculate that $\lim_{R\to\infty}\left[\arg
  {\mathcal C} \right]_{C_R}=\pi$ when $\hat{\tau}_u>0$, and
$\lim_{R\to\infty}\left[\arg {\mathcal C} \right]_{C_R}=0$ when
$\hat{\tau}_u=0$.  For the contour $\Gamma_I^-$, we use that
$\zeta(\overline{\lambda}) = \overline{\zeta(\lambda)}$ so that
$\left[\arg \zeta \right]_{\Gamma_I^-} = \left[\arg
  \zeta\right]_{\Gamma_I^+}$.  In this way, for each $j=1,\ldots,K-1$,
we conclude that
\begin{equation}
  N = \frac{3}{2} + \frac{1}{\pi} \left[\arg \zeta \right]_{\Gamma_I^{+}} \,,
 \quad \mbox{for} \quad \hat{\tau}_u>0 \,; \qquad
  N = 1 + \frac{1}{\pi} \left[\arg \zeta \right]_{\Gamma_I^{+}} \,,
 \quad \mbox{for} \quad \hat{\tau}_u=0 \,. \label{key:wind}
\end{equation}
Here $\left[\arg \zeta \right]_{\Gamma_I^{+}}$ denotes the change
in the argument of $\zeta$ as the imaginary axis $\lambda=i\lambda_I$ is
traversed from $\lambda_I=+\infty$ to $\lambda_I=0$.

We remark that (\ref{key:wind}) determines the number of unstable
eigenvalues of the NLEP (\ref{stab:nlep_final_1}) for any {\em
  specific} asynchronous mode $j=1,\ldots,K-1$. The total number of
such unstable eigenvalues, for all asynchronous modes, is simply the
union of (\ref{key:wind}) over $j=1,\ldots,K-1$. In this way, the
problem of determining $N$ for a particular mode $j$ is reduced to
calculating the change of argument of $\zeta(\lambda)= {\mathcal
  C}(\lambda)-{\mathcal F}(\lambda)$ as we traverse down the positive
imaginary axis. To do so, we will need the properties of ${\mathcal
  F}(i\lambda_I)$ given in Proposition \ref{rig:imag_f}, together with
results for ${\mathcal C}(i\lambda)$ to be obtained from
(\ref{stab:Cval}).

\subsection{The Competition Instability Threshold}\label{sec:stab_compt}

We now determine the competition instability threshold value of the
diffusivity ${\mathcal D}_0$, which is characterized by a zero-eigenvalue 
crossing of the NLEP (\ref{stab:nlep_final_1}). Since
${\mathcal F}(0)={1/2}$ (see (i) of Proposition \ref{rig:real_f}), we
conclude that $\zeta(0)=0$ when ${\mathcal C}(0)={1/2}$. From using
(\ref{stab:merom_2}), or equivalently (\ref{stab:Cval}), for
${\mathcal C}(\lambda)$ we conclude that $\lambda=0$ when
\begin{equation}\label{stab:zero_1}
   \frac{\omega}{qU_0} \left( \frac{1}{\chi_{0,j}} - \frac{3}{2}\right)=
\frac{1}{2} \,, \qquad j=1,\ldots,K-1\,.
\end{equation}
By using (\ref{stab:Cval_2}) for $\chi_{0,j}$, together with
(\ref{stab:kappa_omega}) for $\kappa_3$, (\ref{stab:zero_1}) yields
that $\lambda=0$ when
\begin{equation}\label{stab:zero_2}
   D_j = \frac{\omega^3}{4\pi^2 K^3 \alpha^2} 
\left( 1+ \frac{qU_0}{\omega} \right) \,, \qquad j=1,\ldots,K-1 \,.
\end{equation}
Finally, by using $D_j=2 K {\mathcal D}_0{\left(1-\cos({\pi j/K})\right)/S}$, as
obtained from (\ref{eq:pol-D_j}), we conclude that the NLEP has a 
zero-eigenvalue crossing at the $K-1$ distinct values ${\mathcal D}_{0,j}$
of ${\mathcal D}_0$ given by
\begin{equation}\label{stab:zero_final}
    {\mathcal D}_{0,j} = \frac{\omega^3 S}{8 \pi^2 \alpha^2 K^4 
  \left(1 - \cos\left({\pi j/K}\right)\right)} \left( 1 + \frac{q U_0}{\omega}
  \right) \,, \qquad j=1,\ldots,K-1\,.
\end{equation}
As we show in Proposition \ref{prop:comp_t} below, the competition
instability threshold ${\mathcal D}_{0,c}$ corresponds to the
smallest such ${\mathcal D}_{0,j}$, which occurs when $j=K-1$. This yields that
\begin{equation}\label{stab:zero_d0min}
   {\mathcal D}_{0,c}\equiv {\mathcal D}_{0,K-1} = \frac{\omega^3 S}{8
     \pi^2 \alpha^2 K^4 \left(1 + \cos\left({\pi /K}\right)\right)}
   \left( 1 + \frac{q U_0}{\omega} \right) \,.
\end{equation}
In terms of the unscaled diffusivity $D=\epsilon^{-2}{\mathcal D}_0$,
the competition stability threshold occurs at
$D_{c}\equiv \epsilon^{-2}{\mathcal D}_{0,c}$.

\begin{rem}\label{stab:zero_two_nonlocal} The zero-eigenvalue crossing
condition (\ref{stab:zero_1}) can also be obtained from the NLEP
(\ref{stab:nlep_old}) with two nonlocal terms by setting $\Phi=w$ and
$\lambda=0$ in (\ref{stab:nlep_old}). By using the identity $L_0 w= 2
w^3$, this substitution yields $2-3\chi_{0,j} - q \chi_{1,j}=0$, where
from (\ref{stab:newchi}) we have $\chi_{1,j}={U_0 \chi_{0,j}/\omega}$
at $\lambda=0$. Some simple algebra then recovers (\ref{stab:zero_1}).
\end{rem}

\noindent We now prove an instability result related to
zero-eigenvalue crossings:

\begin{prop}
\label{prop:comp_t} For $\epsilon\to 0$, $K\geq 2$, $q>1$, 
$0<U_0<U_{0,\textrm{max}}$, ${\mathcal D}_0=\epsilon^2 D = {\mathcal
  O}(1)$, a $K$-hotspot steady-state solution for (\ref{eq:pol-main})
is unstable for all $\hat{\tau}_u\geq 0$ when ${\mathcal
  D}_0>{\mathcal D}_{0,c}$, where ${\mathcal D}_{0,c}$ is the
competition stability threshold defined in
(\ref{stab:zero_d0min}). For ${\mathcal D}_0<{\mathcal D}_{0,c}$, a
$K$-hotspot steady-state is linearly stable when $\hat{\tau}_u=0$ and
$q=2,3,4$.
\end{prop}

\begin{proof} We first prove that when ${\mathcal D}_0>{\mathcal D}_{0,c}$,
then $\zeta(\lambda)=0$ in (\ref{stab:merom}) has a positive real root
in $0<\lambda<3$ for each $j=1,\ldots,K-1$. This readily follows from
the fact that ${\mathcal C}(0)>{1/2}$ for each $j=1,\ldots,K-1$, that
${\mathcal C}(\lambda)\to -\infty$ as $\lambda\to 3^{-}$, and
from Proposition \ref{rig:real_f} where we have ${\mathcal
  F}(0)={1/2}$ and ${\mathcal F}(\lambda)\to +\infty$ as $\lambda\to
3^{-}$. Thus, by continuity, there is at least one positive real root
to $\zeta(\lambda)=0$ on $0<\lambda<3$ for each $j=1,\ldots,K-1$ and
for any $\hat{\tau}_u\geq 0$. Next, for ${\mathcal D}_0<{\mathcal
  D}_{0,c}$, we show that $N=0$ by using the winding number
criterion (\ref{key:wind}) and calculating $\left[\arg \zeta
  \right]_{\Gamma_I^{+}}$ explicitly.  From (\ref{stab:Cval_1}), we
decompose ${\mathcal C}(i\lambda_I)={\mathcal C}_R(\lambda_I) +
i{\mathcal C}_I(\lambda_I)$, and for $\hat{\tau}_u=0$ calculate that
\begin{equation*}
   {\mathcal C}_I(\lambda_I) = -\frac{ a b \lambda_I}{9+\lambda_I^2} < 0
\qquad \mbox{for} \quad \lambda_I>0 \,.
\end{equation*}
Since ${\mathcal F}_I(\lambda_I)>0$ for $\lambda_I>0$ for $q=2,3,4$ from
property (v) of Proposition \ref{rig:imag_f}, we conclude that
$\mbox{Im}\left[\zeta(i\lambda_I)\right]<0$ for $\lambda_I>0$. Then, since
${\mathcal C}(0)>{1/2}$ when ${\mathcal D}_0<{\mathcal D}_{0,c}$,
and ${\mathcal F}(0)={1/2}$ from (i) of Proposition \ref{rig:real_f}, 
we have $\zeta(0)>0$ for each $j=1,\ldots,K-1$, and $\zeta(i\lambda_I)
\to {\omega/(qU_0\chi_{0,j})}>0$ as $\lambda_I\to +\infty$. It follows that
$\left[\arg \zeta \right]_{\Gamma_I^{+}}=-\pi$, and consequently $N=0$ from
the second statement in (\ref{key:wind}) for $\hat{\tau}_u=0$.
\end{proof}

We remark that if Conjecture \ref{conj:imag} holds, then a $K$-hotspot
steady-state is linearly stable when ${\mathcal D}_0<{\mathcal
  D}_{0,c}$ and $\hat{\tau}_u=0$ for any $q>1$. Moreover, by
continuity of eigenvalue paths in $\hat{\tau}_u$, the stability result
in Proposition \ref{prop:comp_t} should hold for all $\hat{\tau}_u>0$
but sufficiently small. The possibility of Hopf bifurcations values of
$\hat{\tau}_u$ for the range ${\mathcal D}_0<{\mathcal D}_{0,c}$ is
examined for $q=3$ in \S \ref{sec:stab_q3} and for general $q>1$ in
\S \ref{sec:stab_qn3}.

\subsection{Qualitative Interpretation of the Competition Instability Threshold}\label{sec:qual_comp_d}

Next, we discuss the qualitative behavior of the competition instability
threshold ${\mathcal D}_{0,c}$ with respect to the degree $q$
of patrol focus and the total level $U_0$ of police patrol deployment.

We recall from (\ref{eq:amax}) that the maximum $A_{\max}$ of the
steady-state attractiveness field is $A_{\max}\sim
\epsilon^{-1}{\omega/(K\pi)}$, which decreases as either $\omega$
decreases or $K$ increases. However, from Corollary
\ref{cor:pol-K-hotspots-in-A-rho-U}, the amplitude of the steady-state
criminal density $\rho$ at the hotspot locations is
$\rho_{\max}=\left[w(0)\right]^2=2$, which is independent of all model
parameters, while away from the maxima of $A$ the criminal density is
${\mathcal O}(\epsilon^{2})\ll 1$. Therefore, it is the reduction of
the number of stable steady-state hotspots on a given domain length
that is the primary factor in reducing the total crime in the
domain. As such, we seek to tune the police parameters $q$ and $U_{0}$
so that the range of diffusivity ${\mathcal D}_{0}$ for which a
$K$-hotspot steady-state is unconditionally unstable, i.e.~unstable
for all $\hat{\tau}_u>0$, is as large as possible. This corresponds to
minimizing the competition stability threshold ${\mathcal D}_{0,c}$
in (\ref{stab:zero_d0min}).

From (\ref{stab:zero_d0min}), we observe that ${\mathcal D}_{0,c}$
increases with $q$ in a linear fashion. Within the context of our RD
model (\ref{eq:pol-main}), this predicts that if the police become
increasingly focused on patrolling the more crime-attractive areas,
then paradoxically the range of ${\mathcal D}_{0}$ where a $K$-hotspot
steady-state is unstable decreases. Therefore, for the goal of
reducing the number of stable crime hotspots, a police deployment with
intense focus on crime-attractive areas does not offer an advantage
over that of a less focused patrol (assuming that $q>1$ for our
analysis to hold). At a fixed level $U_0$ of police deployment, and for
integer values of $q$ with $q>1$, the best patrol strategy
is to take $q=2$, which corresponds to the ``cops-on-the-dots''
strategy (cf.~\cite{jbc}, \cite{rick}, \cite{zipkin}) where the police
mimic the movement of the criminals.

For a fixed $q>1$, we next examine how the competition stability
threshold for a $K$-hotspot steady-state changes with the total police
deployment $U_{0}$.  To this end, we substitute
$U_{0}=S(\gamma-\alpha) - \omega$ into (\ref{stab:zero_d0min}),
and write ${\mathcal D}_{0,c}$ as
\begin{equation}
   {\mathcal D}_{0,c} = \frac{S}{8 \pi^2 \alpha^2 K^4 \left(1
      + \cos\left({\pi /K}\right)\right)} \, g(U_0) \,, \qquad
 g(U_0) \equiv\omega^{3}(1-q)+qS(\gamma-\alpha)\omega^{2}\,; \qquad
   \omega \equiv S(\gamma-\alpha) - U_0 \,. \label{eq:pol-D0*-g(w)}
\end{equation}
To analyze the critical points of $g(U_0)$, we first observe that 
${d\omega/dU_{0}}=-1$ and that $U_{0}\to U_{0,\max}=S(\gamma-\alpha)$ as 
$\omega \to 0$. We then calculate that
\begin{equation*}
\frac{dg}{dU_{0}}  =  -3 (1-q) \omega \left(\omega-\omega_c\right) \,,
\qquad \mbox{where}\qquad \omega_{c}\equiv 
\frac{2qS(\gamma-\alpha)}{3(q-1)}\,.
\end{equation*}
We conclude that $\omega_c<S(\gamma-\alpha)$, so that
$0<U_0<U_{0,\max}$, iff $q>3$. Therefore, $g(U_0)$ has a unique maximum
point on $0<U_0<U_{0,\max}$ iff $q>3$.  Alternatively, for $q\leq 3$
we have ${dg/dU_{0}}<0$ for $0<U_{0}<U_{0,\max}$.

This shows that if the degree of patrol focus $q$ satisfies $q\leq3$,
then ${\mathcal D}_{0,c}$ is monotonically decreasing in
$U_{0}$. Therefore, for this range of $q$, increasing the level $U_0$
of police deployment leads to a larger range of ${\mathcal D}_0$ where
the $K$-hotspot steady-state is unconditionally unstable.  However, if
$q>3$, then initially as the level of police deployment is increased
from zero, the range of ${\mathcal D}_0$ where the steady-state
hotspot pattern is unstable is decreased, until the critical value
\begin{equation}\label{stab:optim_u0}
U_{0,c} \equiv
S(\gamma-\alpha)-\omega_{c}=S(\gamma-\alpha)\frac{(q-3)}{3(q-1)} \,,
\qquad q>3 \,,
\end{equation}
is reached. For $U_0>U_{0,c}$, the hotspot pattern becomes less stable
when the policing level is increased. These qualitative results are
displayed graphically in Fig.~\ref{fig:gU0-vs-U0}.

Finally, we can interpret our competition stability threshold in terms
of a critical threshold $K_{c}$ for which a steady-state pattern of
$K\geq 2$ hotspots is unconditionally unstable when $K>K_{c}$.  This
instability, which develops on an ${\mathcal O}(1)$ time scale as
$\eps\to 0$, is due to a real positive eigenvalue of the NLEP, and as
we show from full numerical simulations in \S \ref{sec:numerics_q3}
and in \S \ref{sec:numerics_qn3} it triggers the collapse of some of
the hotspots in the pattern. By writing (\ref{eq:pol-D0*-g(w)}) in
terms of $K$, this critical threshold $K_{c}>0$ when $D={{\mathcal
    D}_0/\epsilon^2}$, and where $g(U_0)$ is defined in
(\ref{eq:pol-D0*-g(w)}), is the unique root of
\begin{equation}
  K \left[ 1 + \cos\left({\pi/K}\right)\right]^{1/4} =
 \frac{\left({S/D}\right)^{1/4}}{2^{3/4} \sqrt{\pi \epsilon \alpha}} \left[
   g(U_0) \right]^{1/4} \,. \label{stab:Kc+}
\end{equation}


\begin{figure}[htbp]
\centering
\includegraphics[width=9cm,height=5.0cm]{figs/comp_gplot}
\caption{\label{fig:gU0-vs-U0}Competition instability threshold
  nonlinearity $g(U_{0})$, as defined in (\ref{eq:pol-D0*-g(w)}), versus
  total police deployment $U_{0}$ for patrol focus parameters
  $q=2,3,4,5$. Other model parameters are $S=6$, $\gamma=2$, $\alpha=1$, so
  that $U_{0,\max}=6$ as shown in the right-most tick of the figure.
  The competition instability threshold ${\mathcal D}_{0,c}$ is
  simply a positive scaling of $g(U_{0})$ according to
  (\ref{eq:pol-D0*-g(w)}).}
\end{figure}

\setcounter{equation}{0}
\setcounter{section}{4}
\section{Explicitly Solvable Case $q=3$: Asynchronous Hotspot Oscillations}
\label{sec:stab_q3}

For each $j=1,\ldots,K-1$, we now analyze the quadratic equation
(\ref{stab:q3_quad}) in the eigenvalue parameter $\lambda$
characterizing the discrete spectrum of the NLEP
(\ref{stab:nlep_final_1}) for the special case where $q=3$. In terms
of the coefficients of the quadratic (\ref{stab:q3_quad_1}), for each
$j=1,\ldots,K$ the eigenvalues $\lambda_1$ and $\lambda_2$ satisfy
\begin{equation}\label{q3:lam12}
   \lambda_1 \lambda_2 = \frac{c_0}{c_2} \,, \qquad
  \lambda_1 + \lambda_2 = - \frac{c_1}{c_2} \,,
\end{equation}
where $c_0$, $c_1$, and $c_2>0$, are given in (\ref{stab:q3_quad_2}). For
the $j$-th mode, we conclude that $\mbox{Re}(\lambda)<0$ when $c_0>0$
and $c_1>0$. We have instability of the $j$-th mode if either
$c_0<0$, or if $c_0>0$ and $c_1<0$. We have purely complex eigenvalues,
corresponding to a Hopf bifurcation point, when $c_0>0$ and $c_1=0$.

We first determine the signs of $c_0$ and $c_1$ in terms of $D_j$ and
$\hat{\tau}_{u}$. From (\ref{stab:q3_quad_2}) we observe that $c_0=0$
when the zero-eigenvalue crossing condition (\ref{stab:zero_1}) holds,
which yields (\ref{stab:zero_2}) for $D_j$, which we relabel as
\begin{equation}\label{q3:dup}
     D_j = \dustar \equiv \frac{\omega^3}{4\pi^2 K^3 \alpha^2}
  \left( 1 + \frac{3U_0}{\omega}\right) \,.
\end{equation}
Next, we set $c_1=0$ in (\ref{stab:q3_quad_2}) to get, using
(\ref{stab:newchi}) for $\chi_{0,j}^{-1}$, that
\begin{equation}\label{q3:c1zero_old}
  \frac{\hat{\tau}_u}{D_j \kappa_3} = \frac{2 \chi_{0,j}^{-1}}
  {3 \left( 2 \chi_{0,j}^{-1} -3 \right)} = \frac{1}{3}
  \frac{ \left(D_j + {\alpha/\kappa_3}\right)}{\left(D_{j}- 
  {\alpha/(2\kappa_3)}\right)} \,.
\end{equation}
The denominator of this expression motivates introducing $\dlstar$,
defined by
\begin{equation}\label{q3:dlow}
    \dlstar \equiv \frac{\alpha}{2\kappa_3} =
 \frac{\omega^3}{4\pi^2 K^3 \alpha^2} \,,
\end{equation}
where we have used the expression (\ref{stab:kappa_omega}) for
$\kappa_3$. Upon using (\ref{q3:dlow}) in (\ref{q3:c1zero_old}), we
obtain that $c_1=0$ when $\hat{\tau}_u$ satisfies
\bsub \label{q3:c1zero}
\begin{equation}
   \hat{\tau}_u = \hat{\tau}_{uH,j}\equiv {\mathcal H}\left({D_j/\dlstar}\right)
  \,, \qquad j=1,\ldots,K-1 \,,
\end{equation}
where the function ${\mathcal H}(\beta)$ is defined by
\begin{equation}\label{q3:h}
   {\mathcal H}(\beta) \equiv \frac{\alpha \beta}{2} \left(
  \frac{1}{3} + \frac{1}{\beta -1} \right) \,.
\end{equation}
\esub
Notice that $\hat{\tau}_{uH,j}>0$ only when $D_j>\dlstar$.  Some simple
algebra then shows that we can write $c_1$ in (\ref{stab:q3_quad_2}) as
\begin{equation}\label{q3:c1_new}
  c_1 = \frac{1}{\alpha} \left( \frac{\dlstar}{D_j}-1\right)
  \left(\hat{\tau}_u - \hat{\tau}_{uH,j}\right) \,.
\end{equation}
For the $j$-th mode, we have $c_1<0$ if $D_j>\dlstar$ and
$\hat{\tau}_u>\hat{\tau}_{uH,j}$, while $c_1>0$ if either
$D_j<\dlstar$, or $D_j>\dlstar$ and $\hat{\tau}_u<\hat{\tau}_{uH,j}$.
With these signs for $c_0$ and $c_1$, we summarize our stability
result for the $j$-th mode so far as follows:
\begin{itemize}
  \item For $D_j>\dustar$ ($c_0<0$), we have $\mbox{Re}(\lambda)>0$, and
        instability is due to a positive real eigenvalue.
  \item For $D_j<\dlstar$ ($c_0>0$ and $c_1>0$), we have stability
        $\mbox{Re}(\lambda)<0$.
  \item On the range $\dlstar<D_j<\dustar$ $(c_0>0)$, we have
    instability if $\hat{\tau}_u>\hat{\tau}_{uH,j}$ $(c_1<0)$ and
    stability if $\hat{\tau}_u<\hat{\tau}_{uH,j}$ $(c_1>0)$. On this
    range of $D_j$, the Hopf bifurcation threshold,
    $\hat{\tau}_{uH,j}>0$, is given in (\ref{q3:c1zero}). 
\end{itemize}

Next, we must reformulate this result in terms of ${\mathcal D}_0$
rather than $D_j$, by using $D_j={\mathcal D}_0 \left({2K/S}\right)
\left(1-\cos\left({\pi j/K}\right)\right)$. The interval 
$\dlstar<D_1<\dustar$, where a Hopf bifurcation value of $\hat{\tau}_u$ 
exists, becomes
\begin{equation}\label{q3:interval_old}
 \frac{S\dlstar}{2K\left(1-\cos\left(\pi j/K\right)\right)} \leq {\mathcal D}_0
 \leq \frac{S\dustar}{2K\left(1-\cos\left(\pi j/K\right)\right)} \,,
\end{equation}
where ${\dustar/\dlstar}=1+{3U_0/\omega}$ from (\ref{q3:dup}) and
(\ref{q3:dlow}). It is convenient to write (\ref{q3:interval_old}) in
terms of the competition stability threshold ${\mathcal D}_{0,c}$
defined by setting $q=3$ in (\ref{stab:zero_d0min}). In this way, the
interval in (\ref{q3:interval_old}) becomes 
\bsub \label{q3:interval}
\begin{equation}
   \dzjm < {\mathcal D}_0 <\dzjp \,; \qquad
   \dzjp \equiv {\mathcal D}_{0,c} \left( \frac{1+\cos\left({\pi/K}\right)}
  {1-\cos\left({\pi j /K}\right)}\right) \,, \qquad
   \dzjm \equiv \frac{{\mathcal D}_{0,c}}{\left(1+{3U_0/\omega}\right)}
  \left( \frac{1+\cos\left({\pi/K}\right)}
  {1-\cos\left({\pi j /K}\right)}\right) \,, \label{q3:interval_1}
\end{equation}
where ${\mathcal D}_{0,c}$ is given by
\begin{equation}
{\mathcal D}_{0,c}\equiv \frac{\omega^3 S}{8
     \pi^2 \alpha^2 K^4 \left(1 + \cos\left({\pi /K}\right)\right)}
   \left( 1 + \frac{3 U_0}{\omega} \right) \,. \label{q3:interval_2}
\end{equation}
\esub
We observe that when $j=K-1$, we have $D^{+}_{0,K-1}={\mathcal D}_{0,c}$.
Now since ${D_j/\dlstar}={{\mathcal D}_0/\dzjm}$, the Hopf bifurcation
threshold in (\ref{q3:c1zero}) becomes
\begin{equation}\label{q3:hopf}
   \hat{\tau}_{uH,j} = {\mathcal H}\left({{\mathcal D}_0/\dzjm}\right)
   \,, \qquad \mbox{on} \quad  \dzjm < {\mathcal D}_0 <\dzjp \,.
\end{equation}
From this expression, we readily derive the following limiting
behavior for $\hat{\tau}_{uH,j}$ at the two ends of the interval:
\begin{equation}\label{q3:tau_lim}
 \hat{\tau}_{uH,j} \sim \frac{\alpha}{2} \left[ \frac{{\mathcal D}_0}
 {D^{-}_{0,j}}-1 \right]^{-1} \,, \quad \mbox{as} \quad
{\mathcal D}_0\to \dzjm\,; \qquad
 \hat{\tau}_{uH_j}\sim \frac{\omega \alpha}{6 U_0} \left( \frac{U_0}{\omega}
  + 1 \right) \left( \frac{3U_0}{\omega} + 1\right) \,, \quad \mbox{as}
  \quad {\mathcal D}_0\to \dzjp \,.
\end{equation}

For each fixed $j=1,\ldots,K-1$, we summarize the behavior of the roots
of the quadratic (\ref{stab:q3_quad}), corresponding to the discrete
eigenvalues of the NLEP (\ref{stab:nlep_final_1}) as follows:

\begin{prop}\label{q3:roots_quad} For each fixed $j=1,\ldots,K-1$,
let $\lambda_{+}$ and $\lambda_{-}$, with
$\mbox{Re}(\lambda_{+})\geq\mbox{Re}(\lambda_{-})$, denote the two
solutions of the quadratic equation (\ref{stab:q3_quad}). Then, their
location in the complex plane depends on ${\mathcal D}_0$ and
$\hat{\tau}_u$ as follows:
\begin{itemize}
  \item For ${\mathcal D}_0>\dzjp$, we have $\lambda_+>0$ and $\lambda_-<0$
     for all $\hat{\tau}_u\geq 0 \,.$
  \item For ${\mathcal D}_0<\dzjm$, we have $\mbox{Re}(\lambda_{\pm})<0$
     for all $\hat{\tau}_u\geq 0 \,.$
  \item For $\dzjm<{\mathcal D}_0<\dzjp$ we have $\mbox{Re}(\lambda_{\pm})>0$
   when $\hat{\tau}_u>\hat{\tau}_{uH,j}$ and $\mbox{Re}(\lambda_{\pm})<0$ when
   $0\leq \hat{\tau}_{uH,j}<\hat{\tau}_u$.
\end{itemize}
Here $\dzjm$ and $\dzjp$ are defined in (\ref{q3:interval_1}). The
Hopf bifurcation threshold $\hat{\tau}_{uH,j}$, which is defined on
the interval $\dzjm < {\mathcal D}_0 <\dzjp$, is given in
(\ref{q3:hopf}).
\end{prop}

Since ${\dzjp/\dzjm}=1+{3U_0/\omega}$, we observe that the width of
the interval $\dzjm<{\mathcal D}_0<\dzjp$ where an asynchronous
oscillatory instability in the hotspot amplitudes occurs is nonzero only as
a result of the simple coupling term $-U$ in our three-component RD
system (\ref{eq:crs-main}). In the absence of police, this interval
disappears and the basic crime model does not support oscillatory
instabilities in this parameter regime.

Next, we examine the monotonicity properties of the universal function 
${\mathcal H}(\beta)$ characterizing Hopf bifurcations, as defined
in (\ref{q3:h}), on the interval $1<\beta<{\dzjp/\dzjm}=1+{3U_0/\omega}$.
We calculate ${\mathcal H}^{\prime}(\beta)$ to get
\begin{equation*}
   {\mathcal H}^{\prime}(\beta) = \frac{\alpha}{6 (\beta-1)^2} \left[
  (\beta-1)^2 -3 \right] \,,
\end{equation*}
so that ${\mathcal H}^{\prime}(\beta)<0$ if $1<\beta<1+\sqrt{3}$ and
${\mathcal H}^{\prime}(\beta)>0$ if $\beta>1+\sqrt{3}$. We conclude
that ${\mathcal H}^{\prime}(\beta)<0$ on $1<\beta<1+{3U_0/\omega}$ only
when $\omega>\sqrt{3}U_0$. Since $\omega=S(\gamma-\alpha)-U_0$ we conclude
that
\begin{equation}\label{q3:hmonot}
   {\mathcal H}^{\prime}(\beta)<0 \, \quad \mbox{on} \quad 1<\beta<{3U_0/\omega}
  \,, \quad \mbox{iff} \quad U_0 < \frac{S(\gamma-\alpha)}{1+\sqrt{3}} \,.
\end{equation}
If $\frac{S(\gamma-\alpha)}{1+\sqrt{3}}<U_0<U_{0,\max}$, then
${\mathcal H}(\beta)$ increases on $1+\sqrt{3}<\beta<1+{3U_0/\omega}$.

Next, we rewrite the coefficients $c_0$, $c_1$, and $c_2$, in the
quadratic (\ref{stab:q3_quad}) so as to readily calculate the Hopf
bifurcation eigenvalue $\lambda=i\lambda_{IH}$. After some algebra we
obtain that
\begin{equation}\label{q3:new_c0c1c2}
  c_0 = -\frac{1}{2}\left(1+\frac{3U_0}{\omega}\right)
   \left( \frac{{\mathcal D}_0}{\dzjp}-1\right) \,, \qquad
  c_1 = \frac{\hat{\tau}_{uH,j}}{\alpha} \left( \frac{\dzjm}{{\mathcal D}_0}
  -1 \right)\left( \frac{\hat{\tau}_u}{\hat{\tau}_{uH,j}} -1 \right) \,,
  \qquad c_2 = \frac{\hat{\tau}_{uH,j}}{3\alpha}
  \left(\frac{2\dzjm}{{\mathcal D}_0} + 1 \right) \,,
\end{equation}
where $\hat{\tau}_{uH,j}$ is defined in (\ref{q3:hopf}). The Hopf bifurcation
eigenvalue $\lambda=i\lambda_{IH}$ with $\lambda_{IH}>0$ is 
$\lambda_{IH}=\sqrt{c_0/c_2}$, which yields
\begin{equation}\label{q3:dlami}
   \lambda_{IH}=\frac{3}{\left(2 + {{\mathcal D}_0/\dzjm}\right)} 
  \sqrt{ \left( 1 + \frac{3U_0}{\omega}\right) 
    \left(1 - \frac{{\mathcal D}_0}{\dzjp}\right)
\left(\frac{{\mathcal D}_0}{\dzjm} -1 \right)}\,, \quad \mbox{on} \quad
   \dzjm < {\mathcal D}_0 < \dzjp \,.
\end{equation}
This shows that $\lambda_{IH}$ vanishes at both endpoints.  We use the
asymptotic behaviors ${{\mathcal D}_0/\dzjp}\to
\left(1+{3U_0/\omega}\right)^{-1}$ as ${\mathcal D}_0\to \dzjm$ and
${{\mathcal D}_0/\dzjm}\to \left(1+{3U_0/\omega}\right)$ as ${\mathcal
  D}_0\to \dzjp$, so that from (\ref{q3:dlami}) we obtain the limiting
asymptotic behavior
\begin{equation}\label{q3:lim_val}
   \lambda_{IH}\sim \frac{1}{\left(1 + {U_0/\omega}\right)}
  \sqrt{ \frac{3U_0}{\omega} \left( 1 + \frac{3U_0}{\omega}\right) 
    \left(1 - \frac{{\mathcal D}_0}{\dzjp}\right)} \quad \mbox{as}
 \quad {\mathcal D}_0\to \dzjp\,; \qquad
  \lambda_{IH}\sim \sqrt{ \frac{3U_0}{\omega} 
  \left( \frac{{\mathcal D}_0}{\dzjm}-1\right)} \quad \mbox{as} \quad
  {\mathcal D}_0\to \dzjm \,.
\end{equation}

\begin{figure}[htbp]
\centering
\includegraphics[width=9cm,height=5.0cm]{figs/hopf_2_u02.eps}
\includegraphics[width=9cm,height=5.0cm]{figs/hopf_2_u04.eps}
\caption{\label{fig:hopf_tau_2} Plot of the Hopf bifurcation threshold
  $\hat{\tau}_{uH_1}$ versus ${\mathcal D}_0$ on the range ${{\mathcal
      D}_{0,c}/(1+{3U_0/\omega})}<{\mathcal D}_0<{\mathcal D}_{0c}$
  for $K=2$, $q=3$, $S=6$, $\gamma=2$, $\alpha=1$, and with $U_0=2$
  (left panel) and $U_0=4$ (right panel). The shaded region is where
  the steady-state two-hotspot pattern is linearly stable. The thin
  vertical line in each figure is the lower boundary ${{\mathcal
      D}_{0,c}/(1+{3U_0/\omega})}$, while the right edge of the shaded
  region is the competition stability threshold.  The Hopf bifurcation
  curve in the right panel is not monotonic since
  $U_0>{S(\gamma-\alpha)/(1+\sqrt{3})}$ when $U_0=4$.}
\end{figure}

\begin{figure}[htbp]
\centering
\includegraphics[width=9cm,height=5.0cm]{figs/ss_2_pol_u04.eps}
\includegraphics[width=9cm,height=5.0cm]{figs/hopf_2_pol_u04.eps}
\caption{\label{fig:hopf_pol_2} Left panel: the steady-state
  two-hotspot solution for $S=6$, $\gamma=2$, $\alpha=1$, $U_0=4$,
  ${\mathcal D}_0=0.15$, $\epsilon=0.035$, and $q=3$. Right panel:
  Plot of the Hopf bifurcation threshold for the scaled police
  diffusivity $\epsilon^{2}D_p\equiv {{\mathcal
      D}_0/\hat{\tau}_{uH_1}}$ versus ${\mathcal D}_0$ on the range
  ${{\mathcal D}_{0,c}/(1+{3U_0/\omega})} <{\mathcal D}_0<{\mathcal
    D}_{0c}$. The thin vertical line is the competition stability
  threshold ${\mathcal D}_{0,c}$ given in Proposition
  \ref{q3:main_twospots}. The shaded region is where the steady-state
  two-hotspot pattern is linearly stable. For ${\mathcal
    D}_0>{\mathcal D}_{0c}$ the hotspot solution is unstable due to a
  competition instability, whereas in the small unshaded region for
  ${\mathcal D}_0<{\mathcal D}_{0c}$, the hotspot steady-state is
  unstable to an asynchronous oscillatory instability of the hotspot
  amplitudes. The full PDE simulations in
  Fig.~\ref{fig:valid_2_q3_u04} and in Fig.~\ref{fig:valid_2_q3_u04_b}
  are done at the marked points.}
\end{figure}

\noindent For the special case $K=2$, we now state our main result stability
result related to Hopf bifurcations.

\begin{prop}\label{q3:main_twospots} For $\epsilon\to 0$, 
$q=3$, $0<U_0<U_{0,\textrm{max}}$, $\hat{\tau}_u\ll {\mathcal
    O}(\epsilon^{-2})$, and ${\mathcal D}_0=\epsilon^2 D = {\mathcal
    O}(1)$, the linear stability properties of a two-hotspot
  steady-state solution of (\ref{eq:pol-main}) are as follows:
\begin{itemize}
  \item For ${\mathcal D}_0>D^{+}_{0,1}\equiv {\mathcal D}_{0,c}$, the
    NLEP (\ref{stab:nlep_final_1}) has a positive real eigenvalue for
    all $\hat{\tau}_u\geq 0$ and so the two-hotspot steady-state is
    unstable. Here ${\mathcal D}_{0,c}={S\omega^3
      \left(1+{3U_0/\omega}\right)/[128\pi^2\alpha^2]}$ is the
    competition stability threshold with $\omega\equiv
    S(\gamma-\alpha)-U_0$.
 \item On the range $D^{-}_{0,1} \equiv {{\mathcal
      D}_{0,c}/\left(1+{3U_0/\omega}\right)} < {\mathcal D}_0
    <{\mathcal D}_{0,c}$, there is a Hopf bifurcation corresponding to an
   asynchronous oscillatory instability of the hotspot
    amplitudes when 
\begin{equation}\label{q3:2osc}
 \hat{\tau}_{u}\equiv \hat{\tau}_{uH,1}= {\mathcal H}\left({{\mathcal
     D}_0/D^{-}_{0,1}}\right) \,, \quad \mbox{on} \quad
 D^{-}_{0,1}<{\mathcal D}_0 <{\mathcal D}_{0,c} \,,
\end{equation}
where ${\mathcal H}(\beta)$ is defined in (\ref{q3:h}).  When
$\hat{\tau}_u>\hat{\tau}_{uH,1}$, the two-hotspot steady-state is
unstable, while if $\hat{\tau}_u<\hat{\tau}_{uH,1}$ the two-hotspot
pattern is linearly stable.
\item On the range $0<{\mathcal D}_0 < D^{-}_{0,1} \equiv {{\mathcal
    D}_{0,c}/\left(1+{3U_0/\omega}\right)}$, the two-hotspot
  steady-state is linearly stable for all $\hat{\tau}_u\geq 0$.
\end{itemize}
\end{prop}

\noindent In terms of a scaled police diffusivity defined by 
$\epsilon^2 D_p\equiv {{\mathcal D}_0/\hat{\tau}_u}$, Proposition
\ref{q3:main_twospots} implies the following:

\begin{cor}\label{q3:main_twospots_pol} Under the conditions of
Proposition \ref{q3:main_twospots} we have the following:
\begin{itemize}
  \item For ${\mathcal D}_0>{\mathcal D}_{0,c} \equiv {S\omega^3
    \left(1+{3U_0/\omega}\right)/[128\pi^2\alpha^2]}$, the two-hotspot
    steady-state is unstable for all scaled police diffusivities $\epsilon^2
    D_p>0$.
 \item On the range ${{\mathcal D}_{0,c}/\left(1+{3U_0/\omega}\right)}
   < {\mathcal D}_0 <{\mathcal D}_{0,c}$, the two hotspot
   steady-state is unstable to an asynchronous oscillatory instability
   of the hotspot amplitudes if $\epsilon^2 D_p<{{\mathcal
       D}_{0}/\hat{\tau}_{uH,1}}$, while the steady-state is linearly
   stable when $\epsilon^2 D_p>{{\mathcal
       D}_{0}/\hat{\tau}_{uH,1}}$. Here $\hat{\tau}_{uH,1}$ is the
   Hopf bifurcation threshold in (\ref{q3:2osc}).
\item On the range $0<{\mathcal D}_0 < {{\mathcal
    D}_{0,c}/\left(1+{3U_0/\omega}\right)}$, the two-hotspot
  steady-state is linearly stable for all $\epsilon^2 D_p\geq 0$.
\end{itemize}
\end{cor}

We now illustrate our main stability results for $K=2$, $S=6$,
$\gamma=2$, and $\alpha=1$. In Fig.~\ref{fig:hopf_tau_2} we plot the
region of linear stability in the $\hat{\tau}_u$ versus ${\mathcal
  D}_0$ parameter plane for $U_0=2$ (left panel) and $U_0=4$ (right
panel). For $U_0=4$, we have $\omega<\sqrt{3}U_0$, and so the Hopf
bifurcation threshold $\hat{\tau}_{uH,1}$ is not monotone in
${\mathcal D}_0$, as seen in the right panel of
Fig.~\ref{fig:hopf_tau_2}. From this figure, we observe that as $U_0$
increases the region where the two-hotspot steady-state is linearly
stable is smaller, as expected. With regards to the scaled police
diffusivity $\epsilon^2 D_p\equiv {{\mathcal D}_0/\hat{\tau}_u}$, in
the right panels of Fig.~\ref{fig:hopf_pol_intro} and
Fig.~\ref{fig:hopf_pol_2} we plot the corresponding region of linear
stability in the $\epsilon^2 D_p$ versus ${\mathcal D}_{0}$ plane for
$U_0=2$ and $U_0=4$, respectively. For $U_0=2$ and $U_0=4$, the
corresponding steady-state two-hotspot solution is shown in the left
panels of Fig.~\ref{fig:hopf_pol_intro} and Fig.~\ref{fig:hopf_pol_2}.
For $U_0=2$, the predicted linear stability results were validated in
Fig.~\ref{fig:valid_2spot_q3} by performing full numerical
solutions of the PDE system (\ref{eq:pol-main}). A similar validation
for the linear stability phase diagram in the right panel of
Fig.~\ref{fig:hopf_pol_2} for $U_0=4$ is given by the full PDE
simulations reported in Fig.~\ref{fig:valid_2_q3_u04} and
Fig.~\ref{fig:valid_2_q3_u04_b} below in \S \ref{sec:numerics_q3}.


\subsection{The Stability Phase Diagram for $q=3$: $K\geq 3$ Hotspots}
\label{sect:q3_phase}

Next, we determine the parameter range of ${\mathcal D}_0$ and
$\hat{\tau}_u$ for which a $K$-hotspot steady-state solution, with
$K\geq 3$, is linearly stable. To do so, we need to guarantee that
$\mbox{Re}(\lambda)<0$ for each of the quadratics in
(\ref{stab:q3_quad}), i.e.~for each $j=1,\ldots,K-1$. In this way, we will
ensure that any discrete eigenvalue of the NLEP
(\ref{stab:nlep_final_1}) satisfies $\mbox{Re}(\lambda)\leq 0$.

By using (\ref{q3:interval_1}), we readily obtain the ordering principle that
\begin{equation}\label{q3:d_order}
    D^{\pm}_{0,j+1} < D^{\pm}_{0,j} \,, \quad \mbox{and} \quad
    D^{-}_{0,j} < D^{+}_{0,j} \,, \qquad \mbox{for} \quad j=1,\ldots,K-2 \,.
\end{equation}
We conclude that 
\begin{equation}\label{q3:d_order_1}
    D^{+}_{0,K-1} = \min_{j=1,\ldots,K-1} \lbrace{ D^{+}_{0,j} \rbrace} \,,
    \qquad 
    D^{-}_{0,K-1} = \min_{j=1,\ldots,K-1} \lbrace{ D^{-}_{0,j} \rbrace} \,.
\end{equation}
From Proposition \ref{q3:roots_quad}, we conclude for each of the
quadratics (\ref{stab:q3_quad}), i.e.~for each $j=1,\ldots,K-1$, that
$\mbox{Re}(\lambda)<0$ for any $\hat{\tau}_u\geq 0$ when ${\mathcal
  D}_0<D^{-}_{0,K-1}$. Therefore, a $K$-hotspot steady-state pattern
is linearly stable for all $\hat{\tau}_u\geq 0$ on the range
$0<{\mathcal D}_0<D^{-}_{0,K-1}$. For the range ${\mathcal
  D}_0>D^{+}_{0,K-1}$, we conclude from Proposition
\ref{q3:roots_quad} that the $K-1$ mode must be unstable due to a
positive real eigenvalue for any $\hat{\tau}_u\geq 0$.  Therefore, for
${\mathcal D}_0>D^{+}_{0,K-1}$, a $K$-hotspot steady-state solution is
unstable for all $\hat{\tau}_u\geq 0$.  Additional unstable
eigenvalues due to Hopf bifurcations associated with the remaining modes
$j=1,\ldots,K-2$ are possible depending on the value of
$\hat{\tau}_u$.

To complete the stability phase diagram in the $\hat{\tau}_u$ versus 
${\mathcal D}_0$ parameter plane, we need only focus on the interval
$D^{-}_{0,K-1}<{\mathcal D}_0<D^{+}_{0,K-1}$. On this interval, the 
sign-alternating $K-1$ mode undergoes a Hopf bifurcation at 
$\hat{\tau}_u=\hat{\tau}_{uH,K-1}$, as given from (\ref{q3:hopf}) by
\begin{equation}\label{q3:hopf_K-1}
     \hat{\tau}_{uH,K-1} \equiv {\mathcal H}\left(\beta\right) \,,
     \qquad \mbox{on} \quad 1\leq \beta \leq
     \frac{D^{+}_{0,K-1}}{D^{-}_{0,K-1}} = 1+ \frac{3U_0}{\omega} \,,
     \qquad \mbox{where} \quad \beta \equiv \frac{{\mathcal
         D}_0}{D^{-}_{0,K-1}} \,.
\end{equation}
Here ${\mathcal H}(\beta)$ is defined in (\ref{q3:h}).  When
$\hat{\tau}_u>\hat{\tau}_{uH,K-1}$ the $K-1$ mode is unstable, whereas
if $\hat{\tau}_u<\hat{\tau}_{uH,K-1}$ the $K-1$ mode is linearly
stable.

We now seek to determine conditions for which the Hopf bifurcation
threshold for the $K-1$ mode is smaller than any of the other
$K-2$ possible Hopf bifurcation values $\hat{\tau}_{uH,j}$ for
$j=1,\ldots,K-2$ when restricted to the interval
$D^{-}_{0,K-1}<{\mathcal D}_0<D^{+}_{0,K-1}$. From (\ref{q3:hopf}),
these other Hopf bifurcation thresholds, for $j=1,\ldots,K-2$, can be
written in terms of $\beta$, as defined in (\ref{q3:hopf_K-1}), by
\bsub \label{q3:hopf_jall}
\begin{equation}\label{q3:hopf_j}
    \hat{\tau}_{uH,j} = {\mathcal H}(\xi_j\beta) \,, \qquad \mbox{on}\quad
  \frac{1}{\xi_j} \leq \beta \leq \frac{1}{\xi_j}\left(1 + \frac{3U_0}{\omega}
  \right) \,,
\end{equation}
where, from (\ref{q3:interval}), we define
\begin{equation}\label{q3:hopf_jxi}
 \xi_j \equiv \frac{D^{-}_{0,K-1}}{D^{-}_{0,j}} = \frac{1- \cos\left({\pi j/K}
  \right)}{1 + \cos\left({\pi /K}\right)} \,, \qquad j=1,\ldots,K-2\,.
\end{equation}
We observe from (\ref{q3:hopf_jxi}) that the following ordering
principle holds:
\begin{equation}\label{q3:xi_ord}
   \xi_{j}<\xi_{j+1} <1 \,, \qquad
  j=1,\ldots,K-3 \,, \qquad \xi_{K-2} = \max_{j=1,\ldots,K-2}
  \lbrace{\xi_j\rbrace}\,.
\end{equation}
\esub

Comparing the intervals in (\ref{q3:hopf_j}) and (\ref{q3:hopf_K-1}),
we want to determine a specific parameter range of the total police
deployment $U_0$ for which, for any $j=1,\ldots,K-2$, we have that
$\hat{\tau}_{uH,K-1}<\hat{\tau}_{uH,j}$ on the overlap domain
$\xi_j^{-1}\leq \beta \leq 1+{3U_0/\omega}$. If the overlap domain is
the null-set for the $j$-th mode, i.e.~if
$\xi_j<{1/(1+{3U_0/\omega})}$, then we can simply ignore the $j$-th
mode on $D^{-}_{0,K-1}<{\mathcal D}_0<D^{+}_{0,K-1}$. As such, we need
only consider values of $j$ (if any) for which $\xi_j^{-1} <
1+{3U_0/\omega}$, so that an overlap domain exists. Since ${\mathcal
  H}(\beta)$ is monotone decreasing on $1<\beta<1+\sqrt{3}$, we
readily obtain that ${\mathcal H}(\beta)-{\mathcal
  H}(\xi_j\beta)\equiv \int_{\xi_j\beta}^{\beta} {\mathcal
  H}^{\prime}(y) \, dy <0$ on the interval
$\xi_j^{-1}<\beta<1+\sqrt{3}$. In this way, we conclude that
\begin{equation}\label{q3:hineq}
  {\mathcal H}(\xi_j\beta) < {\mathcal H}(\beta) \,, \qquad \mbox{on}
  \quad \xi_j^{-1}\leq \beta \leq 1 + \frac{3U_0}{\omega} \,, \qquad
  \mbox{when} \quad \omega > \sqrt{3} U_0 \,.
\end{equation}
Therefore, on the range for which ${\mathcal H}(\beta)$ is
monotonically decreasing, it follows that the Hopf bifurcation
threshold of $\hat{\tau}_u$ for any mode $j=1,\ldots,K-2$ cannot be
smaller than that for the $K-1$ mode.  Although the monotonicity of
${\mathcal H}(\beta)$ on $\xi_j^{-1}\leq \beta\leq 1+{3U_0/\omega}$
for $\omega>\sqrt{3}U_0$ provides a sufficient condition for the
ordering principle $\hat{\tau}_{uH,K-1}<\hat{\tau}_{uH,j}$ for
$j=1,\ldots,K-2$ to hold, we now show explicitly that the monotonicity
of ${\mathcal H}(\beta)$ is not strictly necessary.

We now determine a precise condition that ensures that
$\hat{\tau}_{uH,K-1}<\hat{\tau}_{uH,K-2}$ on an assumed overlap domain
$\xi_{K-2}^{-1}\leq \beta \leq 1+{3U_0/\omega}$. Owing to the ordering
principle $\xi_{j}<\xi_{j+1}$ for $j=1,\ldots,K-3$ from
(\ref{q3:xi_ord}), the first Hopf threshold to potentially decrease
below that of the $K-1$ mode must be the $K-2$ mode, and so we focus
only on a comparison with the $K-2$ mode. From (\ref{q3:hopf_K-1}) and
(\ref{q3:hopf_j}), and by using the explicit expression for ${\mathcal
  H}(\beta)$ in (\ref{q3:h}), we calculate after some algebra that
${\mathcal H}(\xi_{K-2}\beta)\geq {\mathcal H}(\beta)$ on
$\xi_{K-2}^{-1}\leq \beta\leq 1+{3U_0/\omega}$, if and only if
\begin{equation}\label{q3:swit_1}
   {\mathcal K}(\beta) \equiv \left(\xi_{K-2}\beta -1\right)(\beta-1)<3
\,, \qquad \mbox{on} \quad 1<\xi_{K-2}^{-1}\leq \beta\leq 1+{3U_0/\omega}\,.
\end{equation}
Since ${\mathcal K}^{\prime}(\beta)>0$ on this interval, this inequality
holds if and only if $1+{3U_0/\omega}<\beta_{\max}$, where
${\mathcal  K}(\beta_{\max})=3$. By setting ${\mathcal K}(\beta)=3$, and
solving the quadratic for $\beta=\beta_{\max}$,
we obtain that (\ref{q3:swit_1}) holds if and only if
\begin{equation}\label{q3:key_swit}
     \frac{\sqrt{3}U_0}{\omega} < {\mathcal Z}(\xi_{K-2}) \,, \qquad
     \mbox{where} \quad {\mathcal Z}(\xi_{K-2}) \equiv
     \frac{1}{\sqrt{3}} \left( -\frac{1}{2} + \frac{1}{2\xi_{K-2}}
       \left[1 + \sqrt{\xi^2_{K-2}+10 \xi_{K-2}+1} \right] \right) \,.
\end{equation}
Here $\omega=S(\gamma-\alpha)-U_0$ and $\xi_{K-2}$ can be found from
(\ref{q3:hopf_jxi}). On $0<\xi<1$, we have that ${\mathcal Z}(\xi)$ 
satisfies
\begin{equation}\label{q:zprop}
   {\mathcal Z}(\xi)\to +\infty \quad \mbox{as} \quad \xi\to 0^{+} \,,
\qquad {\mathcal Z}(1)=1\,, \qquad {\mathcal Z}^{\prime}(\xi)<0 \,, \quad
  \mbox{on} \quad 0<\xi<1 \,.
\end{equation}
It follows that ${\mathcal Z}(\xi)>1$ on $0<\xi<1$. The key inequality
(\ref{q3:key_swit}) implies that $\omega>{\sqrt{3}U_0/{\mathcal
    Z}(\xi_{K-2})}$, which yields a larger range of $\omega$ than the
range $\omega>\sqrt{3}U_0$ where ${\mathcal H}(\beta)$ is monotonic.
This inequality (\ref{q3:key_swit}) can also be used to give a precise
upper bound on $U_0$ for which the $K-1$ mode determines the Hopf
bifurcation threshold for $\hat{\tau}_u$ on the entire range
$D^{-}_{0,K-1}<{\mathcal D}_0<D^{+}_{0,K-1}$. In this way, for $K\geq 3$,
we summarize our main stability result for a $K$-hotspot steady-state 
solution as follows:

\begin{prop}\label{q3:main_Kspots} For $\epsilon\to 0$, 
$q=3$, $K\geq 3$, $0<U_0<U_{0,\textrm{max}}$, $\hat{\tau}_u\ll
  {\mathcal O}(\epsilon^{-2})$, and ${\mathcal D}_0=\epsilon^2 D =
  {\mathcal O}(1)$, the linear stability properties of a $K$-hotspot
  steady-state solution of (\ref{eq:pol-main}) are as follows:
\begin{itemize}
  \item For ${\mathcal D}_0>D^{+}_{0,K-1} \equiv {\mathcal D}_{0,c}$, the
    NLEP (\ref{stab:nlep_final_1}) has at least one positive real eigenvalue for
    all $\hat{\tau}_u\geq 0$. Additional unstable eigenvalues as a result
    of Hopf bifurcations associated with the other modes $j=1,\ldots,K-2$ 
    are possible  depending on the value of $\hat{\tau}_u$. Here 
    ${\mathcal D}_{0,c}$ is the competition stability threshold given in
    (\ref{stab:zero_d0min}) with $q=3$.
 \item On the range $D^{-}_{0,K-1} \equiv 
  {{\mathcal D}_{0,c}/\left(1+{3U_0/\omega}\right)} < {\mathcal D}_0
    <{\mathcal D}_{0,c}$, and when $U_0$ satisfies
\begin{equation}\label{q3:u0mon}
     U_0 < U_{0,\textrm{swit}} \equiv \left( \frac{ {\mathcal Z}(\xi_{K-2})}
  { \sqrt{3} + {\mathcal Z}(\xi_{K-2})} \right) \, S(\gamma-\alpha)
  \,, \qquad \mbox{where} \quad \xi_{K-2} \equiv 
  \frac{1- \cos\left({\pi (K-2)/K}\right)}{1 + \cos\left({\pi /K}\right)} \,, 
\end{equation}
and where ${\mathcal Z}(\xi)$ is defined in (\ref{q3:key_swit}), the
$K-1$ sign-alternating mode sets the stability threshold on the entire
range.  For $\hat{\tau}_u>\hat{\tau}_{uH,K-1}$, the $K$-hotspot
pattern is unstable, while if $\hat{\tau}_u<\hat{\tau}_{uH,K-1}$ the
$K$-hotspot pattern is linearly stable. With ${\mathcal H}(\beta)$, as
defined in (\ref{q3:h}), the minimal Hopf bifurcation value of
$\hat{\tau}_u$ is 
\begin{equation}\label{q3:Kosc}
 \hat{\tau}_{uH,K-1}\equiv {\mathcal H}\left({{\mathcal
     D}_0/D^{-}_{0,K-1}}\right) \,, \quad \mbox{on} \quad
 D^{-}_{0,K-1}\equiv \frac{{\mathcal D}_{0c}}{1+{3U_0/\omega}}<{\mathcal
   D}_0 <{\mathcal D}_{0,c} \,,
\end{equation}
where 
\item On the range $0<{\mathcal D}_0 < D^{-}_{0,K-1} \equiv {{\mathcal
    D}_{0,c}/\left(1+{3U_0/\omega}\right)}$, the $K$-hotspot
  steady-state is linearly stable for all $\hat{\tau}_u\geq 0$.
\end{itemize}
\end{prop}

\noindent In terms of a scaled police diffusivity defined by 
$\epsilon^2 D_p\equiv {{\mathcal D}_0/\hat{\tau}_u}$, Proposition
\ref{q3:main_Kspots} implies the following:

\begin{cor}\label{q3:main_Kspots_pol} Under the conditions of
Proposition \ref{q3:main_Kspots}, we have the following:
\begin{itemize}
  \item For ${\mathcal D}_0>{\mathcal D}_{0,c}$, the $K$-hotspot
    steady-state is unstable for all scaled police diffusivities $\epsilon^2
    D_p>0$. Here ${\mathcal D}_{0,c}$ is defined in
   (\ref{stab:zero_d0min}) with $q=3$.
 \item On the range ${{\mathcal D}_{0,c}/\left(1+{3U_0/\omega}\right)}
   < {\mathcal D}_0 <{\mathcal D}_{0,c}$, and when $U_0<U_{0,\textrm{swit}}$,
   as defined in (\ref{q3:u0mon}), the $K$-hotspot steady-state is 
  unstable to a sign-alternating asynchronous oscillatory instability
   of the hotspot amplitudes if $\epsilon^2 D_p<{{\mathcal
       D}_{0}/\hat{\tau}_{uH,K-1}}$ where $\hat{\tau}_{uH,K-1}$ is defined in
  (\ref{q3:Kosc}). Alternatively, this steady-state is linearly
   stable when $\epsilon^2 D_p>{{\mathcal
       D}_{0}/\hat{\tau}_{uH,K-1}}$.
\item On the range $0<{\mathcal D}_0 < {{\mathcal
    D}_{0,c}/\left(1+{3U_0/\omega}\right)}$, the $K$-hotspot
  steady-state is linearly stable for all $\epsilon^2 D_p\geq 0$.
\end{itemize}
\end{cor}

We remark that the upper bound $U_{0,\textrm{swit}}$ in
(\ref{q3:u0mon}) can be calculated explicitly when $K=3$ and
$K=4$. When $K=3$, we calculate $\xi_1={1/3}$ and ${\mathcal
  Z}(\xi_1)={\left(1+\sqrt{10}\right)/\sqrt{3}}$.  We then obtain from
(\ref{q3:u0mon}) that the sign-alternating $K-1$ mode sets the Hopf
bifurcation threshold when
\begin{equation}\label{q3:U_0swit_3}
    U_0< U_{0,\textrm{swit}} \equiv \frac{3 S(\gamma-\alpha)}{2+\sqrt{10}} \approx
   (0.58114) S(\gamma-\alpha) \,, \qquad \mbox{for} \quad K=3 \,.
\end{equation}
Similarly, for $K=4$, we calculate $\xi_2=2-\sqrt{2}$, and 
\begin{equation*}
   {\mathcal Z}(\xi_2) =\frac{1}{\sqrt{3}}\left[ \frac{\sqrt{2}}{4} +
     \left( \frac{1}{2} + \frac{\sqrt{2}}{4} \right) \sqrt{27 -
       14\sqrt{2}}\right] \approx 1.5265 \,.
\end{equation*}
From (\ref{q3:u0mon}), the $K-1$ mode sets the Hopf bifurcation threshold when
\begin{equation}\label{q3:U_0swit_4}
    U_0< U_{0,\textrm{swit}} \approx (0.46847) S(\gamma-\alpha) \,, \qquad 
\mbox{for} \quad K=4 \,.
\end{equation}

\begin{figure}[htbp]
\centering
\includegraphics[width=0.47\textwidth,height=5.0cm]{figs/hopf_3_u02.eps}
\includegraphics[width=0.47\textwidth,height=5.0cm]{figs/hopf_3_u04.eps}
\caption{\label{fig:hopf_tau_3} Linear stability (shaded) region in
  the $\hat{\tau}_u$ versus ${\mathcal D}_0$ plane for $K=3$ when
  $S=6$, $\gamma=2$, and $\alpha=1$, and for $U_0=2$ (left panel) and
  $U_{0}=4$ (right panel), as characterized by Proposition
  \ref{q3:main_Kspots}. To the left of the thin vertical line the
  steady-state is unconditionally stable. The solid and dot-dashed curves
  are the Hopf bifurcation boundaries for the (sign-alternating) $j=2$
  mode and the $j=1$ mode, respectively. For $U_0=2$ (left panel) the
  Hopf boundary is determined by the $j=2$ mode.  For
  $U_0=4>U_{0,\textrm{swit}}\approx 3.478$ (right panel) the Hopf
  boundary consists of both the $j=2$ and $j=1$ mode. The
  three-hotspot steady-state is unstable to an oscillatory instability
  above the solid or dotted curves. At the ends of the Hopf
  bifurcation curves the Hopf eigenvalue tends to zero.}
\end{figure}

\begin{figure}[htbp]
\centering
\includegraphics[width=0.47\textwidth,height=5.0cm]{figs/pol_3_u02.eps}
\includegraphics[width=0.47\textwidth,height=5.0cm]{figs/pol_3_u04.eps}
\caption{\label{fig:hopf_pol_3} Same plot as Fig.~\ref{fig:hopf_tau_3}
  except in the scaled police diffusivity $\epsilon^{2}D_p={{\mathcal
      D}_0/\hat{\tau}_u}$ versus ${\mathcal D}_0$ plane for $K=3$,
  $S=6$, $\gamma=2$, and $\alpha=1$, with $U_0=2$ (left panel) and
  $U_{0}=4$ (right panel) (see Corollary
  \ref{q3:main_Kspots_pol}). The three-hotspot steady-state is
  linearly stable in the shaded region. This steady-state undergoes an
  oscillatory instability below the solid or dot-dashed curves. In the
  left panel the thin vertical line is the competition threshold
  ${\mathcal D}_{0,c}$. The additional thin vertical line in the right
  panel is where the Hopf boundary switches from $j=2$ to $j=1$. This
  switch occurs since $U_0=4>U_{0,\textrm{swit}}\approx 3.478$ (see
  (\ref{q3:U_0swit_3}) and the second statement of Corollary
  \ref{q3:main_Kspots_pol}). The full PDE simulations in
  Fig.~\ref{fig:valid_3spot_q3} and in Fig.~\ref{fig:valid_3spot_q3_b}
  are done at the marked points in the left and right panels,
  respectively.}
\end{figure}



We now illustrate our main stability results in Proposition
\ref{q3:main_Kspots} and Corollary \ref{q3:main_Kspots_pol} for 
$S=6$, $\gamma=2$, and $\alpha=1$. We take $K=3$ or $K=4$, and either
$U_0=2$ and $U_0=4$. For these parameters, (\ref{q3:U_0swit_3}) and
(\ref{q3:U_0swit_4}) yield that $U_{0,\textrm{swit}}\approx 3.487$ for
$K=3$ and $U_{0,\textrm{swit}}\approx 2.811$ for $K=4$. Therefore, for
both $K=3$ and $K=4$ it is only for the smaller value $U_0=2$ that the
sign-alternating mode sets the Hopf bifurcation threshold.

For $K=3$, the shaded region in Fig.~\ref{fig:hopf_tau_3} is the
theoretically predicted region of linear stability in the
$\hat{\tau}_u$ versus ${\mathcal D}_0$ parameter plane for $U_0=2$
(left panel) and for $U_0=4$ (right panel). In this figure the dotted
curve and solid curves are the Hopf bifurcation thresholds for the
$j=1$ mode and the sign-alternating $j=2$ mode. When $U_0=2$ (left
panel), the sign-alternating mode sets the boundary of the region of
stability, whereas for $U_0=4$ (right panel) both the $j=1$ and $j=2$
Hopf bifurcation thresholds form the boundary of the region of
stability. The corresponding region of stability in the scaled police
diffusivity $\epsilon^2 D_p$ versus ${\mathcal D}_0$ parameter plane
is shown in Fig.~\ref{fig:hopf_pol_3}.

Similar results for $K=4$ and for $U_0=2$ and $U_0=4$ are shown in
Fig.~\ref{fig:hopf_tau_4} in the $\hat{\tau}_u$ versus ${\mathcal D}_0$
plane and in Fig.~\ref{fig:hopf_pol_4} in the $\epsilon^2 D_p$ versus
${\mathcal D}_0$ plane. From the left panels of
Fig.~\ref{fig:hopf_tau_4} and Fig.~\ref{fig:hopf_pol_4}, the $j=K-1=3$
sign-alternating mode always sets the Hopf bifurcation boundary. However,
as seen in the right panels of Fig.~\ref{fig:hopf_tau_4} and 
Fig.~\ref{fig:hopf_pol_4}, where $U_0=4>U_{0,\textrm{swit}}\approx 2.811$, we 
observe that both the $j=3$ and $j=2$ modes determine the Hopf bifurcation
boundary when ${\mathcal D}_0<{\mathcal D}_{0,c}$.


\begin{figure}[htbp]
\centering
\includegraphics[width=0.47\textwidth,height=5.0cm]{figs/hopf_4_u02.eps}
\includegraphics[width=0.47\textwidth,height=5.0cm]{figs/hopf_4_u04.eps}
\caption{\label{fig:hopf_tau_4} Linear stability (shaded) region in
  the $\hat{\tau}_u$ versus ${\mathcal D}_0$ plane for $K=4$ when
  $S=6$, $\gamma=2$, and $\alpha=1$, and for $U_0=2$ (left panel) and
  $U_{0}=4$ (right panel). The solid, dot-dashed, and dashed curves
  are the Hopf bifurcation boundaries for the (sign-alternating) $j=3$
  mode and the other $j=2$ and $j=1$ modes. For $U_0=2$ (left panel)
  the Hopf boundary is determined by the sign-alternating $j=3$ mode.
  For $U_0=4>U_{0,\textrm{swit}}\approx 2.811$ (see
  (\ref{q3:U_0swit_4})), the Hopf boundary consists of both the $j=3$
  and $j=2$ mode. Asynchronous oscillatory instabilities of the
  hotspot amplitudes occur above any of the Hopf bifurcation curves.}
\end{figure}

\begin{rem}
More generally, for $K\geq 3$ there can be $K-2$ distinct mode
switches for the minimal Hopf bifurcation threshold on the interval
$D^{-}_{0,K-1}<{\mathcal D}_0<D^{+}_{0,K-1}$ when $U_0$ increases
beyond $U_{0,\textrm{swit}}$ towards $U_{0,\max}$. Although we do not
work out precise conditions for this {\em cascading behavior} of the
minimal Hopf bifurcation value here, we illustrate this phenomena in
Fig.~\ref{fig:cascade} for $K=4$, $S=6$, $\gamma=2$, $\alpha=1$, and
$U_0=5$. From this figure, we observe two mode switches of the minimal
Hopf bifurcation threshold. This suggests that as $U_0$ approaches the
existence threshold $U_{0,\max}$, there is a window of ${\mathcal
  D}_0$ where $K-2$ distinct modes of oscillatory instability of the
hotspot amplitudes can occur if $\hat{\tau}_u$ is large enough,
suggesting the possibility of intricate spatio-temporal dynamics in
this parameter regime.
\end{rem}

\begin{figure}[htbp]
\centering
\includegraphics[width=0.47\textwidth,height=5.0cm]{figs/pol_4_u02.eps}
\includegraphics[width=0.47\textwidth,height=5.0cm]{figs/pol_4_u04.eps}
\caption{\label{fig:hopf_pol_4} Plot corresponding to
  Fig.~\ref{fig:hopf_tau_4} in the scaled police diffusivity
  $\epsilon^{2}D_p={{\mathcal D}_0/\hat{\tau}_u}$ versus ${\mathcal
    D}_0$ plane for $K=4$, $S=6$, $\gamma=2$, and $\alpha=1$, with
  $U_0=2$ (left panel) and $U_{0}=4$ (right panel). The four-hotspot
  steady-state is linearly stable in the shaded region. This
  steady-state undergoes an asynchronous oscillatory instability below
  either of the three Hopf bifurcation curves.  In the left panel the
  thin vertical line is the competition threshold ${\mathcal
    D}_{0,c}$. The additional thin vertical line in the right panel is
  where the Hopf boundary switches from $j=3$ to $j=2$. The Hopf
  eigenvalue tends to zero at the ends of each of the two curves.}
\end{figure}

\begin{figure}[htbp]
\centering
\includegraphics[width=0.54\textwidth,height=5.0cm]{figs/hopf_cascade.eps}
\caption{\label{fig:cascade} Linear stability (shaded) region in the
  $\hat{\tau}_u$ versus ${\mathcal D}_0$ plane for $K=4$ when $S=6$,
  $\gamma=2$, and $\alpha=1$, and for $U_0=5$. The solid, dot-dashed,
  and dashed curves are the Hopf bifurcation boundaries for the
  (sign-alternating) $j=3$ mode, the $j=2$ mode, and the $j=1$ mode,
  respectively. Notice that the minimal Hopf bifurcation threshold on
  $D^{-}_{0,K-1}<{\mathcal D}_0<D^{+}_{0,K-1}$ now consists of all
  three modes. Asynchronous oscillatory instabilities of the hotspot
  amplitudes occur above either of the three Hopf bifurcation curves.}
\end{figure}



\subsection{Comparison of Linear Stability Theory with PDE Simulations: $q=3$}
\label{sec:numerics_q3}


\begin{figure}[htbp]
\centering
\includegraphics[width=0.47\textwidth,height=5.0cm]{figs/amp_2_run6.eps}
\includegraphics[width=0.47\textwidth,height=5.0cm]{figs/amp_2_run4.eps}
\caption{\label{fig:valid_2_q3_u04} Plot of the spot amplitudes
  computed numerically from the full PDE system (\ref{eq:pol-main})
  for a two-spot pattern with $S=6$, $\gamma=2$, $\alpha=1$, $U_0=4$,
  $\epsilon=0.035$, and $q=3$, at two of the marked points in the
  right panel of Fig.~\ref{fig:hopf_pol_2}.  Left panel:
  $\hat{\tau}_u=1.8$ and ${\mathcal D}_0=0.3$, so that $\epsilon^2
  D_p\approx 0.1667$. Spot amplitudes are unstable to a competition
  instability. Right panel: $\hat{\tau}_u=1.0$ and ${\mathcal D}_0=0.15$, so
  that $\epsilon^2D_p = 0.15$. Spot amplitude asynchronous
  oscillations decay in time and there is no competition instability.
  These results are consistent with the linear stability predictions
  in the right panel of Fig.~\ref{fig:hopf_pol_2} (see also the right
  panel of Fig.~\ref{fig:hopf_tau_2}).}
\end{figure}

\begin{figure}[htbp]
\centering
\includegraphics[width=0.47\textwidth,height=5.0cm]{figs/amp_2_run5.eps}
\includegraphics[width=0.47\textwidth,height=5.0cm]{figs/amp_2_run7.eps}
\caption{\label{fig:valid_2_q3_u04_b} Plot of oscillatory
  instabilities of the spot amplitudes computed numerically from the
  full PDE system (\ref{eq:pol-main}) for a two-spot pattern with
  $S=6$, $\gamma=2$, $\alpha=1$, $U_0=4$, $\epsilon=0.035$, and $q=3$,
  at the marked points in the right panel of Fig.~\ref{fig:hopf_pol_2}
  where an oscillatory instability occurs.  Left panel:
  $\hat{\tau}_u=2.6$ and ${\mathcal D}_0=0.15$, so that $\epsilon^2
  D_p\approx 0.0577$. Spot amplitudes are unstable to asynchronous
  oscillations, which leads to the collapse of both hotspots.  Right
  panel: $\hat{\tau}_u=3.0$ and ${\mathcal D}_0=0.15$, so that
  $\epsilon^2 D_p= 0.05$. Spot amplitudes are unstable to asynchronous
  oscillations, but now only one of the two hotspots collapses.}
\end{figure}

\begin{figure}[htbp]
\centering
\includegraphics[width=0.32\textwidth,height=5.0cm]{figs/amp_3_run3.eps}
\includegraphics[width=0.32\textwidth,height=5.0cm]{figs/amp_3_run1.eps}
\includegraphics[width=0.32\textwidth,height=5.0cm]{figs/amp_3_run2.eps}
\caption{\label{fig:valid_3spot_q3} Plot of the spot amplitudes
  computed numerically from the full PDE system (\ref{eq:pol-main})
  for a three-spot pattern with $S=6$, $\gamma=2$, $\alpha=1$,
  $U_0=2$, $\epsilon=0.05$, and $q=3$.  Left panel: $\hat{\tau}_u=1.3$
  and ${\mathcal D}_0=0.07$, so that $\epsilon^2 D_p\approx
  0.0538$. Spot amplitudes are stable to asynchronous oscillations and
  to the competition instability.  Middle panel: $\hat{\tau}_u=2.5$
  and ${\mathcal D}_0=0.07$, so that $\epsilon^2 D_p=0.028$.  Spot
  amplitudes are unstable to asynchronous oscillations due to the
  sign-altering mode, which leads to the collapse of middle
  hotspot. Right panel: $\hat{\tau}_u=2.5$ and ${\mathcal D}_0=0.12$,
  so that $\epsilon^2 D_p=0.048$. Spot amplitudes are unstable to a
  competition instability due to the sign-altering mode, which leads
  to the collapse of the first and third hotspots. These results are
  consistent with the linear stability predictions in the left panels
  of Fig.~\ref{fig:hopf_tau_3} and Fig.~\ref{fig:hopf_pol_3}.  The
  results correspond to the marked points in the left panel of
  Fig.~\ref{fig:hopf_pol_3}. }
\end{figure}

\begin{figure}[htbp]
\centering
\includegraphics[width=0.47\textwidth,height=5.0cm]{figs/amp_3_run4.eps}
\includegraphics[width=0.47\textwidth,height=5.0cm]{figs/amp_3_run6.eps}
\caption{\label{fig:valid_3spot_q3_b} Plot of the spot amplitudes
  computed numerically from the full PDE system (\ref{eq:pol-main})
  for a three-spot pattern with $S=6$, $\gamma=2$, $\alpha=1$,
  $U_0=2$, $\epsilon=0.025$, and $q=3$, for the marked points in the
  right panel of Fig.~\ref{fig:hopf_pol_3}. Left panel:
  $\hat{\tau}_u=2.5$ and ${\mathcal D}_0=0.01$, so that $\epsilon^2
  D_p\approx 0.004$. Unstable asynchronous spot amplitude oscillations
  are of sign-altering type as predicted by the theory. Right panel:
  $\hat{\tau}_u=1.6$ and ${\mathcal D}_0=0.03$, so that $\epsilon^2
  D_p=0.01875$.  The unstable spot amplitude oscillations are no
  longer sign-altering. Here the first and third spots exhibit
  anti-phase oscillations as predicted by the right panel of
  Fig.~\ref{fig:hopf_pol_3}.}
\end{figure}

\setcounter{equation}{0}
\setcounter{section}{5}
\section{Oscillatory Instabilities of the Hotspot Amplitudes: $q\protect\neq3$
and $q>1$}\label{sec:stab_qn3}

In this section we analyze the NLEP (\ref{stab:nlep_final_1}) for the
general case where $q\neq 3$, by determining the roots of
$\zeta(\lambda)=0$ in $\mbox{Re}(\lambda)>0$. In 
(\ref{stab:Cval_1}) we re-write ${\mathcal C}(\lambda)$ as
\begin{equation}\label{genq:C}
\mathcal{C}(\lambda)=\frac{\eta}{b}\left(1+\tilde{\tau}_{j}\lambda\right)
\left(1-\frac{b}{3-\lambda}\right)\,, \qquad\mbox{where} \qquad
\eta \equiv \frac{9\omega}{2qU_0}  \,, \quad 
b\equiv \frac{9\chi_{0,j}}{2}\,, \quad
\tilde{\tau}_{j} \equiv \frac{\hat{\tau}_{u}}{D_j \kappa_q} \,, \quad
 \frac{1}{\chi_{0,j}} = 1 + \frac{\kappa_3 D_j}{\alpha} \,.
\end{equation}
To relate our key parameter $b$ (which depends on $j$) to the
diffusivity ${\mathcal D}_0$, we first use the expression for
$\chi_{0,j}$ to write $D_j$ in terms of $b$ as
$D_{j}=\left[{\alpha/(2\kappa_3)}\right] \left({9/b}-2\right)$. Then,
upon using (\ref{q3:dlow}) for $\alpha/(2\kappa_3)$ and
(\ref{eq:pol-D_j}) to relate $D_j$ to ${\mathcal D}_0$, we obtain that
\begin{equation}\label{param:b}
      \frac{{\mathcal D}_0}{D^{-}_{0,j}}= \frac{9}{b}-2 \,, \qquad
  \mbox{or} \qquad   b = \frac{9}{2 + {\mathcal D}_0/D^{-}_{0,j}}  \,.
\end{equation}
Thus, ${\mathcal D}_0>0$ only when $b<{9/2}$.  Here $\dzjm$ is
defined in terms of the competition threshold ${\mathcal D}_{0c}$ of
(\ref{stab:zero_d0min}) by
\begin{equation}\label{param:dzjm}
   \dzjm \equiv \frac{{\mathcal D}_{0,c}}{\left(1+{q U_0/\omega}\right)}
  \left( \frac{1+\cos\left({\pi/K}\right)}
  {1-\cos\left({\pi j /K}\right)}\right) \,, \qquad
  \dzjp \equiv {\mathcal D}_{0,c} \left( \frac{1+\cos\left({\pi/K}\right)}
  {1-\cos\left({\pi j /K}\right)}\right) \,, \qquad
  \frac{\dzjp}{\dzjm}=1+ \frac{qU_0}{\omega} \,,
\end{equation}
where $D^{+}_{0,K-1}={\mathcal D}_{0,c}$.  In our analysis below, the
following ranges of $b$ will play a prominent role:
\bsub \label{genq:b_range}
\begin{align}
  (I):\quad  & 3<b<{9/2}  \quad \implies \quad \dzjm>{\mathcal D}_0>0 
   \quad \implies \quad {\mathcal C}(0)<0  \,, \\
  (II):\quad  & b_c\equiv {3\eta/\left(\eta + {3/2}\right)} < b< 3 \quad
    \implies \quad \dzjp >{\mathcal D}_0 >\dzjm \quad
    \implies \quad {1/2}>{\mathcal C}(0)>0 \,, \\
  (III):\quad  & b< b_c  \quad \implies \quad {\mathcal D}_0>\dzjp 
 \quad \implies \quad {\mathcal C}(0)>{1/2}\,.
\end{align}
\esub Since ${\mathcal F}(0)={1/2}$, and ${\mathcal C}(0)={1/2}$ when
$b=b_c$, we conclude that $b=b_c$ corresponds to a zero-eigenvalue
crossing.

\subsection{Analytical Results Based on the Winding Number Criterion}
\label{genq:wind_number}

Here we use the winding number criterion of \S \ref{sec:stab_arg} to
determine some rigorous results for the number $N$ of unstable
eigenvalues in $\mbox{Re}(\lambda)>0$ for the ranges of $b$ listed in
(\ref{genq:b_range}). In our analysis we will assume that Conjecture
\ref{conj:imag} on ${\mathcal F}_R(\lambda_I)$ and ${\mathcal
  F}_I(\lambda_I)$ holds for $q>1$.  With $\zeta(\lambda)$ as defined
in (\ref{stab:merom}), (\ref{key:wind}) when $\hat{\tau}_u>0$ yields
that
\begin{equation}\label{genq:wind}
  N = \frac{3}{2} + \frac{1}{\pi} \left[\arg \zeta \right]_{\Gamma_I^{+}} \,.
\end{equation}
To calculate $\left[\arg \zeta \right]_{\Gamma_I^{+}}$ we need the
following properties of the real and imaginary parts of ${\mathcal
  C}(i\lambda_I)$:

\begin{lem}\label{genq:prop_C} Let ${\mathcal C}(i\lambda_I)=
{\mathcal C}_R(\lambda_I) + i{\mathcal C}_{I}(\lambda_I)$. Then, from
(\ref{genq:C}) we have
\begin{equation}\label{genq:CrCI}
\mathcal{C}_{R}(\lambda_{I})  = \frac{\eta}{b}
 \left(1+\tilde{\tau}_{j} b-\frac{3b}{9+\lambda_{I}^{2}}
\left(1+3\tilde{\tau}_{j}\right)\right)\,, \qquad
\mathcal{C}_{I}(\lambda_{I})  =  \frac{\eta\lambda_{I}}{b}
  \left[ \tilde{\tau}_j - \frac{b(1+3\tilde{\tau}_j)}{(9+\lambda_{I}^{2})}
  \right]\,.
\end{equation}
For the imaginary part we have:
\begin{itemize}
\item [{(i)}] $\mathcal{C}_{I}(\lambda_{I})\sim \left(9 b\right)^{-1}
  \eta \lambda_{I}\left[3\tilde{\tau}_{j}(3-b)-b\right]$ as
  $\lambda_{I}\to 0^{+}$,
\item[{(ii)}] $\mathcal{C}_{I}(\lambda_{I})\sim b^{-1}\eta \tilde{\tau}_j
 \lambda_{I}$ as $\lambda_{I}\to+\infty$.
\item [{(iii)}] If $b<3/(1+\frac{1}{3\tilde{\tau}_{j}})$,
then $\mathcal{C}_{I}(\lambda_{I})>0$ for all $\lambda_I>0$.
\item[{(iv)}] If $b>3/(1+\frac{1}{3\tilde{\tau}_{j}})$, then
$\mathcal{C}_{I}(\lambda_{I})<0$ on $
0<\lambda_{I}<\sqrt{3(b-3)+\frac{b}{\tilde{\tau}_{j}}}\equiv\lambda_{I_{I}}$
and $\mathcal{C}_{I}(\lambda_{I})>0$ on $\lambda_I>\lambda_{I_{I}}$.
\end{itemize}
Alternatively, for the real part we have:
\begin{itemize}
\item [{(v)}] $\mathcal{C}_{R}^{\prime}(\lambda_{I})>0$ for $\lambda_{I}>0$.
\item [{(vi)}] ${\mathcal C}_{R}(\lambda_I)\sim
  b^{-1}\eta(1+\tilde{\tau}_j b)$ as $\lambda_I\to \infty$.
\item [{(vii)}] $\mathcal{C}_{R}(0)>0$ if $b<3$ and 
$\mathcal{C}_{R}(0)<0$ if $b>3$. When
$b>3$, then $\mathcal{C}_{R}(\lambda_{I})<0$ on 
$0<\lambda_{I}<\sqrt{\frac{3(b-3)}{1+\tilde{\tau}_{j}b}}\equiv \lambda_{IR}$,
and $\mathcal{C}_{R}(\lambda_{I})>0$ on $\lambda_I>\lambda_{IR}$.
\item [{(viii)}] $\mathcal{C}_{R}(0)>{1/2}$ iff $b<b_c\equiv
  {3\eta/(\eta+{3/2})}$, where $b_c$ is the zero-eigenvalue crossing.
\end{itemize}
\end{lem}

With properties (ii) and (vi) of Lemma \ref{genq:prop_C}, together
with the decay of ${\mathcal F}_R(\lambda_I)$ and ${\mathcal
  F}_I(\lambda_I)$ as $\lambda_I\to +\infty$ (see Proposition
\ref{rig:imag_f}), we conclude that
\begin{equation*}
\zeta(i\lambda_{I})\sim b^{-1}\eta \left(1+\tilde{\tau}_{j}b\right)+i
b^{-1}\eta \tilde{\tau}_{j}\lambda_{I} \quad \mbox{ as } \quad
\lambda_{I}\to+\infty\,.
\end{equation*}
Therefore, with respect to the origin, the path
$\zeta(i\lambda_I)=\zeta_R(\lambda_I)+i\zeta_I(\lambda_I)$ begins (as
$\lambda_I\to \infty$) asymptotically close to the positive infinity
of the imaginary axis in the complex $\zeta$ plane. 

Moreover, from property (v) of Lemma \ref{genq:prop_C}, and under 
Conjecture \ref{conj:imag} that ${\mathcal F}_R^{\prime}(\lambda_I)<0$
for all $\lambda_I>0$, we conclude that
\begin{equation}
\zeta_{R}^{\prime}(\lambda_{I})>0 \quad \mbox{for all} \quad
\lambda_{I}>0\,.\label{genq:zetar}
\end{equation}
With this key result, the path $\zeta(i\lambda_I)$ in the
$\zeta$-plane for $\lambda_I>0$ can only intersect the imaginary
$\zeta_I$ axis exactly one or zero times. In particular if
$\zeta(0)\equiv \zeta_R(0)>0$, then $\zeta_R(\lambda_I)>0$ for all
$\lambda_I>0$ so that $\left[\arg \zeta \right]_{\Gamma_I^{+}} =
-{\pi/2}$ and $N=1$ from (\ref{genq:wind}). In contrast, if
$\zeta(0)\equiv \zeta_R(0)<0$, then there is a unique
$\lambda_{I}^{\star}>0$ for which $\zeta_R(\lambda_I^{\star})=0$. In
this case, (\ref{genq:wind}) yields that
\bsub\label{genq:one_cross}
\begin{align}
   (I):\quad &  \zeta_I(\lambda_I^{\star})>0 \quad \implies  \quad
\left[\arg \zeta \right]_{\Gamma_I^{+}} = {\pi/2} \quad \implies \quad N=2\,
 \label{genq:N=2} \\
   (II):\quad &  \zeta_I(\lambda_I^{\star})<0 \quad \implies  \quad
\left[\arg \zeta \right]_{\Gamma_I^{+}} = {-3\pi/2} \quad \implies \quad N=0\,.
 \label{genq:N=0} 
\end{align}
\esub

With these preliminary observations, we obtain the following
instability result for the $j$-th mode on the range ${\mathcal D}_0>\dzjp$.

\begin{prop}\label{prop:genq:N=1} Suppose that ${\mathcal D}_0>\dzjp$ and 
that Conjecture \ref{conj:imag} holds. Then, for the $j$-th mode with
$j=1,\ldots,K-1$, we have $N=1$ for all $\tilde{\tau}_j\geq 0$.
\end{prop}

\begin{proof} When ${\mathcal D}_0>\dzjp$, we have 
$b<b_c\equiv {3\eta/(\eta+{3/2})}$ and consequently ${\mathcal
    C}_R(0)>{1/2}$ from (III) of (\ref{genq:b_range}). This yields
  $\zeta_R(0)>0$, and thus $\zeta_R(\lambda_I)>0$ for all
  $\lambda_I>0$ using the monotonicity result (\ref{genq:zetar}),
  which holds when ${\mathcal F}_R^{\prime}(\lambda_I)<0$ for all
  $\lambda_I>0$. Therefore $\left[\arg \zeta \right]_{\Gamma_I^{+}}
  =-{\pi/2}$ and (\ref{genq:wind}) yields $N=1$ for all
  $\tilde{\tau}_j>0$.
\end{proof}

Together with the ordering principle $D^{+}_{0,j+1}<D^{+}_{0,j}$ for
$j=1,\ldots, K-2$ from (\ref{q3:d_order}), this instability result
proves, for any $\hat{\tau}_u>0$, that there are exactly $K-1$
positive real eigenvalues of the NLEP (\ref{stab:nlep_final_1}) when
${\mathcal D}_0>D^{+}_{0,1}$.

\begin{prop}\label{prop:genq:N=02} Suppose that ${\mathcal D}_0<\dzjp$ and 
that Conjecture \ref{conj:imag} holds. Then, for the $j$-th mode with
$j=1,\ldots,K-1$, we have either $N=0$ or $N=2$ for all $\tilde{\tau}_j>0$.
Moreover, if $\tilde{\tau}_j\ll 1$ we have $N=0$.
\end{prop}

\begin{proof}
If ${\mathcal D}_0<\dzjp$, then $b>{3\eta/(\eta+{3/2})}$, and so
${\mathcal C}_R(0)<{1/2}$ by (II) of (\ref{genq:b_range}). Therefore,
since $\zeta_{R}(0)<0$, $\zeta_R(\infty)>0$, and
$\zeta_R^{\prime}(\lambda_I)>0$, which holds when ${\mathcal
  F}_R^{\prime}(\lambda_I)<0$ for all $\lambda_I>0$, it follows that
there is a unique root $\lambda_I^{\star}$ for which
$\zeta_R(\lambda^{\star}_I)=0$. From (\ref{genq:one_cross}), we have
for all $\hat{\tau}_j>0$ that either $N=0$ or $N=2$ depending on the sign of
$\zeta_I(\lambda_I^{\star})$. 

Next, we prove that $N=0$ if $\tilde{\tau}_j\ll 1$. For
$\tilde{\tau}_j\ll 1$, (\ref{genq:CrCI}) yields that
\begin{equation*}
    {\mathcal C}_R(\lambda_I) \sim \frac{\eta}{b}\left( 1 - 
 \frac{3b}{9+\lambda_I^2} \right) \,,
\end{equation*}
uniformly in $\lambda_I$. It follows that
$\zeta_R(\lambda_I^{\star})=0$ at some $\lambda_I^{\star}={\mathcal
  O}(1)$ when $\tilde{\tau}_j\ll 1$.  However, for $\tilde{\tau}_j\ll
1$, we have from (\ref{genq:CrCI}) that ${\mathcal
  C}_I(\lambda_I^{\star})\sim -{\eta
  \lambda_I/\left(9+\lambda_I^2\right)}<0$.  Under Conjecture
\ref{conj:imag} that ${\mathcal F}_I(\lambda_I)>0$, we conclude that
$\zeta_I(\lambda_I^{\star})={\mathcal C}_{I}(\lambda_I^{\star})-
    {\mathcal F}_{I}(\lambda_{I}^{\star})<0$. Therefore, from
    (\ref{genq:N=0}) we conclude that $N=0$.
\end{proof}

This result proves that the $j$-th mode is linearly stable on the
range ${\mathcal D}_0<\dzjp$ whenever $\tilde{\tau}_j\ll 1$. The next
result determines $N$ for $\tilde{\tau}_j\gg 1$ on the entire range
${\mathcal D}_0<\dzjp$.

\begin{prop}\label{prop:genq:hopf} Suppose that $\dzjm<{\mathcal D}_0<\dzjp$
and that Conjecture \ref{conj:imag} holds. Then, for the $j$-th mode
with $j=1,\ldots,K-1$, we have $N=2$ when $\tilde{\tau}_j \gg 1$.  In
contrast, if ${\mathcal D}_0<\dzjm$, then $N=0$ when $\tilde{\tau}_j
\gg 1$.
\end{prop}

\begin{proof}
We first observe from (\ref{genq:b_range}) that $\dzjm<{\mathcal
  D}_0<\dzjp$ when $b_c<b<3$ and ${\mathcal D}_0<\dzjm$ when
$b>3$. For ${\mathcal D}_0<\dzjp$, we have $\zeta_R(0)<0$ and so there
is a unique root $\lambda_I^{\star}$ to
$\zeta_R(\lambda_I^{\star})=0$. For $\tilde{\tau}_j\gg 1$, and for
$b>b_c\equiv {3\eta/(\eta+{3/2})}$, this unique root of
$\zeta_R(\lambda_I)$ occurs for $\lambda_I={\mathcal
  O}(\tilde{\tau}_j^{-1/2})\ll 1$. By setting ${\mathcal
  C}_R(\lambda_I)={\mathcal F}_{R}(\lambda_I)$, and using
$\lambda_I={\mathcal O}(\tilde{\tau}_j^{-1/2})\ll 1$ together with
(\ref{genq:CrCI}) for ${\mathcal C}_R$, we get that
\begin{equation*}
  \frac{\eta}{b} \left( 1 + \frac{b}{9} (\tilde{\tau}_j \lambda_I^2-3)
  \right) \sim {\mathcal F}_R(0)=\frac{1}{2} \,.
\end{equation*}
In this way, we obtain for $\tilde{\tau}_j\gg 1$ that the unique root
$\lambda_{I}^{\star}$ of $\zeta_R(\lambda_I)=0$ occurs when
\begin{equation}
   \lambda_{I}^{\star} \sim {\beta/\tilde{\tau}_j^{1/2}} \,, \qquad 
 \beta \equiv \sqrt{3 \left(1- \frac{3}{b}\right) + \frac{9}{2\eta}}\,.
\end{equation}
From (\ref{genq:CrCI}) for ${\mathcal C}_I$, we then calculate for
$\tilde{\tau}_j\gg 1$ that
\begin{equation}\label{shit:eq}
   {\mathcal C}_I(\lambda_I^{\star}) \sim \frac{\eta\beta}{3b}(3-b) 
  \tilde{\tau}_j^{1/2} = {\mathcal O}(\tilde{\tau}_j^{1/2}) \,.
\end{equation}
When $b>3$, corresponding to ${\mathcal D}_0<\dzjm$, we have ${\mathcal
  C}_I(\lambda_I^{\star})<0$. Therefore, with ${\mathcal
  F}_I(\lambda_I)>0$ from Conjecture \ref{conj:imag}, we conclude that
$\zeta_I(\lambda_{I}^{\star})<0$. This yields that $N=0$ from
(\ref{genq:N=0}). Now suppose that $b_c<b<3$, corresponding to
$\dzjm<{\mathcal D}_0<\dzjp$.  Then, since ${\mathcal
  C}_I(\lambda_I^{\star})>0$, with ${\mathcal
  C}_I(\lambda_I^{\star})=\tilde{\tau}_j^{1/2}\gg 1$ for
$\tilde{\tau}_j\gg 1$, while ${\mathcal
  F}_I(\lambda_I^{\star})={\mathcal O}(1)$, we conclude that
$\zeta_I(\lambda_I^{\star})>0$. This yields that $N=2$ from 
(\ref{genq:N=2}).
\end{proof}

On the range $\dzjm<{\mathcal D}_0<\dzjp$, Propositions
\ref{prop:genq:hopf} and \ref{prop:genq:N=02} show for the $j$-th mode
that $N=2$ for $\tilde{\tau}_j\gg 1$ and $N=0$ for $\tilde{\tau}_j\ll
1 $. Therefore, by continuity, there is a Hopf bifurcation value of
$\tilde{\tau}_j$ on this range of ${\mathcal D}_0$. Although this
proves the existence of a Hopf bifurcation threshold for
$\hat{\tau}_u$ on the range $\dzjm<{\mathcal D}_0<\dzjp$ for any mode
$j=1,\ldots,K-1$, it does not establish uniqueness of this threshold 
or provide any its qualitative properties. This is done numerically in
\S \ref{genq:param}.

The remaining issue relates to the range ${\mathcal D}_0<\dzjm$, where
we have proved that $N=0$ when either $\tilde{\tau}_j\ll 1$ or
$\tilde{\tau}_j\gg 1$. Next, we study for this range of ${\mathcal
  D}_0$ whether $N=0$ {\em for all} $\tilde{\tau}_j>0$.

To examine this question we will proceed as follows: Suppose that
${\mathcal C}_R(0)<{1/2}$ and that ${\mathcal C}_{R}(\infty)>{1/2}$.
Then, with Conjecture \ref{conj:imag} that ${\mathcal
  F}_R^{\prime}(\lambda_I)<0$ for all $\lambda_I>0$, and with
${\mathcal F}_R(0)={1/2}$, it follows that the unique root
$\lambda_{I}^{\star}$ to $\zeta_R(\lambda_I)=0$ satisfies
$\lambda_{I}^{\star}<\lambda_{Im}$ where $\lambda_{Im}$ is defined by
$C_{R}(\lambda_{Im})={1/2}$. If we can then show that
$C_{I}(\lambda_{Im})<0$, it follows that
$C_{I}(\lambda_{I}^{\star})<0$. Then, by Conjecture \ref{conj:imag}
that ${\mathcal F}_I(\lambda_I)>0$ for all $\lambda_I>0$, we obtain
that $\zeta_I(\lambda_{I}^{\star})<0$, and consequently $N=0$ from
(\ref{genq:N=0}).  Therefore, our goal is to determine the range of
$b$, with $b>3$, for which
\begin{equation}\label{genq:bound}
      {\mathcal C}_{R}(0)<{1/2} \,, \qquad
{\mathcal C}_{R}(\infty)>{1/2} \,, \qquad \mbox{and} \quad
  {\mathcal C}_{I}(\lambda_{Im})<0  \quad \mbox{where} \quad
 {\mathcal C}_{R}(\lambda_{Im}) = {1/2} \,.
\end{equation}

When $b>b_c$ we have ${\mathcal C}_R(0)<1/2$, and from property (vi)
of Lemma \ref{genq:prop_C} we have ${\mathcal C}_R(\infty)>{1/2}$
provided that
\begin{equation}\label{eq1:gap}
   \frac{\eta}{b} \left(1+ \tilde{\tau}_j b\right) > \frac{1}{2}\,.
\end{equation}
Now by using (\ref{genq:CrCI}) for ${\mathcal C}_R$, we obtain that
${\mathcal C}_R(\lambda_{Im})={1/2}$ when
\begin{equation*}
     \frac{1+3\tilde{\tau}_j}{9+\lambda_{Im}^2} = \frac{1}{3\eta} \left[
   \frac{\eta}{b} (1+\tilde{\tau}_j b) - \frac{1}{2}\right]\,.
\end{equation*}
By using this expression in (\ref{genq:CrCI}) for ${\mathcal
  C}_I(\lambda_I)$ we get, after some algebra, that
\begin{equation*}
    {\mathcal C_{I}}(\lambda_{Im}) = \frac{\eta \lambda_{Im}}{b} \left(
  \tilde{\tau}_j - \frac{b (1+3\tilde{\tau}_j)}{9+\lambda_{Im}^2} \right)
  = -\frac{\eta \lambda_{Im}}{3b} \left[ \tilde{\tau}_j (b-3) + 1 - 
  \frac{b}{2\eta} \right] \,.
\end{equation*}
We conclude that ${\mathcal C_{I}}(\lambda_{Im})<0$ when
$1+\tilde{\tau}_j b - {b/(2\eta)}>3\tilde{\tau}_j$. 
Since, from (\ref{eq1:gap}),  we have ${\mathcal C}_R(\infty)>{1/2}$ when
$1+\tilde{\tau}_j b - {b/(2\eta)}>0$, we conclude that the inequalities in 
(\ref{genq:bound}) hold when
\begin{equation}\label{eq2:gap}
   b>b_c \equiv \frac{3\eta}{\eta + {3/2}} \qquad \mbox{and} \qquad
1+\tilde{\tau}_j b - {b/(2\eta)}>3 \tilde{\tau}_j \,.
\end{equation}

We now determine a range of $b$, independent of $\tilde{\tau}_j$, for
which the inequalities (\ref{eq2:gap}) hold.  A sufficient condition
for (\ref{eq2:gap}) to hold is that $b>3$ and $b<2\eta$. If
$\eta<{3/2}$, there is no such range of $b$. If $\eta>{9/2}$, 
these inequalities hold on the full range $3<b<{9/2}$ where ${\mathcal
  D}_0<\dzjm$, as given in (I) in (\ref{genq:b_range}). However, when
${3/2}<\eta<{9/2}$, the interval $3<b<2\eta$ is only a subset of the
full range in (I) of (\ref{genq:b_range}) where ${\mathcal
  D}_0<\dzjm$.  By using (\ref{genq:C}) for $\eta$, together with
(\ref{param:b}) to relate $b$ to ${\mathcal D}_0$, we summarize our
result as follows:

\begin{prop}\label{prop:genq:gap} Suppose that Conjecture \ref{conj:imag} 
holds. Then, for the $j$-th mode with $j=1,\ldots,K-1$ we have the following:
\bsub\label{genq:gap}
\begin{align}
  (I) \quad  &\mbox{Suppose } U_0<{2\omega/q} \mbox{ and } 
    {\mathcal D}_0<\dzjm\,. \mbox{ Then } N=0 \,\, \mbox{for all }
     \hat{\tau}_u >0 \,.\label{genq:gap_1} \\
  (II) \quad  &\mbox{Suppose } {2\omega/q} < U_0<{3\omega/q} \mbox{ and } 
    \left(\frac{q U_0}{\omega}-2\right) \dzjm<{\mathcal D}_0<\dzjm\,. 
   \mbox{ Then } N=0 \,\, \mbox{for all } \hat{\tau}_u>0 \,.\label{genq:gap_2}
\end{align}
\esub This result provides no stability information for the range
${\mathcal D}_0<\dzjm$ when $U_0>{3\omega/q}$. 
\end{prop}

As a result of the ordering principle $D^{-}_{0,K-1}<D^{-}_{0,j}$ for
$j=1,\ldots,K-2$ and $K\geq 3$, we conclude from (I) of Proposition
\ref{prop:genq:gap}, upon recalling $\omega=S(\gamma-\alpha)-U_0$,
that a $K$-hotspot pattern is linearly stable for all $\hat{\tau}_u>0$
on the range ${\mathcal D}_0<D^{-}_{0,K-1}$ when
$U_0<{2S(\gamma-\alpha)/(q+2)}$. This result is weaker than that
obtained for the explicitly solvable case $q=3$, In fact, since
$\omega=S(\gamma-\alpha)-U_0$, it provides no stability information on
${\mathcal D}_0<D^{-}_{0,j}$ when $U_0>3{(S-\gamma)/(q+3)}$. For
$q=3$, we recall from Proposition \ref{q3:roots_quad} that for the
$j$-th mode we have $N=0$ for ${\mathcal D}_0<\dzjm$ for all
$\hat{\tau}_u>0$ without any restriction on $U_0$.

In \S \ref{genq:param} we will investigate numerically the possibility
of a Hopf bifurcation for the range $0<{\mathcal D}_0<D^{-}_{0,j}$
when $U_0>{3\omega/q}$, for which Proposition \ref{prop:genq:gap} does not
apply. Our numerical procedure in \S \ref{genq:param} suggests that
no Hopf bifurcation exists on the range $0<{\mathcal
  D}_0<D^{-}_{0,j}$ for any $U_0<U_{0,\max}$, as qualitatively identical to
the second statement proved in Proposition \ref{q3:roots_quad} for the
explicitly solvable case $q=3$.

To gain some insight into the behavior of the Hopf bifurcation
threshold $\tilde{\tau}_{j}$ for the $j$-th mode, we now derive a
scaling law for it to show that $\tilde{\tau}_{j}\to
\infty$ and $\lambda_{I}\to 0^{+}$ as ${\mathcal D}_0\to D^{-}_{0,j}$
from above, or equivalently as $b\to 3$ from below.  For $b\to 3^{-}$,
we look for a solution to $\zeta(i\lambda_I)=0$ with $\lambda_I\to
0$, $\tilde{\tau}_j\to \infty$, with the distinguished balance
$\lambda_{I}={\mathcal O}(\tilde{\tau}_j^{-1/2})$.  By setting
${\mathcal C}_R(\lambda_I)={\mathcal F}_{R}(\lambda_I)$, and using
${\mathcal F}_{R}(\lambda_I)\sim {1/2}$ together with (\ref{genq:CrCI}) for
${\mathcal C}_R$, we get that
\begin{equation*}
    \frac{\eta}{b} \left(1 - \frac{b}{9} \left(3 - \tilde{\tau}_j \lambda_I^2
  \right)\right) \sim \frac{1}{2} \,.
\end{equation*}
By solving for $\tilde{\tau}_j\lambda_I^2$ and letting $b\to 3$, we obtain 
that $\tilde{\tau}_j\lambda_I^2 = 9/(2\eta) + {3(b-3)/b}$. By letting $b\to 3$,
we conclude that
\begin{equation}\label{genq:lami_asy}
    \lambda_I \sim \sqrt{ \frac{9}{2\eta\tilde{\tau}_j}}  
 \,, \qquad \mbox{as} \quad \tilde{\tau}_j \to \infty \,.
\end{equation}
Then, we set ${\mathcal C}_I(\lambda_I)={\mathcal F}_I(\lambda_I)$, and use
(\ref{genq:CrCI}) for ${\mathcal C}_I$, together with the local behavior 
for ${\mathcal F}_I(\lambda_I)$ as $\lambda_I\to 0$ from (iv) of 
Proposition \ref{rig:imag_f}. This yields that
\begin{equation*}
    \frac{\eta\lambda_I}{b(9+\lambda_I^2)} \left(3 \tilde{\tau}_j(3-b)-b
  + \tilde{\tau}_j \lambda_I^2\right) \sim \frac{\lambda_I}{4}\left(1 -
  \frac{1}{q}\right) \,.
\end{equation*}
Upon cancelling the factor of $\lambda_I$ and using 
$\tilde{\tau}_j\lambda_I^2\sim {9/(2\eta)}$, we solve for $\tilde{\tau}_j$
in the expression above to get
\begin{equation}\label{genq:shit_m}
   \tilde{\tau}_j \sim \frac{1}{3(3-b)} \left[ b - \frac{9}{2\eta} +
  \frac{9b}{4\eta}\left(1 - \frac{1}{q}\right) \right] \sim
 \frac{1}{3-b} \left[1 + \frac{3}{4\eta}
   \left(1- \frac{3}{q}\right) \right] \,, \quad \mbox{as} \quad b \to 3\,.
\end{equation}
In terms of the original variables ${\mathcal D}_0$, $U_0$ and
$\hat{\tau}_u$ we use (\ref{genq:C}) and (\ref{param:b}) to get
\begin{equation}\label{genq:form}
   \frac{3}{4\eta}=\frac{qU_0}{6\omega} \,, \qquad 
  \hat{\tau}_u = \frac{\alpha}{2} \left( \frac{\kappa_q}{\kappa_3}\right)
   \frac{{\mathcal D}_0}{D^{-}_{0,j}}  \tilde{\tau}_j \,, \qquad
   b-3 \sim 1- \frac{{\mathcal D}_0}{{\mathcal D}^{-}_{0,j}} 
  \quad \mbox{as} \quad  {\mathcal D}_0\to D^{-}_{0,j} \,.
\end{equation}
Upon substituting (\ref{genq:form}) into (\ref{genq:shit_m}) and
(\ref{genq:lami_asy}) we get the limiting Hopf bifurcation threshold
\begin{equation}\label{genq:asy_bto3}
  \hat{\tau}_{u,H}\sim \frac{\alpha}{2} \left( \frac{\kappa_q}{\kappa_3}\right)
   \frac{1}{\left({{\mathcal D}_0/D^{-}_{0.j}} -1\right)} 
 \left( 1 + \frac{qU_0}{6\omega} \left(1 - \frac{3}{q}\right) \right)^{1/2} \,,
 \qquad
  \lambda_{IH}\sim \sqrt{ \frac{{\mathcal D}_0}{D^{-}_{0,j}}-1 }
   \left( \frac{qU_0}{\omega} \right)^{1/2} 
   \left[ 1 + \frac{qU_0}{6\omega}
   \left(1 - \frac{3}{q}\right) \right]^{-1/2} \,,
\end{equation}
as ${\mathcal D}_0\to D^{-}_{0,j}$.  For the special case where $q=3$,
this limiting result for $\lambda_{IH}$ and $\hat{\tau}_{u,H}$ agrees with
that in (\ref{q3:lim_val}) and (\ref{q3:tau_lim}), respectively.

Finally, for the $j$-th mode we will calculate an additional scaling
law for the Hopf bifurcation threshold and the Hopf eigenvalue as
$b\to b_c\equiv {3\eta/(\eta+{3/2})}$ from above, corresponding to the
limit ${\mathcal D}_0\to D^{+}_{0,j}$ from below. We look for a root
$\lambda_I\ll 1$ to ${\mathcal C}_R(\lambda_I)={\mathcal
  F}_R(\lambda_I)$ and use
\begin{equation}\label{genq:kr}
    {\mathcal F}_{R}\sim \frac{1}{2} - k_R \lambda_I^2 + \cdots \,, \quad
\mbox{as} \quad \lambda_I \to 0 \,,
\end{equation}
for some $k_R>0$, together with (\ref{genq:CrCI}) for ${\mathcal
  C}_R(\lambda_I)$, to obtain that
\begin{equation*}
     \frac{\eta}{b(9+\lambda_I^2)} \left[ 9 -3b + \lambda_I^2
   (1+\tilde{\tau}_j b) \right] \sim \frac{1}{2} - k_R \lambda_I^2 \,.
\end{equation*}
Upon isolating $\lambda_I$, we get
\begin{equation}\label{genq:zero}
     \frac{\eta}{b}(9-3b) - \frac{9}{2} = \lambda_I^2 \left( \frac{1}{2}
   - 9\kappa_R - \frac{\eta}{b_c} \left(1+\tilde{\tau}_j b_c\right)\right)\,.
\end{equation}
We then set ${\mathcal C}_I(\lambda_I)={\mathcal F}_I(\lambda_I)$ as 
$\lambda_I\to 0$ using the local behavior (i) of Lemma
\ref{genq:prop_C} for ${\mathcal C}_I(\lambda_I)$ and that in
(iv) of Proposition \ref{rig:imag_f} for ${\mathcal F}_{I}(\lambda_I)$.
This yields that
\begin{equation*}
   \frac{\eta}{9b_c} \left( 3\tilde{\tau}_j (3-b_c) - b_c\right) \sim
  \frac{1}{4}\left( 1 - \frac{1}{q}\right) \,.
\end{equation*}
Upon using $b_c={3\eta/(\eta+{3/2})}$, we solve for $\tilde{\tau}_j$
in this expression to obtain
\begin{equation}\label{genq:z1}
   \tilde{\tau}_j \sim \frac{2\eta}{9} \left[ 1 + \frac{9}{4\eta}
  \left( 1 - \frac{1}{q} \right) \right]\,, \quad \mbox{as} \quad 
b \to b_c \,.
\end{equation}
Upon substituting (\ref{genq:z1}) into (\ref{genq:zero}) and solving
for $\lambda_I$ we obtain, after some algebra, that $\lambda_I\equiv
\lambda_{IH}$ satisfies
\begin{equation}\label{genq:z1_lam}
\lambda_{IH} \sim \sqrt{\frac{27}{2}} \left( \frac{1-{b_c/b}}{3-b_c}\right)^{1/2}
  \left( 9\kappa_R + \frac{\eta}{3} + \frac{2\eta^2}{9} + \frac{\eta}{2}
  \left(1- \frac{1}{q}\right) \right)^{-1/2} \,, \quad \mbox{as} \quad
 b \to b_c \,.
\end{equation}
To write $\tilde{\tau}_j$ in (\ref{genq:z1}) in terms of the original
variables, we use (\ref{genq:form}), together with
${D^{+}_{0,j}/D^{-}_{0,j}}=1+{qU_0/\omega}$, to obtain
\begin{equation}\label{genq:final_lim}
   \hat{\tau}_{u,H}\sim \frac{\alpha\omega}{2q U_0} 
  \left( \frac{\kappa_q}{\kappa_3} \right) \left( 1 + \frac{q U_0}{\omega} 
 \right) \left( 1 + \frac{U_0}{2\omega}(q-1)\right) \,, \qquad
 \mbox{as} \quad {\mathcal D}_0 \to D^{+}_{0,j} \,.
\end{equation}

When $q=3$, the expression in (\ref{genq:final_lim}) agrees with that
obtained in (\ref{q3:tau_lim}) for the explicitly solvable case.  To
determine if (\ref{genq:z1_lam}) yields the limiting Hopf eigenvalue
given in (\ref{q3:lim_val}) when $q=3$, we use ${\mathcal
  F}(i\lambda_I)={3/\left[2(3-i\lambda_I)\right]}$, from
(\ref{eq:exactF_q3}). This yields ${\mathcal F}_R(\lambda_I)\sim
{1/2}-{\lambda_I^2/18}$, which identifies $k_R={1/18}$ in
(\ref{genq:kr}). With this value for $k_R$ and $q=3$, the expression
in the brackets in (\ref{genq:z1_lam}) is a perfect square, and
simplifies to
\begin{equation*}
\lambda_{IH} \sim \sqrt{\frac{27}{2}} \left( 1-\frac{b_c}{b} \right)^{1/2}
 \frac{\left(9/2\right)^{1/2}}{(3-b_c)^{1/2} (\eta+{3/2})} \,.
\end{equation*}
We then use $b_c={3\eta/(\eta+{3/2})}$ together with
$\eta={3\omega/(2U_0)}$ when $q=3$ from (\ref{genq:C}), to get
\begin{equation}\label{genq:up_lim}
\lambda_{IH} \sim \sqrt{\frac{27}{2}} \sqrt{ 1-\frac{b_c}{b}} 
 \left( 1 + \frac{\omega}{U_0}\right)^{-1/2} \,.
\end{equation}
Finally, we use (\ref{param:b}) for $b$,
${D^{+}_{0,j}/D^{-}_{0,j}}=1+3{U_0/\omega}$, and
$\eta={3\omega/(2U_0)}$ to calculate
\begin{equation}
  \frac{b_c}{b}-1 = \frac{\eta}{3(\eta+{3/2})}\left( \frac{{\mathcal D}_0}
   {{\mathcal D}^{-}_{0,j}}+2\right) = \left(\frac{{\mathcal D}_0} {
   D^{+}_{0,j}} -1\right) \left(\frac{ {1+3U_0/\omega}}{3\left(1+ 
{U_0/\omega}\right)}\right) \,.
\end{equation}
Upon substituting this expression into (\ref{genq:up_lim}), we recover
the result for $\lambda_{IH}$ given in (\ref{q3:lim_val}) for
${\mathcal D}_0\to D^{+}_{0,j}$ from below.

\subsection{Parameterization of the Hopf Bifurcation Curve}\label{genq:param}

To compute the Hopf bifurcation threshold numerically for the $j$-th
mode on the range $\dzjm<{\mathcal D}_0<\dzjp$, and to explore the
range ${\mathcal D}_0<\dzjm$ where Proposition \ref{prop:genq:gap} only gives
partial stability information, we now formulate a convenient
parameterization of any Hopf bifurcation curve in the $\hat{\tau}_u$
versus ${\mathcal D}_0$ parameter plane.

We set $\zeta(i\lambda_I)=0$ in (\ref{stab:merom}) to get ${\mathcal
  C}(i\lambda_I)={\mathcal F}(i\lambda_I)$, where ${\mathcal
  C}(i\lambda_I)$ and ${\mathcal F}(i\lambda_I)$ are given in
(\ref{genq:C}) and (\ref{eq:pol-F_R-F_I}) respectively. By taking the
squared modulus of both sides we get
\begin{equation*}
   \frac{\eta^2}{b^2} \left( 1 + \tilde{\tau}_j^2 \lambda_I^2\right)
 \left[ \frac{ (3-b)^2 + \lambda_I^2}{9+\lambda_I^2}\right] = 
  | {\mathcal F}(i\lambda_I)|^{2} \,,
\end{equation*}
which we solve for $\tilde{\tau}_j^2$ to get
\begin{equation}\label{param:mod}
  \tilde{\tau}_j^2 = \frac{1}{\lambda_I^2} \left[ -1 + \frac{b^2}{\eta^2}
     \left( \frac{9+\lambda_I^2}{(3-b)^2+\lambda_I^2} \right) 
  | {\mathcal F}(i\lambda_I)|^{2} \right] \,.
\end{equation}
To derive a second equation for $\tilde{\tau}_j$ we set 
$\mbox{Im}\left[\zeta(i\lambda_I)\right]=0$ to get, upon using 
(\ref{genq:CrCI}) for ${\mathcal C}_I(\lambda_I)$, that
\begin{equation*}
\frac{\eta\lambda_{I}}{b(9+\lambda_{I}^{2})}
\left(3\tilde{\tau}_{j}\left(3-b\right)-b+\tilde{\tau}_{j}\lambda_{I}^{2}\right)
 = {\mathcal F}_{I}(\lambda_I) \,.
\end{equation*}
Upon isolating $\tilde{\tau}_j$ from this expression we obtain that
\begin{equation}\label{param:tau}
   \tilde{\tau}_j = \frac{b}{\eta \lambda_I} \frac{
\left[\eta \lambda_I + \mu {\mathcal F}_{I}(\lambda_I)\right]}{\mu - 3b}
 \,, \qquad \mbox{where} \quad \mu \equiv 9+\lambda_I^2 \,.
\end{equation}
Then, by eliminating $\tilde{\tau}_j$ between (\ref{param:mod}) and
(\ref{param:tau}) we obtain, after some algebra, that $\lambda_I$ must
satisfy the nonlinear algebraic problem ${\mathcal M}(\lambda_I)=0$,
defined by 
\bsub \label{param:all}
\begin{equation}\label{param:nroot}
   {\mathcal M}(\lambda_I)\equiv  (\mu+b^2-6b)\left[
 \left(\eta \lambda_I + \mu {\mathcal F}_I \right)^2 + 
  \frac{\eta^2}{b^2}(\mu-3b)^2 \right]  - \mu (\mu-3b)^2
 |{\mathcal F}|^2  \,,  \qquad \mbox{where} \quad
  \mu\equiv 9 + \lambda_I^2 \,.
\end{equation}
Here we have labeled ${\mathcal F}_I\equiv {\mathcal F}_I(\lambda_I)$
and $|{\mathcal F}|^2 \equiv |{\mathcal F}(i\lambda_I)|^2 =\left[{\mathcal
  F}_R(\lambda_I)\right]^2 + \left[{\mathcal F}_I^{2}(\lambda_I)\right]^2$.
In terms of the original $\hat{\tau}_u$ variable, we have from
(\ref{genq:form}) and ${\mathcal D}_0/D^{-}_{0,j}={9/b}-2$ that the
Hopf bifurcation threshold is
\begin{equation}\label{param:new_tau}
   \hat{\tau}_{u,H} = \frac{\alpha}{2\eta\lambda_I} \left(\frac{\kappa_q}{
  \kappa_3}\right) \left(9 - 2b \right)
\frac{\left[\eta \lambda_I + \mu {\mathcal F}_{I}(\lambda_I)\right]}{\mu - 3b}
 \,, \qquad \mbox{where} \quad \mu \equiv 9+\lambda_I^2 \,.
\end{equation}

This parameterization (\ref{param:new_tau}) and (\ref{param:nroot}) is
used numerically as follows: For a given $\eta$ and $q$, we will show
numerically that (\ref{param:nroot}) has a unique root
$\lambda_I=\lambda_{IH}(b)$ on the range $b_c<b<3$, where $b_c\equiv
{3\eta/(\eta+{3/2})}$. This corresponds to the range
$D^{-}_{0,j}<{\mathcal D}_0<D^{+}_{0,j}$, via the mapping
\begin{equation}\label{param:last}
    {\mathcal D}_0=D^{-}_{0,j} \left( \frac{9}{b}-2 \right)\,.
\end{equation}
\esub
To compute this root $\lambda_{IH}(b)$ by applying a Newton solver to
(\ref{param:nroot}), we must compute the functions ${\mathcal
  F}_R(\lambda_I)$ and ${\mathcal F}_I(\lambda_I)$, as defined in
(\ref{eq:pol-F_R-F_I}), using a BVP solver, for any $\lambda_I>0$.  A
key simplifying feature of this parameterization is that the root
$\lambda_{IH}(b)$ can be used {\em for all} of the modes
$j=1,\ldots,K-1$, as the range of ${\mathcal D}_0$ for the specific
mode is only identified at the last step (\ref{param:last}). A similar
universality feature of the Hopf bifurcation curve was exploited in
(\ref{q3:hopf}) for the explicitly solvable case $q=3$. We further
remark that for the explicitly solvable case $q=3$ where ${\mathcal
  F}(i\lambda_I)={3/\left[2(3-i\lambda_I)\right]}$, some lengthy but
straightforward algebra shows that the root of (\ref{param:nroot}) is
given explicitly by (\ref{q3:dlami}).

On the range $b_c<b<3$, for which Proposition \ref{prop:genq:hopf}
ensures that a Hopf bifurcation exists, the Hopf threshold
$\hat{\tau}_{u,Hj}$ for a specific mode $j=1,\ldots,K-1$ is given
uniquely by simply evaluating (\ref{param:new_tau}) at the unique root
$\lambda_I=\lambda_{IH}(b)$ of ${\mathcal M}(\lambda_I)=0$. We remark
that on the range $b<3$ we have $\mu-3b>0$ for all $\lambda_I>0$, so
that (\ref{param:new_tau}) is well-defined. To establish that
(\ref{param:nroot}) has a root on $b_c<b<3$, we set $\lambda_I=0$ in
(\ref{param:nroot}) and use ${\mathcal F}_I(0)=0$, $|{\mathcal
  F}(0)|^2={1/4}$, and $\mu=9$, to get that
\begin{equation*}
   {\mathcal M}(0)= (9-3b)^2\left[ \frac{\eta^2}{b^2}(b-3)^2-\frac{9}{4}
\right] \,.
\end{equation*}
From this expression, we conclude that ${\mathcal M}(0)=0$ at
$b=b_c\equiv {3\eta/(\eta+{3/2})}$ and $b=3$, and that ${\mathcal
  M}(0)<0$ on the interval $b_c< b< b$. Since ${\mathcal
  M}(\lambda_I)\to +\infty$ as $\lambda_I\to +\infty$, we conclude
that there exists a $\lambda_{IH}>0$ for which ${\mathcal
  M}(\lambda_{IH})=0$. 

\begin{figure}[htbp]
\centering
\includegraphics[width=9cm,height=5.0cm]{figs/nonlin_hopf_h.eps}
\includegraphics[width=9cm,height=5.0cm]{figs/nonlin_hopf_nh.eps}
\caption{\label{fig:nonlin} Plots of ${{\mathcal M}(\lambda_I)/\mu}$,
  where $\mu\equiv \lambda_I^2+9$, versus $\lambda_I$ for $S=6$,
  $\gamma=2$, $\alpha=1$, $U_0=4$, and $q=2$. In the left panel, where
  $b=1.5,2.0,2.5$, which satisfies $b_c<b<3$, there is a unique root
  to ${\mathcal M}(\lambda_I)=0$, yielding the Hopf eigenvalue. In the
  right panel, where $b=3.2,3.6,4.0$, there is no root to ${\mathcal
    M}(\lambda_I)=0$, and hence no Hopf eigenvalue. For our choice
  $U_0=4$, where $U_0>{3\omega/q}$, Proposition \ref{prop:genq:gap}
  gives no information regarding Hopf bifurcations on the range
  $b>3$.}
\end{figure}

To examine numerically the uniqueness of the Hopf threshold on the
range $b_c<b<3$, we take the same parameter set $S=6$, $\gamma=2$, and
$\alpha=1$, as used in \S \ref{sec:stab_q3} for the $q=3$ case. We
choose $q=2$ and $U_0=4$ for which $\eta\approx 1.286$. For $b=1.5$,
$b=2.0$, and $b=2.5$, in the left panel of Fig.~\ref{fig:nonlin} we
plot ${{\mathcal M}(\lambda_I)/\mu}$ versus $\lambda_I$ for $U_0=4$
showing numerically the existence of a unique root to ${\mathcal
  M}(\lambda_I)=0$, and consequently a unique Hopf bifurcation value
of $\hat{\tau}_{uH,j}$ for the $j$-th mode.

In contrast, suppose that $b>3$. To investigate whether a Hopf
bifurcation is possible for $3<b<{9/2}$, we must determine whether
there is a root of ${\mathcal M}(\lambda_I)=0$ on the range $\mu\equiv
9+\lambda_I^2>3b$ for which $\hat{\tau}_{u,H}>0$ (see
(\ref{param:new_tau})). From (\ref{param:nroot}), we first observe
that ${\mathcal M}(\lambda_I)=(b-3)^2\left(\eta \sqrt{3b-9} + 3b
{\mathcal F}_I(3b) \right)^2 >0$ when $\mu=3b$ and that ${\mathcal
  M}(\lambda_I)\to \infty$ as $\lambda_I\to \infty$. Therefore, if
such a root exists, ${\mathcal M}(\lambda_I)$ cannot be monotone on
$\mu>3b$. We study this issue numerically in the right panel of
Fig.~\ref{fig:nonlin} where we plot ${{\mathcal M}(\lambda_I)/\mu}$
versus $\lambda_I$ for $b=3.2$, $b=3.6$, and $b=4.0$, on the range
$\mu>3b$, for our parameter set $S=6$, $\gamma=2$, $\alpha=1$,
$U_0=4$, and $q=2$. Numerically, we find that there is no root to
${\mathcal M}(\lambda_I)=0$ when $\mu>3b$. For $U_0=4$ and $q=2$, we
remark that $U_0>{3\omega/q}$, and so Proposition \ref{prop:genq:gap}
gives no information regarding Hopf bifurcations on the range
$b>3$. Further computations (not shown) with ${\mathcal M}(\lambda_I)$
for $b>3$ suggest that there is never a root to ${\mathcal
  M}(\lambda_I)=0$ on $\mu>3b$ for any $U_0<U_{0,\max}$. Based on this
numerical evidence, we conjecture that no Hopf bifurcations can occur
when ${\mathcal D}_0<D^{-}_{0,j}$ for any $q>1$ and $U_0<U_{0,\max}$.

\begin{figure}[htbp]
\centering
\includegraphics[width=9cm,height=5.0cm]{figs/genq_2_U02.eps}
\includegraphics[width=9cm,height=5.0cm]{figs/genq_2_U04.eps}
\caption{\label{fig:genq::hopf_tau_2} Plot of the Hopf bifurcation
  threshold $\hat{\tau}_{uH_1}$ versus ${\mathcal D}_0$ for $q=2$
  (solid), $q=3$ (dashed), and $q=4$ (dot-dashed) on the range
  ${{\mathcal D}_{0,c}/(1+{qU_0/\omega})}<{\mathcal D}_0<{\mathcal
    D}_{0c}$ for $K=2$, $S=6$, $\gamma=2$, $\alpha=1$, and with
  $U_0=2$ (left panel) and $U_0=4$ (right panel). Here ${\mathcal
    D}_{0,c}$ is the competition stability threshold defined in
  (\ref{stab:zero_d0min}), which depends on $q$. The lower bound
  ${{\mathcal D}_{0,c}/(1+{qU_0/\omega})}$ is independent of $q$. The
  two-hotspot steady-state is linearly stable for ${\mathcal
    D}_0<{{\mathcal D}_{0,c}/(1+{qU_0/\omega})}$ (shaded region) as
  well as under the Hopf bifurcation curve. For ${\mathcal
    D}_0>{\mathcal D}_{0,c}$ the hotspot pattern is unstable for all
  $\hat{\tau}_u$. We observe that the interval in ${\mathcal D}_0$
  where an oscillatory instability of the hotspot amplitudes can occur
  increases with $q$. The Hopf bifurcation threshold
  $\hat{\tau}_{uH,1}$ also increases with $q$.}
\end{figure}

Next, we use our parameterization to compute the Hopf bifurcation
threshold for $\hat{\tau}_u$ on the range $D^{-}_{0,j}<{\mathcal
  D}_0<D^{+}_{0,j}$, and plot the region of linear stability for $q=2$
and $q=4$ in order to compare with our previous results in \S
\ref{sec:stab_q3} for $q=3$.

For our parameter set, and for a two-hotspot solution, in
Fig.~\ref{fig:genq::hopf_tau_2} we plot the linear stability phase
diagram in the $\hat{\tau}_u$ versus ${\mathcal D}_0$ plane when
either $U_0=2$ (left panel) and for $U_0=4$ (right panel). In both
panels we compare the linear stability thresholds for $q=2,3,4$. In
the left and right panels of Fig.~\ref{fig:genq::hopf_tau_2}, the
two-hotspot steady-state is linearly stable in the shaded region,
which is the same for each $q$, and in the region below the Hopf
bifurcation threshold for the given $q$. Since the competition
instability threshold ${\mathcal D}_{0,c}$ increases with $q$, as was
shown in \S \ref{sec:qual_comp_d}, the interval in ${\mathcal D}_0$
where an oscillatory instability of the hotspot amplitudes can occur
increases with $q$. We further observe from
Fig.~\ref{fig:genq::hopf_tau_2} that the Hopf bifurcation threshold
value of $\hat{\tau}_u$ increases with $q$, and when $U_0=4$ the Hopf
bifurcation threshold is not monotone in ${\mathcal D}_0$ when either
$q=3$ or $q=4$ (recall the right panel of Fig.~\ref{fig:hopf_tau_2}
for the $q=3$ case). These results are discussed qualitatively in
\S \ref{sec:numerics_qn3}.


For a three-hotspot pattern similar results for the linear stability
region in the $\hat{\tau}_u$ versus ${\mathcal D}_0$ plane are shown
in Fig.~\ref{fig:genq:k3:hopf_tau_U02} for $U_0=2$ and in
Fig.~\ref{fig:genq:k3:hopf_tau_U04} for $U_0=4$. Results for $q=2,3,4$
are shown in the three subpanels of these figures. We observe that the
minimal Hopf threshold value for $\hat{\tau}_u$ increases with $q$,
and this increase is more pronounced for $U_0=4$ than for $U_0=2$. For
$U_0=4$, we observe from Fig.~\ref{fig:genq:k3:hopf_tau_U04}, as
similar to that analyzed for the explicitly solvable case $q=3$ in \S
\ref{sect:q3_phase}, that when $q=4$ the minimal Hopf threshold value
for $\hat{\tau}_u$ switches between two modes at some critical ${\mathcal
  D}_0$.

\begin{figure}[htbp]
\centering
\includegraphics[width=0.32\textwidth,height=5.0cm]{figs/genq_3_q2_u02.eps}
\includegraphics[width=0.32\textwidth,height=5.0cm]{figs/genq_3_q3_u02.eps}
\includegraphics[width=0.32\textwidth,height=5.0cm]{figs/genq_3_q4_u02.eps}
\caption{\label{fig:genq:k3:hopf_tau_U02} Linear stability (shaded)
  region in the $\hat{\tau}_u$ versus ${\mathcal D}_0$ plane for
  $K=3$, $S=6$, $U_0=2$, $\gamma=2$ and $\alpha=1$, and for $q=2$ (left panel),
  $q=3$ (middle panel), $q=4$ (right panel). The solid and dashed
  curves are the Hopf bifurcation boundaries for the
  (sign-alternating) $j=2$ mode and the $j=1$ mode, respectively. In
  each case, the minimal Hopf boundary threshold for $\hat{\tau}_u$ is
  determined by the sign-alternating $j=2$ mode. Observe that
  $\hat{\tau}_{u,H}$ increases as $q$ increases.}
\end{figure}

\begin{figure}[htbp]
\centering
\includegraphics[width=0.32\textwidth,height=5.0cm]{figs/genq_3_q2_u04.eps}
\includegraphics[width=0.32\textwidth,height=5.0cm]{figs/genq_3_q3_u04.eps}
\includegraphics[width=0.32\textwidth,height=5.0cm]{figs/genq_3_q4_u04.eps}
\caption{\label{fig:genq:k3:hopf_tau_U04} Same plot and parameters as
  in Fig.~\ref{fig:genq:k3:hopf_tau_U02} except that $U_0=4$: $q=2$
  (left panel), $q=3$ (middle panel), and $q=4$ (right panel). The
  shaded region is the region of linear stability. The solid and
  dashed curves are the Hopf bifurcation boundaries for the
  (sign-alternating) $j=2$ mode and the $j=1$ mode, respectively. The
  minimal Hopf bifurcation threshold switches between the two modes
  only for $q=3$ and $q=4$. As $q$ increases the Hopf bifurcation
  threshold increases significantly (see the different vertical scales
  in the subfigures). The horizontal edge of the stability region is the
  competition stability threshold ${\mathcal D}_{0,c}$.}
\end{figure}

Overall, our numerical results for the linear stability region for
$q=2$ and $q=4$, as computed from our parameterization
(\ref{param:all}), are qualitatively similar to those obtained from our
detailed analysis in \S \ref{sec:stab_q3} for the explicitly solvable
case $q=3$.

\subsection{Comparison of Linear Stability Theory with PDE Simulations: 
$q\neq 3$}\label{sec:numerics_qn3}

\highblue{To be written once I get the full numerics for the PDE system
working.}

% left figure: eps^2 D_p=0.20, D_0=0.5
% middle figure: eps^2 D_p=0.32, D_0=0.8
% right figure: eps^2 D_p=0.4, D_0=0.5
%


Here we discuss our main stability result qualitatively and we validate
our Hopf bifurcation threshold values with full numerical solutions of
the PDE system (\ref{eq:pol-main}).

\highblue{This section is to be written once I get the full numerics working.
I might use some of the blurb below.
This analysis reveals a qualitatively novel phenomenon in the context
of the study of the stability of spike patterns to reaction-diffusion
systems.  In particular, for a two-hotspot equilibrium, then $j=1$ is
the only mode of oscillation, and our theory predicts the possibility
of an \emph{asynchronous} Hopf bifurcation so that the amplitudes of
the two crime hotspots begin to exhibit temporal anti-phase
oscillations. In terms of the urban crime model, this means that when
police patrols with a certain specific diffusivity relative to the
criminals (determined by $\tilde{\tau}_{j,{\textrm Hopf}}$), one
observes an interesting picture that the police concentration is
drifting to and fro from the hotspots without annihilating any of
them. However, if the police patrol diffusivity exceeds such a
threshold, then one of the hotspot will dissipate due to an
oscillatory instability.  Such a qualitative behavior in the possible
types of detstabilization of localized spike patterns was not observed
in other well-studied reaction-diffusion systems exhibiting similar
concentration phenomena, such as the Gray-Scott, Gierer-Meinhardt and
Schnakenburg models.}




\setcounter{equation}{0}
\setcounter{section}{7}
\section{Discussion}

We have used a combination of matched asymptotic expansions and the
analysis of nonlocal eigenvalue problems (NLEP) to study the existence
and linear stability of localized hotspot steady-state solutions of
the three-component RD system (\ref{eq:crs-main}) in the limit
$\epsilon\to 0$ with $D={{\mathcal D}_0/\epsilon^2}$ on the 1-D domain
$0\leq x\leq S$. We have investigated how the multiple hotspot
steady-states and their linear stability properties depend on the
total police deployment $U_0>0$, the patrol focus parameter $q>1$, and
the police diffusivity $D_p\equiv {D_0/\eps^2 \tau_u}$, where
$\tau_u=\epsilon^{q-3}\hat{\tau}_u$ with $\hat{\tau}_u={\mathcal
  O}(1)$.

In the limit $\epsilon\to 0$, our formal asymptotic construction of
hotspot steady-states for (\ref{eq:crs-main}) with a common hotspot
amplitude, and with $q>1$, has shown that they exist only when
$U_0<U_{0,\max}\equiv S(\gamma-\alpha)$ (see Proposition
\ref{thm:main_eq}. The derivation of the nonlocal eigenvalue problem (NLEP)
characterizing the the linear stability properties of these localized
steady-states was based on combining Floquet theory with the derivation of
appropriate jump conditions for the components of the eigenfunction across
a hotspot region. Although the NLEP has two nonlocal terms, we have shown
how to reduce it to a more traditional form with only one nonlocal term
for which some analytical results are available (cf. \cite{wei_rev}).

Our NLEP linear stability analysis has revealed that a steady-state
with a single hotspot is always linearly stable, and that a
steady-state with $K\geq 2$ hotspots is always linearly stable to
synchronous perturbations in the amplitudes of the hotspots. This
latter conclusion is in marked contrast to previous NLEP analyses of
spike amplitude instabilities for two-component RD systems in the
large diffusivity ration limit whereby the dominant temporal
oscillatory instability in the spike amplitudes resulting froma Hopf
bifurcation is always due to the synchronous mode (cf.~\cite{mjww_1},
\cite{kww_gs}).

For $K\geq 2$ hotspots, and for a given $U_0>0$ and a fixed $q\in
\lbrace{2,3,4\rbrace}$, our NLEP analysis has provided phase diagrams in
the $\hat{\tau}_u$ versus ${\mathcal D}_0$ characterizing the linear
stability of the hotspot steady-state to asynchronous perturbations in
the hotspot amplitudes. 

For the special case $q=3$, and for $K\geq 2$, we have shown that the
discrete spectrum of the NLEP can be reduced to the study of a family
of $K-1$ quadratic equations in the eigenvalue parameter.  Such
explicitly solvable NLEP problems also appear in
\cite{kww_crime,mtw_enlep,nec}. As a result of this simplification, we
have pro



\begin{itemize}
  \item Special case $q=3$ (family of quadratics) as compared to general $q$
   (argument principle: rigorous) (parameterization: numerics).
  \item explain three key regimes
  \item derivation of the NLEP is more challenging; leads to two nonlocal
       terms, one that can be eliminated self-consistently.
\end{itemize}


\cite{kww_crime,mtw_enlep,nec}.  The general case was then studied
using the argument principle to count the number of unstable
eigenvalues in the right half plane.  This procedure was first
developed to study the stability of steady-state spike patterns for
the Gierer-Meinhardt model (cf.~\cite{mjww_1}) and now has a rather
large body of literature (see \cite{kww_crime} and the references
therein). Our conclusions from the explicitly solvable case $q=3$ are
considerably stronger than those for the non-explicitly solvable case
$q\neq3$. 

In particular, when $q=3$, two thresholds ${\mathcal
  D}_{0,{\textrm lower}}$ and ${\mathcal D}_{0,{\textrm upper}}$ given
 were determined so that
the a multiple-hotspot pattern is stable when ${\mathcal
  D}_{0}<{\mathcal D}_{0,{\textrm lower}}$ and unstable due to a
competition instability when ${\mathcal D}_{0}>{\mathcal
  D}_{0,{\textrm upper}}$. Moreover, an explicit formula for the
existence of Hopf bifurcation $\tau_{u}=\tau_{{\textrm Hopf}}$ when
${\mathcal D}_{0,{\textrm lower}}<{\mathcal D}_{0}<{\mathcal
  D}_{0,{\textrm upper}}$ was given in
In contrast to the absence of a
Hopf bifurcation for the basic crime model with no police
intervention, as discovered in \cite{kww_crime}, the window of
existence for a Hopf bifurcation given by $({\mathcal D}_{0,{\textrm
    lower}},{\mathcal D}_{0,{\textrm upper}})$ vanishes exactly when
$U_{0}=0$. In other words, the third component of the PDE system
(\ref{eq:pol-main}),
modeling the police interaction, is essential to inducing the
possibility of oscillations. Moreover, unlike the case of the
Gray-Scott and Gierer-Meinhardt models studied in \cite{kww_gs} and
\cite{mjww_1}, where synchronous oscillatory instabilities of the
spike amplitudes robustly occur and are the dominant instability, our
three-component system exhibits asynchronous oscillatory
instabilities. These asynchronous, anti-phase, oscillations of the
spike amplitudes have the qualitative interpretation that, for a range
of police diffusivities, the police presence is only able to mitigate
the amplitude of certain hotspots at the expense of the growth of
other hotspots in different spatial regions.

However, when $q\neq3$, we had difficulty in analytically proving
results as strong as for the case $q=3$. In particular, we were not
able to prove, without assuming further conditions, that a multiple
hotspot pattern is stable when the rescaled criminal diffusivity
${\mathcal D}_{0}$ is below the same lower threshold defined earlier
in the $q=3$ case. One possibility is that the definition of the lower
threshold should be revised and should change with $q$. When
${\mathcal D}_{0}$ is between the lower and upper thresholds, we were
able to prove the existence of a Hopf bifurcation, but we cannot show
uniqueness of the critical Hopf bifurcation value in $\tau_{u}$. These
are interesting open problems that warrant further study. Most
importantly, we would like to investigate what are the mathematical
relationships between the explicitly solvable case $q=3$ and the
non-explicitly solvable case $q\neq3$, so that the strong results from
the explicitly solvable case can potentially carry over to the general
case.

A reason for considering integral values of $q$ only in the range
$\{2,3,4\}$ is that we assumed in the course of deriving the NLEP
that $\tau_{u}\ll {\mathcal O}(\epsilon^{-2})$
(see the calculations leading to \ref{eq:pol-tau_u-assumption}). Then,
for the parameter
\[
\tilde{\tau}_{j}=\tau_{u}{\mathcal O}(\epsilon^{q-3})\,,
\]
to be ${\mathcal O}(1)$, we must have that $\tau_{u}={\mathcal
  O}(\epsilon^{3-q})$. This establishes that $q<5$ is required to
satisfy the assumption that $\tau_{u}\ll {\mathcal O}(\epsilon^{-2})$.
If $q=1$, then the ODE for the perturbation $\eta$ at
(\ref{eq:pol-eta-ode}) will be changed to
\[
{\mathcal D}_{0}\alpha^{q}\eta_{xx}=\hat{\tau}_u\lambda\alpha^{q}\eta\,,
\]
where $\tau_u=\epsilon^{-2}\hat{\tau}_u$. This will result in a
different problem to be solved for $\eta(x)$, and consequently a
different value for $\eta(0)$. Ultimately, this leads to a different
NLEP that requires a separate analysis.


\subsection{Open Problems and Future Directions}

With regards to our police model, with simple police interaction,
studied in this chapter, it would be interesting to consider the more
challenging $D={\mathcal O}(1)$ regime. One key question would be to
investigate whether the police presence can eliminate the peak
insertion behavior that was found for the basic crime model to lead to
the nucleation of new spikes of criminal activity. In this direction
it would be interesting to determine the influence of the police
presence on the global bifurcation of multiple hotspot steady-state
solutions.

This suggests that there should also be discrete spectra of the full
problem that are near zero as $\epsilon\to 0$. These are the ``small''
eigenvalues that are related to translational instabilities. Our
analysis of the NLEP characterizes only those eigenvalues that are
${\mathcal O}(1)$ as $\epsilon\to 0$, which can lead to ${\mathcal
  O}(1)$ time-scale instabilities.

A second interesting direction would be to study the effect of police
presence on crime patterns when the police interaction is modeled by
the predator-prey dynamics case $I(U,\rho)=U\rho$. Preliminary results
suggest that the NLEP will now have three nonlocal terms, which makes
a detailed stability analysis very challenging. However, the
determination of the competition instability threshold, corresponding
to the zero-eigenvalue crossing, should be readily amenable to
analysis.

A third direction would be to consider spatial patterns in more than a
simple 1-D spatial context. In this context, it would be interesting
to extend the 2-D stability results in \cite{kww_crime} for the basic
crime model, to study the existence of stability of crime hotspots in
2-D domains in the presence of police. In particular, we would like to
investigate the stability and dynamics of 2-D hotspots, allowing for
either of our two different models of police intervention.

\appendix

\section{The Continuum Limit of the Agent-Based Model} 

\section*{Acknowledgments}
M.~J.~Ward was supported by the NSERC Discovery Grant 81541. We gratefully
acknowledge helpful discussions with Theodore Kolokolnikov and Juncheng
Wei on the NLEP analysis.

\begin{thebibliography}{99}

\bibitem{bn} H.~Berestycki, J.-P.~Nadal, \textit{Self-organised
critical hot spots of criminal activity}, Europ. J. Appl. Math.,
\textbf{21}(4-5), (2010), pp.~371--399.

\bibitem{bww} H.~Berestycki, J.~Wei, M.~Winter, \textit{Existence
of symmetric and asymmetric spikes for a crime hotspot model}, SIAM
J. Math. Anal., \textbf{46}(1), (2014), pp.~691--719.

\bibitem{bb} P.~L.~Brantingham, P.~J.~Brantingham, \textit{Crime
patterns}, McMillan, (1987).

\bibitem{ccm} R.~Cantrell, C.~Cosner, R.~Manasevich, \textit{Global
  bifurcation of solutions for crime modeling equations}, SIAM
  J. Math. Anal., \textbf{44}(3), (2012), pp.~1340--1358.

\bibitem{discover} Discover, Science for the Curious (2010), {[}Fight
Crime with Mathematics ranks in Top 100 Stories in 2010{]}. Retrieved
from http://discovermagazine.com/2011/jan-feb/60.

\bibitem{dgk_0} A.~Doelman, R.~A.~Gardner, T.~J.~Kaper, \textit{Large
  stable pulse solutions in reaction-diffusion equations}, Indiana
  U. Math. J., \textbf{50}(1), (2001), pp.~443--507.

\bibitem{dp} A.~Doelman, H.~van der Ploeg, \textit{Homoclinic
  stripe patterns}, SIAM J. Appl. Dyn. Systems, \textbf{1}(1), (2002),
  pp.~65-104.

\bibitem{dOr-rev} M.~R.~D'Orsogna, M.~Perc, \textit{Statistical
  physics of crime: A review}, Physics of Life Reviews, \textbf{12},
  (2015), pp.~1-21

\bibitem{GWY} Y.~Gu, Q.~Wang, G.~Yi, \textit{Stationary
patterns and their selection mechanism of urban crime models with
heterogeneous near-repeat victimization effect}, Europ. J. Appl.
Math., {\bf 28}(1), (2017), pp.~141--178.

\bibitem{iww} D.~Iron, M.~J.~Ward, J.~Wei, \textit{The
stability of spike solutions to the one-dimensional Gierer-Meinhardt
model}, Physica D, \textbf{150}(1-2), (2001), pp.~25--62.

\bibitem{jb} S.~Johnson, K. Bower, \textit{Domestic burglary
repeats and space-time clusters: The dimensions of risk}, Europ. J.
of Criminology, \textbf{2}, (2005), pp.~67--92.

\bibitem{jbc} P.~A.~Jones, P.~J.~Brantingham, L.~Chayes,
  \textit{Statistical models of criminal behavior: The effect of law
    enforcement actions}, Math. Models. Meth. Appl. Sci., \textbf{20},
  Suppl., (2010), pp.~1397--1423.

\bibitem{kww_1} T.~Kolokolnikov, M.~J.~Ward, J.~Wei, \textit{The
  existence and stability of spike equilibria in the one-dimensional
  Gray-Scott model: The low feed-rate regime}, Studies in Appl. Math.,
  \textbf{115}(1), (2005), pp.~21--71.

\bibitem{kww_crime} T.~Kolokolnikov, M.~J.~Ward, J.~Wei, \textit{The
  stability of steady-state hot-spot patterns for a reaction-diffusion
  model of urban crime}, DCDS-B, \textbf{19}(5), (2014),
  p.~1373--1410.

\bibitem{mk_mesa} R.~McKay, T.~Kolokolnikov, \textit{Stability
  transitions and dynamics of localized patterns near the shadow limit
  of reaction-diffusion systems}, DCDS-B, \textbf{17}(1), (2012)
  pp.~191--220.

\bibitem{kww_gs} T.~Kolokolnikov, M.~J.~Ward, J.~Wei, \textit{The 
   existence and stability of spike equilibria in the one-dimensional 
   Gray-Scott model: The low feed rate regime}, Studies in Appl.
   Math, \textbf{115}(1), (2005), pp.~21--71.

\bibitem{kw_xdiff} T.~Kolokolnikov, J.~Wei, \textit{Stability
of spiky solutions in a competition model with cross-diffusion}, SIAM
J. Appl. Math., \textbf{71}(4), (2011), pp. 1428--1457.

\bibitem{lf} D.~J.~B.~Lloyd, H.~O'Farrell, \textit{On
  localised hotspots of an urban crime model}, Physica D,
  \textbf{253}, (2013), pp.~23--39.

\bibitem{mtw_enlep} I.~Moyles, W.-H.~Tse, M.~J.~Ward,
  \textit{Explicitly solvable nonlocal eigenvalue problems and the
    stability of localized stripes in reaction-diffusion systems},
  Studies in Appl. Math., \textbf{136}(1), (2016), pp.~89--136.

\bibitem{nec} Y.~Nec, M.~J.~Ward, \textit{An explicitly solvable
nonlocal eigenvalue problem and the stability of a spike for a class
of reaction-diffusion system with sub-diffusion}, Math. Model. of
Nat. Phenom., \textbf{8}(2), (2013), pp. 55--87.

\bibitem{painter-hillen} K.~Painter, T.~Hillen,
  \textit{Spatio-temporal chaos in a chemotaxis model}, Physica D,
  \textbf{240}, (2011), pp.~363--375.

\bibitem{floq-ref} H.~van der Ploeg, A.~Doelman, \textit{
  Stability of spatially periodic pulse patterns in a class of
  singularly perturbed reaction-diffusion equations}, Indiana
  Univ. Math. J., \textbf{54}(5), (2005), pp.~1219--1301

\bibitem{pitcher} A.~B.~Pitcher, \textit{Adding police to a
  mathematical model of burglary}, Europ. J. Appl. Math.,
  \textbf{21}(4-5), (2010), pp.~401--419.

\bibitem{rade} J.~D.~M.~Rademacher, \textit{First and second order
  semi-strong interaction in reaction-diffusion systems}, SIAM
  J. Appl. Dyn. Syst., \textbf{12}(1), (2013), pp.~175--203.

\bibitem{raidsonline} RAIDS Online, \textit{Crime density maps for West
  Vancouver, B.C., Canada and Santa Clara-Sunnyville, California, US},
  BAIR Analytics, (2015), Inc. Retrieved from http://raidsonline.com/.

\bibitem{rick} L.~Ricketson, \textit{A continuum model of residential
  burglary incorporating law enforcement}, unpublished,
  (2011). Retrieved from
  http://cims.nyu.edu/\textasciitilde{}lfr224/writeup.pdf.

\bibitem{rb} N.~Rodriguez, A.~Bertozzi, \textit{Local existence and
  uniqueness of solutions to a PDE model for criminal behavior}, M3AS
  (special issue on Mathematics and Complexity in Human and Life
  Sciences), \textbf{20}(1), (2010), pp.~1425--1457.

\bibitem{RRW} I.~Rozada, S.~Ruuth, M.~J.~Ward, \textit{The stability of
 localized spot patterns for the Brusselator on the sphere}, SIAM J. Appl. 
Dyn. Sys., \textbf{13}(1), (2014), pp.~564--627.

\bibitem{s_1} M.~B.~Short, M.~R.~D'Orsogna, V.~B.~Pasour, G.~E.~Tita,
  P.~J.~Brantingham, A.~L.~Bertozzi, L.~B.~Chayes, \textit{A
    statistical model of criminal behavior}, Math. Models. Meth. Appl.
  Sci., \textbf{18}, Suppl., (2008), pp.~1249--1267.

\bibitem{s_2} M.~B.~Short, A.~L.~Bertozzi, P.~J.~Brantingham
  \textit{Nonlinear patterns in urban crime - hotspots, bifurcations,
    and suppression}, SIAM J. Appl. Dyn. Sys., \textbf{9}(2), (2010),
  pp.~462--483.

\bibitem{s_3} M.~B.~Short, P.~J.~Brantingham, A.~L.~Bertozzi,
  G.~E.~Tita (2010), \textit{Dissipation and displacement of hotpsots
    in reaction-diffusion models of crime}, Proc. Nat. Acad. Sci.,
  \textbf{107}(9), pp.~3961-3965.

\bibitem{swr} W.~Sun, M.~J.~Ward, R.~Russell, \textit{The slow
  dynamics of two-spike solutions for the Gray-Scott and
  Gierer-Meinhardt systems: competition and oscillatory
  instabilities}, SIAM J. Appl.  Dyn. Syst., \textbf{4}(4), (2005),
  pp.~904--953.

\bibitem{tw} W.-H.~Tse, M.~J.~Ward, \textit{Hotspot formation and
  dynamics for a continuum model of urban crime},
  Europ. J. Appl. Math., \textbf{27}(3), (2016), pp.~583--624.

\bibitem{mjww_1} M.~J.~Ward, J.~Wei, \textit{Hopf bifurcations and
  oscillatory instabilities of spike solutions for the one-dimensional
  Gierer-Meinhardt model}, J. Nonlinear Sci., \textbf{13}(2), (2003),
  pp.~209--264.

\bibitem{wei-1} J.~Wei, \textit{On single interior spike solutions of
  the Gierer-Meinhardt system: uniqueness and spectrum
  estimates}, Europ. J. Appl. Math., \textbf{10}(4), (1999),
  pp.~353--378.

\bibitem{wei_rev} J.~Wei, \textit{Existence and stability
of spikes for the Gierer-Meinhardt system}, book chapter in \textit{Handbook
of Differential Equations, Stationary Partial Differential Equations},
Vol. 5 (M. Chipot ed.), Elsevier, (2008), pp.~489--581.

\bibitem{wk} J.~Q.~Wilson, G.~L.~Kelling, \textit{Broken windows and
  police and neighborhood safety}, Atlantic Mon., \textbf{249},
  (1998), pp.~29--38.

\bibitem{zipkin} J.~R.~Zipkin, M.~B.~Short, A.~L.~Bertozzi,
  \textit{Cops on the dots in a mathematical model of urban crime and
    police response}, DCDS-B, \textbf{19}(5), (2014), pp.~1479--1506.

\end{thebibliography}

\end{document}


